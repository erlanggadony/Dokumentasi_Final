\chapter{Format Surat \LaTeX}
\label{lamp:F}

\begin{lstlisting}[language=tex,basicstyle=
\tiny,caption=Format untuk surat keterangan beasiswa]
	\documentclass[12pt]{letter}
\usepackage[legalpaper, top=53mm,bottom=15mm,left=35mm,right=18mm]{geometry}
\usepackage{mailmerge}

\begin{document}

\mailfields{noSurat,nama,prodi,npm,semester,thnAkademik,penyediabeasiswa,tanggal}

\mailrepeat{

	\begin{center}
	\underline{SURAT KETERANGAN}\\
	{\footnotesize  \field{noSurat}}
	\end{center}
	\vspace{0.75cm}
	Bersama ini kami menerangkan bahwa mahasiswa Fakultas teknologi Informasi dan Sains (FTIS) berikut : \
	\begin{quote}
		Nama\hspace{2cm}: \field{nama}\

		Jurusan\hspace{1.6cm}: \field{prodi}\

		NPM\hspace{2.1cm}: \field{npm}\
	\end{quote}
	BENAR tidak menerima beasiswa dari Universitas Katolik Parahyangan pada Semester \field{semester} Tahun Akademik \field{thnAkademik}.\

	Surat keterangan ini dibuat sebagai kelengkapan bagi ybs. untuk mengajukan diri sebagai penerima beasiswa perusahaan \field{penyediabeasiswa}.
		\vspace{0.5cm}
		\begin{flushright}
			Bandung, \field{tanggal}

			\vspace{2cm}




		\underline{Flaviana Catherine, S.Si, M.T}\\
		WD III FTIS - UNPAR
		\end{flushright}

		\thispagestyle{empty}
			\newpage

}

\end{document}

\end{lstlisting}

\begin{lstlisting}[language=tex,basicstyle=
\tiny,caption=Format untuk surat keterangan mahasiswa aktif]
	\documentclass[12pt]{letter}
%\usepackage[legalpaper, top=55mm,bottom=15mm,left=35mm,right=18mm]{geometry}
\usepackage{mailmerge}

\begin{document}

\mailfields{noSurat,nama,NPM,prodi,tanggalLahir,alamat,semester,wdI,tanggal}

\mailrepeat{

	\begin{center}
	\underline{SURAT KETERANGAN}\\
	{\footnotesize \field{noSurat}}
	\end{center}
	\vspace{0.75cm}
	Fakultas teknologi Informasi dan Sains Universitas Katolik Parahyangan Bandung, menerangkan :


		\begin{quote}
			Nama \hspace{2.4cm}: \field{nama}\\
			NPM \hspace{2.5cm}: \field{NPM}\\
			Program Studi \hspace{0.8cm}: \field{prodi}\\
			Tempat/Tgl. Lahir : \field{tanggalLahir}\\
			Alamat Bandung \hspace{0.4cm}: \field{alamat}\\
		\end{quote}
		
		terdaftar sebagai mahasiswa Fakultas teknologi Informasi dan Sains pada \textbf{semester \field{semester}} .\\
	Demikian surat keterangan ini dibuat agar dapat dipergunakan sebagaimana mestinya.\\
	\\

		
		\begin{flushright}
			Bandung, \field{tanggal}\\
					Wakil Dekan I\\
			\vspace{2cm}
			\underline{\field{wdI}}
		\end{flushright}

	\thispagestyle{empty}
		\newpage

}

\end{document}

\end{lstlisting}

\begin{lstlisting}[language=tex,basicstyle=
\tiny,caption=Format untuk surat pengantar pembuatan visa]
\documentclass[12pt]{letter}
\usepackage[legalpaper, top=55mm,bottom=7mm,left=35mm,right=18mm]{geometry}
\usepackage{mailmerge}

\begin{document}

\mailfields{nama,tanggalLahir,wargaNegara,tahunKe,thnAkademik,negaraTujuan,tanggalKunjungan,organisasiTujuan,tanggal,wdIII}

\mailrepeat{
\begin{flushright}
	Bandung, tanggal
\end{flushright}
	To:\\
	organisasiTujuan\\
	\textbf{VISA SECTION}\\
	Jakarta - Indonesia\\
	
	Re : APPLICATION FOR VISA SCHENGEN\\
	
	This is to acknowledge that\
	
		Name\hspace{2cm}: \field{nama}\\
		Date of Birth\hspace{0.6cm}: \field{tanggalLahir}\\
		Nationality\hspace{1cm}: \field{wargaNegara}\
	
	He is a student at Parahyangan Catholic University, Jl. Ciumbuleuit No. 94 Bandung, in Year \field{tahunKe} for the school year \field{thnAkademik}.\
	
	This letter is written to support his intention to visit \field{negaraTujuan} on \field{tanggalKunjungan}.
	
	\field{nama} will be back to school to continue his study after the holiday.\
	
	We hope that visa can be granted for this  purpose.\
	
	\vspace{2cm}	
			Kind regards\\
			\vspace{1.5cm}			
			
			
			\underline{\field{wdIII}}\\
			Vice Dean of Student Affairs

	\thispagestyle{empty}
		\newpage

}
% \mailfields{nama,NPM,tanggal,semester,thnAkademik,prodi,nextSemester,nextThnAkademik,dekan}

\end{document}
\end{lstlisting}

\begin{lstlisting}[language=tex,basicstyle=
\tiny,caption=Format untuk surat pengantar studi lapangan untuk 1 orang]
	\documentclass[12pt]{letter}
\usepackage[legalpaper, top=55mm,bottom=15mm,left=35mm,right=18mm]{geometry}
\usepackage{mailmerge}

\begin{document}
\mailfields{noSurat,nama,npm,prodi,matkul,judulTopik,instansi,alamat,tujuanKunjungan,kepada,kota}

\mailrepeat{
Nomor : \field{noSurat}\\
Hal\hspace{0.75cm}: Permohonan \field{tujuanKunjungan}\\
Lamp\hspace{0.35cm}: -\\


\textbf{Kepada YTH.\\
KEPALA \field{kepada}\\
\field{alamat}\\
\field{kota}}

 Dengan hormat,\\

Bersama ini kami informasikan bahwa salah seorang mahasiswa kami, yaitu:

	\begin{quote}
	Nama\hspace{2.4cm}: \field{nama}\\
	NPM\hspace{2.5cm}: \field{npm}\\
	Program Studi\hspace{0.8cm}: \field{prodi}
	\end{quote}
	saat ini sedang mengambil Matakuliah \field{matkul} dengan topik "\field{judulTopik}". Ybs. memerlukan informasi sehubungan dengan hal-hal yang terkait dengan masalah tersebut yang diterapkan di \field{instansi}.\

Untuk itu kami memohonkan ijin bagi mahasiswa kami tersebut agar diperbolehkan untuk melakukan \field{tujuanKunjungan} di \field{instansi}.\

Atas perhatian dan kerjasamanya yang diberikan kami mengucapkan terima kasih.
\vspace{2cm}
		\begin{flushright}
			Bandung, \field{tanggal}\\
					Wakil Dekan III\\
			\vspace{2cm}
			\underline{\field{wdIII}}
		\end{flushright}
\thispagestyle{empty}
	\newpage
}

\end{document}

\end{lstlisting}

\begin{lstlisting}[language=tex,basicstyle=
\tiny,caption=Format untuk surat pengantar studi lapangan untuk 2 orang]
	\documentclass[12pt]{letter}
\usepackage[legalpaper, top=55mm,bottom=15mm,left=35mm,right=18mm]{geometry}
\usepackage{mailmerge}

\begin{document}
\mailfields{noSurat,namaPemohon,npmPemohon,prodi,matkul,judulTopik,instansi,alamat,tujuanKunjungan,kepada,kota,namaAnggota,npmAnggota}

\mailrepeat{
Nomor : \field{noSurat}\\
Hal\hspace{0.75cm}: Permohonan \field{tujuanKunjungan}\\
Lamp\hspace{0.35cm}: -\\


\textbf{Kepada YTH.\\
KEPALA \field{kepada}\\
\field{alamat}\\
\field{kota}}

 Dengan hormat,\\

Bersama ini kami informasikan bahwa 2 orang mahasiswa kami, yaitu:

	\begin{quote}
		\begin{tabular}{|l|l|l|}
		\hline
		\textbf{No.}&\textbf{Nama}&\textbf{NPM}\\ \hline
		1&\field{namaPemohon}&\field{npmPemohon}\\ \hline
		2&\field{namaAnggota}&\field{npmAnggota}\\ \hline
	\end{tabular}
	\end{quote}
	saat ini sedang mengambil Matakuliah \field{matkul} dengan topik "\field{judulTopik}". Ybs. memerlukan informasi sehubungan dengan hal-hal yang terkait dengan masalah tersebut yang diterapkan di \field{instansi}.\

Untuk itu kami memohonkan ijin bagi mahasiswa kami tersebut agar diperbolehkan untuk melakukan \field{tujuanKunjungan} di \field{instansi}.\

Atas perhatian dan kerjasamanya yang diberikan kami mengucapkan terima kasih.
\vspace{2cm}
		\begin{flushright}
			Bandung, \field{tanggal}\\
					Wakil Dekan III\\
			\vspace{2cm}
			\underline{\field{wdIII}}
		\end{flushright}
\thispagestyle{empty}
	\newpage
}

\end{document}

\end{lstlisting}

\begin{lstlisting}[language=tex,basicstyle=
\tiny,caption=Format untuk surat pengantar studi lapangan untuk 3 orang]
	\documentclass[12pt]{letter}
\usepackage[legalpaper, top=55mm,bottom=15mm,left=35mm,right=18mm]{geometry}
\usepackage{mailmerge}

\begin{document}
\mailfields{noSurat,namaPemohon,npmPemohon,prodi,namaAnggota1,npmAnggota1,namaAnggota2,npmAnggota2,tujuanKunjungan,kepada,alamat,matkul,judulTopik,instansi,tanggal,wdIII,kota}

\mailrepeat{
Nomor : \field{noSurat}\\
Hal\hspace{0.75cm}: Permohonan \field{tujuanKunjungan}\\
Lamp\hspace{0.35cm}: -\\


\textbf{Kepada YTH.\\
KEPALA \field{kepada}\\
\field{alamat}\\
\field{kota}}

 Dengan hormat,\\

Bersama ini kami informasikan bahwa 3 orang mahasiswa kami, yaitu:

	\begin{quote}
		\textbf{\underline{KOORDINATOR}}\\
		Nama\hspace{2.4cm}: \field{namaPemohon}\\
		NPM\hspace{2.5cm}: \field{npmPemohon}\\
		Program Studi\hspace{0.8cm}: \field{prodi}\\

		\textbf{\underline{ANGGOTA}}\\
		\begin{tabular}{|l|l|l|l|}
		\hline
		\textbf{No.}&\textbf{Nama}&\textbf{NPM}\\ \hline
		1&\field{namaAnggota1}&\field{npmAnggota1}\\ \hline
		2&\field{namaAnggota2}&\field{npmAnggota2}\\ \hline
	\end{tabular}
	\end{quote}
	saat ini sedang mengambil Matakuliah \field{matkul} dengan topik "\field{judulTopik}". Ybs. memerlukan informasi sehubungan dengan hal-hal yang terkait dengan masalah tersebut yang diterapkan di \field{instansi}.\

Untuk itu kami memohonkan ijin bagi mahasiswa kami tersebut agar diperbolehkan untuk melakukan \field{tujuanKunjungan} di \field{instansi}.\

Atas perhatian dan kerjasamanya yang diberikan kami mengucapkan terima kasih.
\vspace{2cm}
		\begin{flushright}
			Bandung, \field{tanggal}\\
					Wakil Dekan III\\
			\vspace{2cm}
			\underline{\field{wdIII}}
		\end{flushright}
\thispagestyle{empty}
	\newpage
}

\end{document}

\end{lstlisting}

\begin{lstlisting}[language=tex,basicstyle=
\tiny,caption=Format untuk surat pengantar studi lapangan untuk 4 orang]
	\documentclass[12pt]{letter}
\usepackage[legalpaper, top=55mm,bottom=15mm,left=35mm,right=18mm]{geometry}
\usepackage{mailmerge}

\begin{document}

\mailfields{noSurat,namaPemohon,npmPemohon,prodi,namaAnggota1,npmAnggota1,namaAnggota2,npmAnggota2,namaAnggota3,npmAnggota3,tujuanKunjungan,kepada,alamat,matkul,judulTopik,instansi,tanggal,wdIII,kota}

\mailrepeat{
Nomor : \field{noSurat}\\
Hal\hspace{0.75cm}: Permohonan \field{tujuanKunjungan}\\
Lamp\hspace{0.35cm}: -\\


\textbf{Kepada YTH.\\
KEPALA \field{kepada}\\
\field{alamat}\\
\field{kota}}

 Dengan hormat,\\

Bersama ini kami informasikan bahwa 4 orang mahasiswa kami, yaitu:

	\begin{quote}
		\textbf{\underline{KOORDINATOR}}\\
		Nama\hspace{2.4cm}: \field{namaPemohon}\\
		NPM\hspace{2.5cm}: \field{npmPemohon}\\
		Program Studi\hspace{0.8cm}: \field{prodi}\\

		\textbf{\underline{ANGGOTA}}\\
		\begin{tabular}{|l|l|l|l|}
		\hline
		\textbf{No.}&\textbf{Nama}&\textbf{NPM}\\ \hline
		1&\field{namaAnggota1}&\field{npmAnggota1}\\ \hline
		2&\field{namaAnggota2}&\field{npmAnggota2}\\ \hline
		3&\field{namaAnggota3}&\field{npmAnggota3}\\ \hline
	\end{tabular}
	\end{quote}
	saat ini sedang mengambil Matakuliah \field{matkul} dengan topik "\field{judulTopik}". Ybs. memerlukan informasi sehubungan dengan hal-hal yang terkait dengan masalah tersebut yang diterapkan di \field{instansi}.\

Untuk itu kami memohonkan ijin bagi mahasiswa kami tersebut agar diperbolehkan untuk melakukan \field{tujuanKunjungan} di \field{instansi}.\

Atas perhatian dan kerjasamanya yang diberikan kami mengucapkan terima kasih.
\vspace{2cm}
		\begin{flushright}
			Bandung, \field{tanggal}\\
					Wakil Dekan III\\
			\vspace{2cm}
			\underline{\field{wdIII}}
		\end{flushright}
\thispagestyle{empty}
	\newpage
}

\end{document}

\end{lstlisting}

\begin{lstlisting}[language=tex,basicstyle=
\tiny,caption=Format untuk surat pengantar studi lapangan untuk 5 orang]
\documentclass[12pt]{letter}
\usepackage[legalpaper, top=55mm,bottom=15mm,left=35mm,right=18mm]{geometry}
\usepackage{mailmerge}

\begin{document}

\mailfields{noSurat,namaPemohon,npmPemohon,prodi,namaAnggota1,npmAnggota1,namaAnggota2,npmAnggota2,namaAnggota3,npmAnggota3,namaAnggota4,npmAnggota4,tujuanKunjungan,kepada,alamat,matkul,judulTopik,instansi,tanggal,wdIII,kota}

\mailrepeat{
Nomor : \field{noSurat}\\
Hal\hspace{0.75cm}: Permohonan \field{tujuanKunjungan}\\
Lamp\hspace{0.35cm}: -\\


\textbf{Kepada YTH.\\
KEPALA \field{kepada}\\
\field{alamat}\\
\field{kota}}

 Dengan hormat,\\

Bersama ini kami informasikan bahwa 5 orang mahasiswa kami, yaitu:

	\begin{quote}
		\textbf{\underline{KOORDINATOR}}\\
		Nama\hspace{2.4cm}: \field{namaPemohon}\\
		NPM\hspace{2.5cm}: \field{npmPemohon}\\
		Program Studi\hspace{0.8cm}: \field{prodi}\\

		\textbf{\underline{ANGGOTA}}\\
		\begin{tabular}{|l|l|l|l|}
		\hline
		\textbf{No.}&\textbf{Nama}&\textbf{NPM}\\ \hline
		1&\field{namaAnggota1}&\field{npmAnggota1}\\ \hline
		2&\field{namaAnggota2}&\field{npmAnggota2}\\ \hline
		3&\field{namaAnggota3}&\field{npmAnggota3}\\ \hline
		4&\field{namaAnggota4}&\field{npmAnggota4}\\ \hline
	\end{tabular}
	\end{quote}
	saat ini sedang mengambil Matakuliah \field{matkul} dengan topik "\field{judulTopik}". Ybs. memerlukan informasi sehubungan dengan hal-hal yang terkait dengan masalah tersebut yang diterapkan di \field{instansi}.\

Untuk itu kami memohonkan ijin bagi mahasiswa kami tersebut agar diperbolehkan untuk melakukan \field{tujuanKunjungan} di \field{instansi}.\

Atas perhatian dan kerjasamanya yang diberikan kami mengucapkan terima kasih.
\vspace{2cm}
		\begin{flushright}
			Bandung, \field{tanggal}\\
					Wakil Dekan III\\
			\vspace{2cm}
			\underline{\field{wdIII}}
		\end{flushright}
\thispagestyle{empty}
	\newpage
}

\end{document}


\end{lstlisting}

\begin{lstlisting}[language=tex,basicstyle=
\tiny,caption=Format untuk surat izin cuti studi]
\documentclass[12pt]{letter}
\usepackage[legalpaper, top=55mm,bottom=7mm,left=35mm,right=18mm]{geometry}
\usepackage{mailmerge}
\usepackage{changepage} %package utk atur margin

\begin{document}

\mailfields{noSurat,nama,NPM,tanggal,semester,thnAkademik,prodi,nextSemester,nextThnAkademik,dekan}

\mailrepeat{

	
	
	Nomor : \field{noSurat}\\
	Hal\hspace{0.75cm}: Surat Ijin Berhenti Studi Sementara\\
	Lamp\hspace*{0.35cm}: Surat Permohonan\\
	\vspace{0.5cm}
	
	Kepada : Yang terhormat\\
	\hspace*{1.7cm} \field{nama}\\
	\hspace*{1.7cm} NPM \field{NPM}\\
	\hspace*{1.7cm} di tempat\\
	\vspace{0.5cm}
	
	\begin{adjustwidth}{1.8cm}{0pt}Menurut surat permohonan cuti studi Anda tertanggal \field{tanggal}, dengan ini diberitahukan bahwa Anda diijinkan cuti studi untuk:
\end{adjustwidth}
		\begin{adjustwidth}{3cm}{0pt}
			\textbf{Semester \field{semester} tahun akademik \field{thnAkademik}}
		\end{adjustwidth}
		
		\begin{adjustwidth}{1.8cm}{0pt}
		Selanjutnya, untuk dapat mengikuti kegiatan akademik kembali, Anda wajib melakukan registrasi pada Masa Perwalian dan \textbf{FRS semester \field{nextSemester} tahun akademik \field{nextThnAkademik}} sesuai kalender akademik dan melaksanakan ketentuan yang telah ditetapkan.\\
		
		Demikian harap Anda maklum.
		\end{adjustwidth}
	\vspace{2cm}	
		\begin{flushright}
			Bandung,\field{tanggal}\\ 
					Dekan\\
			\vspace{2cm}
			\underline{\field{dekan}}
		\end{flushright}
		\vspace{5cm}
		\scriptsize {Tembusan :
		\begin{enumerate}
			\item WR-I, WR-II
			\item KBAA, KBTI
			\item Kajur \field{prodi}, Dosen Wali
			\item Map Ybs
			\item Arsip
		\end{enumerate}}

	\thispagestyle{empty}
		\newpage

}

\end{document}

\end{lstlisting}

\begin{lstlisting}[language=tex,basicstyle=
\tiny,caption=Format untuk surat izin pengunduran diri]
\documentclass[12pt]{letter}
\usepackage[legalpaper, top=55mm,bottom=7mm,left=35mm,right=18mm]{geometry}
\usepackage{mailmerge}
\usepackage{changepage} %package utk ubah margin

\begin{document}

\mailfields{noSurat,nama,NPM,prodi,tanggal,semester,tanggal,dekan}

\mailrepeat{

	
	
	Nomor : \field{noSurat}\\
	Hal\hspace{0.75cm}: Pengunduran diri\\
	Lamp\hspace*{0.35cm}: Surat pengunduran diri ybs.\\
	\vspace{0.5cm}
	
	Kepada : Yang terhormat\\
	\hspace*{1.7cm} Rektor\\
	\hspace*{1.7cm} Universitas Katolik Parahyangan\\
	\hspace*{1.7cm} Di tempat\\
	\vspace{0.5cm}
	
	\hspace*{1.8cm}Berdasarkan surat pengunduran diri :

		\begin{adjustwidth}{2.5cm}{0pt}
			\textbf{\field{nama}\\
			NPM  \field{NPM}\\
			Program Studi \field{prodi}}\\
		\end{adjustwidth}
		
		\begin{adjustwidth}{1.8cm}{0pt}
		tanggal \field{tanggal}, dengan ini kami nyatakan bahwa mahasiswa tersebut telah mengundurkan diri atas permintaan sendiri dari program studi FTIS terhitung mulai \textbf{\textit{semester \field{semester}}}.
		\end{adjustwidth}
		\vspace{2cm}
		\begin{flushright}
			Bandung, \field{tanggal}\\ 
					Dekan\\
			\vspace{2cm}
			\underline{\field{dekan}}
		\end{flushright}
		\vspace{5cm}
		{\scriptsize {Tembusan :
		\begin{enumerate}
			\item WR-I, WR-II
			\item KBAA, KBTI
			\item Kajur \field{prodi}, Dosen Wali
			\item Map Ybs
			\item Arsip
		\end{enumerate}}}

	\thispagestyle{empty}
		\newpage

}

\end{document}

\end{lstlisting}

\begin{lstlisting}[language=tex,basicstyle=
\tiny,caption=Format untuk surat perwakilan perwalian (ambil 1 mata kuliah)]
\documentclass[12pt]{letter}
\usepackage[legalpaper, top=25mm,bottom=15mm,left=35mm,right=18mm]{geometry}
\usepackage{mailmerge}

\begin{document}

\mailfields{semester,thnAKademik,namaPemohon,prodiPemohon,npmPemohon,namaWakil,prodiWakil,npmWakil,alasan,kodeMK,namaMK,sks}

\mailrepeat{

	\begin{center}
	{\Large \textbf{FORMULIR\\
	PERWALIAN YANG DIWAKILKAN}}
	\end{center}
	\vspace{0.25cm}
	Semester \hspace{1.5cm}: \field{semester} \\
	Tahun Akademik : \field{thnAkademik}\\

	{\footnotesize \textbf{IDENTITAS MAHASISWA YANG PERWALIANNYA DIWAKILKAN}}\\
	Nama \hspace{2.4cm}: \field{namaPemohon} \\
	Program Studi \hspace{0.8cm}: \field{prodiPemohon}\\
	NPM \hspace{2.5cm}: \field{npmPemohon}	\\
	{\footnotesize \textbf{IDENTITAS MAHASISWA YANG DIBERI KUASA PERWALIAN}}\\
	Nama \hspace{2.4cm}: \field{namaWakil} \\
	Program Studi \hspace{0.8cm}: \field{prodiWakil}\\
	NPM \hspace{2.5cm}: \field{npmWakil}\\

	Alasan tidak bisa hadir pada saat perwalian :\\
	\field{alasan}\\
	Mata kuliah yang diambil saat FRS : \\
	\begin{tabular}{|l|l|l|l|}
		\hline
		\textbf{No.}&\textbf{Kode MK}&\textbf{Nama Mata Kuliah}&\textbf{Sks}\\ \hline
		1&\field{kodeMK}&\field{namaMK}&\field{sks}\\ \hline
	\end{tabular}

\textbf{\underline{LAMPIRAN}} :
\begin{enumerate}
	\item Fotokopi KTM mahasiswa yag menerima kuasa
\end{enumerate}

		\begin{flushright}
			Bandung, \field{tanggal}
		\end{flushright}
Tanda Tangan \hspace{7.3cm}Tanda Tangan \\
Mahasiswa yang memberi kuasa \hspace{4cm} Mahasiswa yang menerima kuasa \\
\begin{flushright}
		\end{flushright}
			\vspace{1cm}

	(...........................................)\hspace{4.6cm} (...........................................)\\

	\textbf{Mengetahui dosen wali, \hspace{4.8cm} Menyetujui Wakil Dekan I}
	\vspace{2cm}

	(...........................................)\hspace{4.8cm}(\field{wdI})
	\thispagestyle{empty}
		\newpage

}
 

\end{document}

\end{lstlisting}

\begin{lstlisting}[language=tex,basicstyle=
\tiny,caption=Format untuk surat perwakilan perwalian (ambil 2 mata kuliah)]
\documentclass[12pt]{letter}
\usepackage[legalpaper, top=25mm,bottom=15mm,left=35mm,right=18mm]{geometry}
\usepackage{mailmerge}

\begin{document}

\mailfields{semester,thnAKademik,namaPemohon,prodiPemohon,npmPemohon,namaWakil,prodiWakil,npmWakil,alasan,kodeMK1,namaMK1,sks1,kodeMK2,namaMK2,sks2}

\mailrepeat{

	\begin{center}
	{\Large \textbf{FORMULIR\\
	PERWALIAN YANG DIWAKILKAN}}
	\end{center}
	\vspace{0.25cm}
	Semester \hspace{1.5cm}: \field{semester} \\
	Tahun Akademik : \field{thnAkademik}\\

	{\footnotesize \textbf{IDENTITAS MAHASISWA YANG PERWALIANNYA DIWAKILKAN}}\\
	Nama \hspace{2.4cm}: \field{namaPemohon} \\
	Program Studi \hspace{0.8cm}: \field{prodiPemohon}\\
	NPM \hspace{2.5cm}: \field{npmPemohon}	\\
	{\footnotesize \textbf{IDENTITAS MAHASISWA YANG DIBERI KUASA PERWALIAN}}\\
	Nama \hspace{2.4cm}: \field{namaWakil} \\
	Program Studi \hspace{0.8cm}: \field{prodiWakil}\\
	NPM \hspace{2.5cm}: \field{npmWakil}\\

	Alasan tidak bisa hadir pada saat perwalian :\\
	\field{alasan}\\
	Mata kuliah yang diambil saat FRS : \\
	\begin{tabular}{|l|l|l|l|}
		\hline
		\textbf{No.}&\textbf{Kode MK}&\textbf{Nama Mata Kuliah}&\textbf{Sks}\\ \hline
		1&\field{kodeMK1}&\field{namaMK1}&\field{sks1}\\ \hline
		2&\field{kodeMK2}&\field{namaMK2}&\field{sks2}\\ \hline
	\end{tabular}

\textbf{\underline{LAMPIRAN}} :
\begin{enumerate}
	\item Fotokopi KTM mahasiswa yag menerima kuasa
\end{enumerate}

		\begin{flushright}
			Bandung, \field{tanggal}
		\end{flushright}
Tanda Tangan \hspace{7.3cm}Tanda Tangan \\
Mahasiswa yang memberi kuasa \hspace{4cm} Mahasiswa yang menerima kuasa \\
\begin{flushright}
		\end{flushright}
			\vspace{1cm}

	(...........................................)\hspace{4.6cm} (...........................................)\\

	\textbf{Mengetahui dosen wali, \hspace{4.8cm} Menyetujui Wakil Dekan I}
	\vspace{2cm}

	(...........................................)\hspace{4.8cm}(\field{wdI})
	\thispagestyle{empty}
		\newpage

}

\end{document}

\end{lstlisting}




\begin{lstlisting}[language=tex,basicstyle=
\tiny,caption=Format untuk surat perwakilan perwalian (ambil 3 mata kuliah)]
\documentclass[12pt]{letter}
\usepackage[legalpaper, top=25mm,bottom=15mm,left=35mm,right=18mm]{geometry}
\usepackage{mailmerge}

\begin{document}

\mailfields{semester,thnAKademik,namaPemohon,prodiPemohon,npmPemohon,namaWakil,prodiWakil,npmWakil,alasan,kodeMK1,namaMK1,sks1,kodeMK2,namaMK2,sks2,kodeMK3,namaMK3,sks3}

\mailrepeat{

	\begin{center}
	{\Large \textbf{FORMULIR\\
	PERWALIAN YANG DIWAKILKAN}}
	\end{center}
	\vspace{0.25cm}
	Semester \hspace{1.5cm}: \field{semester} \\
	Tahun Akademik : \field{thnAkademik}\\

	{\footnotesize \textbf{IDENTITAS MAHASISWA YANG PERWALIANNYA DIWAKILKAN}}\\
	Nama \hspace{2.4cm}: \field{namaPemohon} \\
	Program Studi \hspace{0.8cm}: \field{prodiPemohon}\\
	NPM \hspace{2.5cm}: \field{npmPemohon}	\\
	{\footnotesize \textbf{IDENTITAS MAHASISWA YANG DIBERI KUASA PERWALIAN}}\\
	Nama \hspace{2.4cm}: \field{namaWakil} \\
	Program Studi \hspace{0.8cm}: \field{prodiWakil}\\
	NPM \hspace{2.5cm}: \field{npmWakil}\\

	Alasan tidak bisa hadir pada saat perwalian :\\
	\field{alasan}\\
	Mata kuliah yang diambil saat FRS : \\
	\begin{tabular}{|l|l|l|l|}
		\hline
		\textbf{No.}&\textbf{Kode MK}&\textbf{Nama Mata Kuliah}&\textbf{Sks}\\ \hline
		1&\field{kodeMK1}&\field{namaMK1}&\field{sks1}\\ \hline
		2&\field{kodeMK2}&\field{namaMK2}&\field{sks2}\\ \hline
		3&\field{kodeMK3}&\field{namaMK3}&\field{sks3}\\ \hline
	\end{tabular}

\textbf{\underline{LAMPIRAN}} :
\begin{enumerate}
	\item Fotokopi KTM mahasiswa yag menerima kuasa
\end{enumerate}

		\begin{flushright}
			Bandung, \field{tanggal}
		\end{flushright}
Tanda Tangan \hspace{7.3cm}Tanda Tangan \\
Mahasiswa yang memberi kuasa \hspace{4cm} Mahasiswa yang menerima kuasa \\
\begin{flushright}
		\end{flushright}
			\vspace{1cm}

	(...........................................)\hspace{4.6cm} (...........................................)\\

	\textbf{Mengetahui dosen wali, \hspace{4.8cm} Menyetujui Wakil Dekan I}
	\vspace{2cm}

	(...........................................)\hspace{4.8cm}(\field{wdI})
	\thispagestyle{empty}
		\newpage

}



\end{document}

\end{lstlisting}

\begin{lstlisting}[language=tex,basicstyle=
\tiny,caption=Format untuk surat perwakilan perwalian (ambil 4 mata kuliah)]
\documentclass[12pt]{letter}
\usepackage[legalpaper, top=25mm,bottom=15mm,left=35mm,right=18mm]{geometry}
\usepackage{mailmerge}

\begin{document}

\mailfields{semester,thnAKademik,namaPemohon,prodiPemohon,npmPemohon,namaWakil,prodiWakil,npmWakil,alasan,kodeMK1,namaMK1,sks1,kodeMK2,namaMK2,sks2,kodeMK3,namaMK3,sks3,kodeMK4,namaMK4,sks4,kodeMK5,namaMK5,sks5,kodeMK6,namaMK6,sks6}

\mailrepeat{

	\begin{center}
	{\Large \textbf{FORMULIR\\
	PERWALIAN YANG DIWAKILKAN}}
	\end{center}
	\vspace{0.25cm}
	Semester \hspace{1.5cm}: \field{semester} \\
	Tahun Akademik : \field{thnAkademik}\\

	{\footnotesize \textbf{IDENTITAS MAHASISWA YANG PERWALIANNYA DIWAKILKAN}}\\
	Nama \hspace{2.4cm}: \field{namaPemohon} \\
	Program Studi \hspace{0.8cm}: \field{prodiPemohon}\\
	NPM \hspace{2.5cm}: \field{npmPemohon}	\\
	{\footnotesize \textbf{IDENTITAS MAHASISWA YANG DIBERI KUASA PERWALIAN}}\\
	Nama \hspace{2.4cm}: \field{namaWakil} \\
	Program Studi \hspace{0.8cm}: \field{prodiWakil}\\
	NPM \hspace{2.5cm}: \field{npmWakil}\\

	Alasan tidak bisa hadir pada saat perwalian :\\
	\field{alasan}\\
	Mata kuliah yang diambil saat FRS : \\
	\begin{tabular}{|l|l|l|l|}
		\hline
		\textbf{No.}&\textbf{Kode MK}&\textbf{Nama Mata Kuliah}&\textbf{Sks}\\ \hline
		1&\field{kodeMK1}&\field{namaMK1}&\field{sks1}\\ \hline
		2&\field{kodeMK2}&\field{namaMK2}&\field{sks2}\\ \hline
		3&\field{kodeMK3}&\field{namaMK3}&\field{sks3}\\ \hline
		4&\field{kodeMK4}&\field{namaMK4}&\field{sks4}\\ \hline
		5&\field{kodeMK5}&\field{namaMK5}&\field{sks5}\\ \hline
	\end{tabular}

\textbf{\underline{LAMPIRAN}} :
\begin{enumerate}
	\item Fotokopi KTM mahasiswa yag menerima kuasa
\end{enumerate}

		\begin{flushright}
			Bandung, \field{tanggal}
		\end{flushright}
Tanda Tangan \hspace{7.3cm}Tanda Tangan \\
Mahasiswa yang memberi kuasa \hspace{4cm} Mahasiswa yang menerima kuasa \\
\begin{flushright}
		\end{flushright}
			\vspace{1cm}

	(...........................................)\hspace{4.6cm} (...........................................)\\

	\textbf{Mengetahui dosen wali, \hspace{4.8cm} Menyetujui Wakil Dekan I}
	\vspace{2cm}

	(...........................................)\hspace{4.8cm}(\field{wdI})
	\thispagestyle{empty}
		\newpage

}



\end{document}

\end{lstlisting}

\begin{lstlisting}[language=tex,basicstyle=
\tiny,caption=Format untuk surat perwakilan perwalian (ambil 5 mata kuliah)]
\documentclass[12pt]{letter}
\usepackage[legalpaper, top=25mm,bottom=15mm,left=35mm,right=18mm]{geometry}
\usepackage{mailmerge}

\begin{document}

\mailfields{semester,thnAKademik,namaPemohon,prodiPemohon,npmPemohon,namaWakil,prodiWakil,npmWakil,alasan,kodeMK1,namaMK1,sks1,kodeMK2,namaMK2,sks2,kodeMK3,namaMK3,sks3,kodeMK4,namaMK4,sks4,kodeMK5,namaMK5,sks5}

\mailrepeat{

	\begin{center}
	{\Large \textbf{FORMULIR\\
	PERWALIAN YANG DIWAKILKAN}}
	\end{center}
	\vspace{0.25cm}
	Semester \hspace{1.5cm}: \field{semester} \\
	Tahun Akademik : \field{thnAkademik}\\

	{\footnotesize \textbf{IDENTITAS MAHASISWA YANG PERWALIANNYA DIWAKILKAN}}\\
	Nama \hspace{2.4cm}: \field{namaPemohon} \\
	Program Studi \hspace{0.8cm}: \field{prodiPemohon}\\
	NPM \hspace{2.5cm}: \field{npmPemohon}	\\
	{\footnotesize \textbf{IDENTITAS MAHASISWA YANG DIBERI KUASA PERWALIAN}}\\
	Nama \hspace{2.4cm}: \field{namaWakil} \\
	Program Studi \hspace{0.8cm}: \field{prodiWakil}\\
	NPM \hspace{2.5cm}: \field{npmWakil}\\

	Alasan tidak bisa hadir pada saat perwalian :\\
	\field{alasan}\\
	Mata kuliah yang diambil saat FRS : \\
	\begin{tabular}{|l|l|l|l|}
		\hline
		\textbf{No.}&\textbf{Kode MK}&\textbf{Nama Mata Kuliah}&\textbf{Sks}\\ \hline
		1&\field{kodeMK1}&\field{namaMK1}&\field{sks1}\\ \hline
		2&\field{kodeMK2}&\field{namaMK2}&\field{sks2}\\ \hline
		3&\field{kodeMK3}&\field{namaMK3}&\field{sks3}\\ \hline
		4&\field{kodeMK4}&\field{namaMK4}&\field{sks4}\\ \hline
		5&\field{kodeMK5}&\field{namaMK5}&\field{sks5}\\ \hline
	\end{tabular}

\textbf{\underline{LAMPIRAN}} :
\begin{enumerate}
	\item Fotokopi KTM mahasiswa yag menerima kuasa
\end{enumerate}

		\begin{flushright}
			Bandung, \field{tanggal}
		\end{flushright}
Tanda Tangan \hspace{7.3cm}Tanda Tangan \\
Mahasiswa yang memberi kuasa \hspace{4cm} Mahasiswa yang menerima kuasa \\
\begin{flushright}
		\end{flushright}
			\vspace{1cm}

	(...........................................)\hspace{4.6cm} (...........................................)\\

	\textbf{Mengetahui dosen wali, \hspace{4.8cm} Menyetujui Wakil Dekan I}
	\vspace{2cm}

	(...........................................)\hspace{4.8cm}(\field{wdI})
	\thispagestyle{empty}
		\newpage

}


\end{document}

\end{lstlisting}

\begin{lstlisting}[language=tex,basicstyle=
\tiny,caption=Format untuk surat perwakilan perwalian (ambil 6 mata kuliah)]
\documentclass[12pt]{letter}
\usepackage[legalpaper, top=25mm,bottom=15mm,left=35mm,right=18mm]{geometry}
\usepackage{mailmerge}

\begin{document}

\mailfields{semester,thnAKademik,namaPemohon,prodiPemohon,npmPemohon,namaWakil,prodiWakil,npmWakil,alasan,kodeMK1,namaMK1,sks1,kodeMK2,namaMK2,sks2,kodeMK3,namaMK3,sks3,kodeMK4,namaMK4,sks4,kodeMK5,namaMK5,sks5,kodeMK6,namaMK6,sks6}

\mailrepeat{

	\begin{center}
	{\Large \textbf{FORMULIR\\
	PERWALIAN YANG DIWAKILKAN}}
	\end{center}
	\vspace{0.25cm}
	Semester \hspace{1.5cm}: \field{semester} \\
	Tahun Akademik : \field{thnAkademik}\\

	{\footnotesize \textbf{IDENTITAS MAHASISWA YANG PERWALIANNYA DIWAKILKAN}}\\
	Nama \hspace{2.4cm}: \field{namaPemohon} \\
	Program Studi \hspace{0.8cm}: \field{prodiPemohon}\\
	NPM \hspace{2.5cm}: \field{npmPemohon}	\\
	{\footnotesize \textbf{IDENTITAS MAHASISWA YANG DIBERI KUASA PERWALIAN}}\\
	Nama \hspace{2.4cm}: \field{namaWakil} \\
	Program Studi \hspace{0.8cm}: \field{prodiWakil}\\
	NPM \hspace{2.5cm}: \field{npmWakil}\\

	Alasan tidak bisa hadir pada saat perwalian :\\
	\field{alasan}\\
	Mata kuliah yang diambil saat FRS : \\
	\begin{tabular}{|l|l|l|l|}
		\hline
		\textbf{No.}&\textbf{Kode MK}&\textbf{Nama Mata Kuliah}&\textbf{Sks}\\ \hline
		1&\field{kodeMK1}&\field{namaMK1}&\field{sks1}\\ \hline
		2&\field{kodeMK2}&\field{namaMK2}&\field{sks2}\\ \hline
		3&\field{kodeMK3}&\field{namaMK3}&\field{sks3}\\ \hline
		4&\field{kodeMK4}&\field{namaMK4}&\field{sks4}\\ \hline
		5&\field{kodeMK5}&\field{namaMK5}&\field{sks5}\\ \hline
		6&\field{kodeMK6}&\field{namaMK6}&\field{sks6}\\ \hline
	\end{tabular}

\textbf{\underline{LAMPIRAN}} :
\begin{enumerate}
	\item Fotokopi KTM mahasiswa yag menerima kuasa
\end{enumerate}

		\begin{flushright}
			Bandung, \field{tanggal}
		\end{flushright}
Tanda Tangan \hspace{7.3cm}Tanda Tangan \\
Mahasiswa yang memberi kuasa \hspace{4cm} Mahasiswa yang menerima kuasa \\
\begin{flushright}
		\end{flushright}
			\vspace{1cm}

	(...........................................)\hspace{4.6cm} (...........................................)\\

	\textbf{Mengetahui dosen wali, \hspace{4.8cm} Menyetujui Wakil Dekan I}
	\vspace{2cm}

	(...........................................)\hspace{4.8cm}(\field{wdI})
	\thispagestyle{empty}
		\newpage

}


\end{document}

\end{lstlisting}

\begin{lstlisting}[language=tex,basicstyle=
\tiny,caption=Format untuk surat perwakilan perwalian (ambil 7 mata kuliah)]
\documentclass[12pt]{letter}
\usepackage[legalpaper, top=25mm,bottom=15mm,left=35mm,right=18mm]{geometry}
\usepackage{mailmerge}

\begin{document}

\mailfields{semester,thnAKademik,namaPemohon,prodiPemohon,npmPemohon,namaWakil,prodiWakil,npmWakil,alasan,kodeMK1,namaMK1,sks1,kodeMK2,namaMK2,sks2,kodeMK3,namaMK3,sks3,kodeMK4,namaMK4,sks4,kodeMK5,namaMK5,sks5,kodeMK6,namaMK6,sks6,kodeMK7,namaMK7,sks7}

\mailrepeat{

	\begin{center}
	{\Large \textbf{FORMULIR\\
	PERWALIAN YANG DIWAKILKAN}}
	\end{center}
	\vspace{0.25cm}
	Semester \hspace{1.5cm}: \field{semester} \\
	Tahun Akademik : \field{thnAkademik}\\

	{\footnotesize \textbf{IDENTITAS MAHASISWA YANG PERWALIANNYA DIWAKILKAN}}\\
	Nama \hspace{2.4cm}: \field{namaPemohon} \\
	Program Studi \hspace{0.8cm}: \field{prodiPemohon}\\
	NPM \hspace{2.5cm}: \field{npmPemohon}	\\
	{\footnotesize \textbf{IDENTITAS MAHASISWA YANG DIBERI KUASA PERWALIAN}}\\
	Nama \hspace{2.4cm}: \field{namaWakil} \\
	Program Studi \hspace{0.8cm}: \field{prodiWakil}\\
	NPM \hspace{2.5cm}: \field{npmWakil}\\

	Alasan tidak bisa hadir pada saat perwalian :\\
	\field{alasan}\\
	Mata kuliah yang diambil saat FRS : \\
	\begin{tabular}{|l|l|l|l|}
		\hline
		\textbf{No.}&\textbf{Kode MK}&\textbf{Nama Mata Kuliah}&\textbf{Sks}\\ \hline
		1&\field{kodeMK1}&\field{namaMK1}&\field{sks1}\\ \hline
		2&\field{kodeMK2}&\field{namaMK2}&\field{sks2}\\ \hline
		3&\field{kodeMK3}&\field{namaMK3}&\field{sks3}\\ \hline
		4&\field{kodeMK4}&\field{namaMK4}&\field{sks4}\\ \hline
		5&\field{kodeMK5}&\field{namaMK5}&\field{sks5}\\ \hline
		6&\field{kodeMK6}&\field{namaMK6}&\field{sks6}\\ \hline
		7&\field{kodeMK7}&\field{namaMK7}&\field{sks7}\\ \hline
	\end{tabular}

\textbf{\underline{LAMPIRAN}} :
\begin{enumerate}
	\item Fotokopi KTM mahasiswa yag menerima kuasa
\end{enumerate}

		\begin{flushright}
			Bandung, \field{tanggal}
		\end{flushright}
Tanda Tangan \hspace{7.3cm}Tanda Tangan \\
Mahasiswa yang memberi kuasa \hspace{4cm} Mahasiswa yang menerima kuasa \\
\begin{flushright}
		\end{flushright}
			\vspace{1cm}

	(...........................................)\hspace{4.6cm} (...........................................)\\

	\textbf{Mengetahui dosen wali, \hspace{4.8cm} Menyetujui Wakil Dekan I}
	\vspace{2cm}

	(...........................................)\hspace{4.8cm}(\field{wdI})
	\thispagestyle{empty}
		\newpage

}



\end{document}

\end{lstlisting}

\begin{lstlisting}[language=tex,basicstyle=
\tiny,caption=Format untuk surat perwakilan perwalian (ambil 8 mata kuliah)]
\documentclass[12pt]{letter}
\usepackage[legalpaper, top=25mm,bottom=15mm,left=35mm,right=18mm]{geometry}
\usepackage{mailmerge}

\begin{document}

\mailfields{semester,thnAKademik,namaPemohon,prodiPemohon,npmPemohon,namaWakil,prodiWakil,npmWakil,alasan,kodeMK1,namaMK1,sks1,kodeMK2,namaMK2,sks2,kodeMK3,namaMK3,sks3,kodeMK4,namaMK4,sks4,kodeMK5,namaMK5,sks5,kodeMK6,namaMK6,sks6,kodeMK7,namaMK7,sks7,kodeMK8,namaMK8,sks8}

\mailrepeat{

	\begin{center}
	{\Large \textbf{FORMULIR\\
	PERWALIAN YANG DIWAKILKAN}}
	\end{center}
	\vspace{0.25cm}
	Semester \hspace{1.5cm}: \field{semester} \\
	Tahun Akademik : \field{thnAkademik}\\

	{\footnotesize \textbf{IDENTITAS MAHASISWA YANG PERWALIANNYA DIWAKILKAN}}\\
	Nama \hspace{2.4cm}: \field{namaPemohon} \\
	Program Studi \hspace{0.8cm}: \field{prodiPemohon}\\
	NPM \hspace{2.5cm}: \field{npmPemohon}	\\
	{\footnotesize \textbf{IDENTITAS MAHASISWA YANG DIBERI KUASA PERWALIAN}}\\
	Nama \hspace{2.4cm}: \field{namaWakil} \\
	Program Studi \hspace{0.8cm}: \field{prodiWakil}\\
	NPM \hspace{2.5cm}: \field{npmWakil}\\

	Alasan tidak bisa hadir pada saat perwalian :\\
	\field{alasan}\\
	Mata kuliah yang diambil saat FRS : \\
	\begin{tabular}{|l|l|l|l|}
		\hline
		\textbf{No.}&\textbf{Kode MK}&\textbf{Nama Mata Kuliah}&\textbf{Sks}\\ \hline
		1&\field{kodeMK1}&\field{namaMK1}&\field{sks1}\\ \hline
		2&\field{kodeMK2}&\field{namaMK2}&\field{sks2}\\ \hline
		3&\field{kodeMK3}&\field{namaMK3}&\field{sks3}\\ \hline
		4&\field{kodeMK4}&\field{namaMK4}&\field{sks4}\\ \hline
		5&\field{kodeMK5}&\field{namaMK5}&\field{sks5}\\ \hline
		6&\field{kodeMK6}&\field{namaMK6}&\field{sks6}\\ \hline
		7&\field{kodeMK7}&\field{namaMK7}&\field{sks7}\\ \hline
		8&\field{kodeMK8}&\field{namaMK8}&\field{sks8}\\ \hline
	\end{tabular}

\textbf{\underline{LAMPIRAN}} :
\begin{enumerate}
	\item Fotokopi KTM mahasiswa yag menerima kuasa
\end{enumerate}

		\begin{flushright}
			Bandung, \field{tanggal}
		\end{flushright}
Tanda Tangan \hspace{7.3cm}Tanda Tangan \\
Mahasiswa yang memberi kuasa \hspace{4cm} Mahasiswa yang menerima kuasa \\
\begin{flushright}
		\end{flushright}
			\vspace{1cm}

	(...........................................)\hspace{4.6cm} (...........................................)\\

	\textbf{Mengetahui dosen wali, \hspace{4.8cm} Menyetujui Wakil Dekan I}
	\vspace{2cm}

	(...........................................)\hspace{4.8cm}(\field{wdI})
	\thispagestyle{empty}
		\newpage

}


\end{document}

\end{lstlisting}

\begin{lstlisting}[language=tex,basicstyle=
\tiny,caption=Format untuk surat perwakilan perwalian (ambil 9 mata kuliah)]
\documentclass[12pt]{letter}
\usepackage[legalpaper, top=25mm,bottom=15mm,left=35mm,right=18mm]{geometry}
\usepackage{mailmerge}

\begin{document}

\mailfields{semester,thnAKademik,namaPemohon,prodiPemohon,npmPemohon,namaWakil,prodiWakil,npmWakil,alasan,kodeMK1,namaMK1,sks1,kodeMK2,namaMK2,sks2,kodeMK3,namaMK3,sks3,kodeMK4,namaMK4,sks4,kodeMK5,namaMK5,sks5,kodeMK6,namaMK6,sks6,kodeMK7,namaMK7,sks7,kodeMK8,namaMK8,sks8,kodeMK9,namaMK9,sks9}

\mailrepeat{

	\begin{center}
	{\Large \textbf{FORMULIR\\
	PERWALIAN YANG DIWAKILKAN}}
	\end{center}
	\vspace{0.25cm}
	Semester \hspace{1.5cm}: \field{semester} \\
	Tahun Akademik : \field{thnAkademik}\\

	{\footnotesize \textbf{IDENTITAS MAHASISWA YANG PERWALIANNYA DIWAKILKAN}}\\
	Nama \hspace{2.4cm}: \field{namaPemohon} \\
	Program Studi \hspace{0.8cm}: \field{prodiPemohon}\\
	NPM \hspace{2.5cm}: \field{npmPemohon}	\\
	{\footnotesize \textbf{IDENTITAS MAHASISWA YANG DIBERI KUASA PERWALIAN}}\\
	Nama \hspace{2.4cm}: \field{namaWakil} \\
	Program Studi \hspace{0.8cm}: \field{prodiWakil}\\
	NPM \hspace{2.5cm}: \field{npmWakil}\\

	Alasan tidak bisa hadir pada saat perwalian :\\
	\field{alasan}\\
	Mata kuliah yang diambil saat FRS : \\
	\begin{tabular}{|l|l|l|l|}
		\hline
		\textbf{No.}&\textbf{Kode MK}&\textbf{Nama Mata Kuliah}&\textbf{Sks}\\ \hline
		1&\field{kodeMK1}&\field{namaMK1}&\field{sks1}\\ \hline
		2&\field{kodeMK2}&\field{namaMK2}&\field{sks2}\\ \hline
		3&\field{kodeMK3}&\field{namaMK3}&\field{sks3}\\ \hline
		4&\field{kodeMK4}&\field{namaMK4}&\field{sks4}\\ \hline
		5&\field{kodeMK5}&\field{namaMK5}&\field{sks5}\\ \hline
		6&\field{kodeMK6}&\field{namaMK6}&\field{sks6}\\ \hline
		7&\field{kodeMK7}&\field{namaMK7}&\field{sks7}\\ \hline
		8&\field{kodeMK8}&\field{namaMK8}&\field{sks8}\\ \hline
		9&\field{kodeMK9}&\field{namaMK9}&\field{sks9}\\ \hline
	\end{tabular}

\textbf{\underline{LAMPIRAN}} :
\begin{enumerate}
	\item Fotokopi KTM mahasiswa yag menerima kuasa
\end{enumerate}

		\begin{flushright}
			Bandung, \field{tanggal}
		\end{flushright}
Tanda Tangan \hspace{7.3cm}Tanda Tangan \\
Mahasiswa yang memberi kuasa \hspace{4cm} Mahasiswa yang menerima kuasa \\
\begin{flushright}
		\end{flushright}
			\vspace{1cm}

	(...........................................)\hspace{4.6cm} (...........................................)\\

	\textbf{Mengetahui dosen wali, \hspace{4.8cm} Menyetujui Wakil Dekan I}
	\vspace{2cm}

	(...........................................)\hspace{4.8cm}(\field{wdI})
	\thispagestyle{empty}
		\newpage

}


\end{document}

\end{lstlisting}

\begin{lstlisting}[language=tex,basicstyle=
\tiny,caption=Format untuk surat perwakilan perwalian (ambil 10 mata kuliah)]
\documentclass[12pt]{letter}
\usepackage[legalpaper, top=25mm,bottom=15mm,left=35mm,right=18mm]{geometry}
\usepackage{mailmerge}

\begin{document}

\mailfields{semester,thnAKademik,namaPemohon,prodiPemohon,npmPemohon,namaWakil,prodiWakil,npmWakil,alasan,kodeMK1,namaMK1,sks1,kodeMK2,namaMK2,sks2,kodeMK3,namaMK3,sks3,kodeMK4,namaMK4,sks4,kodeMK5,namaMK5,sks5,kodeMK6,namaMK6,sks6,kodeMK7,namaMK7,sks7,kodeMK8,namaMK8,sks8,kodeMK9,namaMK9,sks9,kodeMK10,namaMK10,sks10}

\mailrepeat{

	\begin{center}
	{\Large \textbf{FORMULIR\\
	PERWALIAN YANG DIWAKILKAN}}
	\end{center}
	\vspace{0.25cm}
	Semester \hspace{1.5cm}: \field{semester} \\
	Tahun Akademik : \field{thnAkademik}\\

	{\footnotesize \textbf{IDENTITAS MAHASISWA YANG PERWALIANNYA DIWAKILKAN}}\\
	Nama \hspace{2.4cm}: \field{namaPemohon} \\
	Program Studi \hspace{0.8cm}: \field{prodiPemohon}\\
	NPM \hspace{2.5cm}: \field{npmPemohon}	\\
	{\footnotesize \textbf{IDENTITAS MAHASISWA YANG DIBERI KUASA PERWALIAN}}\\
	Nama \hspace{2.4cm}: \field{namaWakil} \\
	Program Studi \hspace{0.8cm}: \field{prodiWakil}\\
	NPM \hspace{2.5cm}: \field{npmWakil}\\

	Alasan tidak bisa hadir pada saat perwalian :\\
	\field{alasan}\\
	Mata kuliah yang diambil saat FRS : \\
	\begin{tabular}{|l|l|l|l|}
		\hline
		\textbf{No.}&\textbf{Kode MK}&\textbf{Nama Mata Kuliah}&\textbf{Sks}\\ \hline
		1&\field{kodeMK1}&\field{namaMK1}&\field{sks1}\\ \hline
		2&\field{kodeMK2}&\field{namaMK2}&\field{sks2}\\ \hline
		3&\field{kodeMK3}&\field{namaMK3}&\field{sks3}\\ \hline
		4&\field{kodeMK4}&\field{namaMK4}&\field{sks4}\\ \hline
		5&\field{kodeMK5}&\field{namaMK5}&\field{sks5}\\ \hline
		6&\field{kodeMK6}&\field{namaMK6}&\field{sks6}\\ \hline
		7&\field{kodeMK7}&\field{namaMK7}&\field{sks7}\\ \hline
		8&\field{kodeMK8}&\field{namaMK8}&\field{sks8}\\ \hline
		9&\field{kodeMK9}&\field{namaMK9}&\field{sks9}\\ \hline
		10&\field{kodeMK10}&\field{namaMK10}&\field{sks10}\\ \hline
	\end{tabular}

\textbf{\underline{LAMPIRAN}} :
\begin{enumerate}
	\item Fotokopi KTM mahasiswa yag menerima kuasa
\end{enumerate}

		\begin{flushright}
			Bandung, \field{tanggal}
		\end{flushright}
Tanda Tangan \hspace{7.3cm}Tanda Tangan \\
Mahasiswa yang memberi kuasa \hspace{4cm} Mahasiswa yang menerima kuasa \\
\begin{flushright}
		\end{flushright}
			\vspace{1cm}

	(...........................................)\hspace{4.6cm} (...........................................)\\

	\textbf{Mengetahui dosen wali, \hspace{4.8cm} Menyetujui Wakil Dekan I}
	\vspace{2cm}

	(...........................................)\hspace{4.8cm}(\field{wdI})
	\thispagestyle{empty}
		\newpage

}


\end{document}


\end{lstlisting}

\begin{lstlisting}[language=tex,basicstyle=
\tiny,caption=Format untuk surat keterangan beasiswa]
	\documentclass[12pt]{letter}
\usepackage[legalpaper, top=53mm,bottom=15mm,left=35mm,right=18mm]{geometry}
\usepackage{mailmerge}

\begin{document}

\mailfields{noSurat,nama,prodi,npm,semester,thnAkademik,penyediabeasiswa,tanggal}

\mailrepeat{

	\begin{center}
	\underline{SURAT KETERANGAN}\\
	{\footnotesize  \field{noSurat}}
	\end{center}
	\vspace{0.75cm}
	Bersama ini kami menerangkan bahwa mahasiswa Fakultas teknologi Informasi dan Sains (FTIS) berikut : \
	\begin{quote}
		Nama\hspace{2cm}: \field{nama}\

		Jurusan\hspace{1.6cm}: \field{prodi}\

		NPM\hspace{2.1cm}: \field{npm}\
	\end{quote}
	BENAR tidak menerima beasiswa dari Universitas Katolik Parahyangan pada Semester \field{semester} Tahun Akademik \field{thnAkademik}.\

	Surat keterangan ini dibuat sebagai kelengkapan bagi ybs. untuk mengajukan diri sebagai penerima beasiswa perusahaan \field{penyediabeasiswa}.
		\vspace{0.5cm}
		\begin{flushright}
			Bandung, \field{tanggal}

			\vspace{2cm}




		\underline{Flaviana Catherine, S.Si, M.T}\\
		WD III FTIS - UNPAR
		\end{flushright}

		\thispagestyle{empty}
			\newpage

}

\end{document}

\end{lstlisting}

\begin{lstlisting}[language=tex,basicstyle=
\tiny,caption=Format untuk surat keterangan mahasiswa aktif]
	\documentclass[12pt]{letter}
%\usepackage[legalpaper, top=55mm,bottom=15mm,left=35mm,right=18mm]{geometry}
\usepackage{mailmerge}

\begin{document}

\mailfields{noSurat,nama,NPM,prodi,tanggalLahir,alamat,semester,wdI,tanggal}

\mailrepeat{

	\begin{center}
	\underline{SURAT KETERANGAN}\\
	{\footnotesize \field{noSurat}}
	\end{center}
	\vspace{0.75cm}
	Fakultas teknologi Informasi dan Sains Universitas Katolik Parahyangan Bandung, menerangkan :


		\begin{quote}
			Nama \hspace{2.4cm}: \field{nama}\\
			NPM \hspace{2.5cm}: \field{NPM}\\
			Program Studi \hspace{0.8cm}: \field{prodi}\\
			Tempat/Tgl. Lahir : \field{tanggalLahir}\\
			Alamat Bandung \hspace{0.4cm}: \field{alamat}\\
		\end{quote}
		
		terdaftar sebagai mahasiswa Fakultas teknologi Informasi dan Sains pada \textbf{semester \field{semester}} .\\
	Demikian surat keterangan ini dibuat agar dapat dipergunakan sebagaimana mestinya.\\
	\\

		
		\begin{flushright}
			Bandung, \field{tanggal}\\
					Wakil Dekan I\\
			\vspace{2cm}
			\underline{\field{wdI}}
		\end{flushright}

	\thispagestyle{empty}
		\newpage

}

\end{document}

\end{lstlisting}

\begin{lstlisting}[language=tex,basicstyle=
\tiny,caption=Format untuk surat pengantar pembuatan visa]
\documentclass[12pt]{letter}
\usepackage[legalpaper, top=55mm,bottom=7mm,left=35mm,right=18mm]{geometry}
\usepackage{mailmerge}

\begin{document}

\mailfields{nama,tanggalLahir,wargaNegara,tahunKe,thnAkademik,negaraTujuan,tanggalKunjungan,organisasiTujuan,tanggal,wdIII}

\mailrepeat{
\begin{flushright}
	Bandung, tanggal
\end{flushright}
	To:\\
	organisasiTujuan\\
	\textbf{VISA SECTION}\\
	Jakarta - Indonesia\\
	
	Re : APPLICATION FOR VISA SCHENGEN\\
	
	This is to acknowledge that\
	
		Name\hspace{2cm}: \field{nama}\\
		Date of Birth\hspace{0.6cm}: \field{tanggalLahir}\\
		Nationality\hspace{1cm}: \field{wargaNegara}\
	
	He is a student at Parahyangan Catholic University, Jl. Ciumbuleuit No. 94 Bandung, in Year \field{tahunKe} for the school year \field{thnAkademik}.\
	
	This letter is written to support his intention to visit \field{negaraTujuan} on \field{tanggalKunjungan}.
	
	\field{nama} will be back to school to continue his study after the holiday.\
	
	We hope that visa can be granted for this  purpose.\
	
	\vspace{2cm}	
			Kind regards\\
			\vspace{1.5cm}			
			
			
			\underline{\field{wdIII}}\\
			Vice Dean of Student Affairs

	\thispagestyle{empty}
		\newpage

}
% \mailfields{nama,NPM,tanggal,semester,thnAkademik,prodi,nextSemester,nextThnAkademik,dekan}

\end{document}
\end{lstlisting}

\begin{lstlisting}[language=tex,basicstyle=
\tiny,caption=Format untuk surat pengantar studi lapangan untuk 1 orang]
	\documentclass[12pt]{letter}
\usepackage[legalpaper, top=55mm,bottom=15mm,left=35mm,right=18mm]{geometry}
\usepackage{mailmerge}

\begin{document}
\mailfields{noSurat,nama,npm,prodi,matkul,judulTopik,instansi,alamat,tujuanKunjungan,kepada,kota}

\mailrepeat{
Nomor : \field{noSurat}\\
Hal\hspace{0.75cm}: Permohonan \field{tujuanKunjungan}\\
Lamp\hspace{0.35cm}: -\\


\textbf{Kepada YTH.\\
KEPALA \field{kepada}\\
\field{alamat}\\
\field{kota}}

 Dengan hormat,\\

Bersama ini kami informasikan bahwa salah seorang mahasiswa kami, yaitu:

	\begin{quote}
	Nama\hspace{2.4cm}: \field{nama}\\
	NPM\hspace{2.5cm}: \field{npm}\\
	Program Studi\hspace{0.8cm}: \field{prodi}
	\end{quote}
	saat ini sedang mengambil Matakuliah \field{matkul} dengan topik "\field{judulTopik}". Ybs. memerlukan informasi sehubungan dengan hal-hal yang terkait dengan masalah tersebut yang diterapkan di \field{instansi}.\

Untuk itu kami memohonkan ijin bagi mahasiswa kami tersebut agar diperbolehkan untuk melakukan \field{tujuanKunjungan} di \field{instansi}.\

Atas perhatian dan kerjasamanya yang diberikan kami mengucapkan terima kasih.
\vspace{2cm}
		\begin{flushright}
			Bandung, \field{tanggal}\\
					Wakil Dekan III\\
			\vspace{2cm}
			\underline{\field{wdIII}}
		\end{flushright}
\thispagestyle{empty}
	\newpage
}

\end{document}

\end{lstlisting}

\begin{lstlisting}[language=tex,basicstyle=
\tiny,caption=Format untuk surat pengantar studi lapangan untuk 2 orang]
	\documentclass[12pt]{letter}
\usepackage[legalpaper, top=55mm,bottom=15mm,left=35mm,right=18mm]{geometry}
\usepackage{mailmerge}

\begin{document}
\mailfields{noSurat,namaPemohon,npmPemohon,prodi,matkul,judulTopik,instansi,alamat,tujuanKunjungan,kepada,kota,namaAnggota,npmAnggota}

\mailrepeat{
Nomor : \field{noSurat}\\
Hal\hspace{0.75cm}: Permohonan \field{tujuanKunjungan}\\
Lamp\hspace{0.35cm}: -\\


\textbf{Kepada YTH.\\
KEPALA \field{kepada}\\
\field{alamat}\\
\field{kota}}

 Dengan hormat,\\

Bersama ini kami informasikan bahwa 2 orang mahasiswa kami, yaitu:

	\begin{quote}
		\begin{tabular}{|l|l|l|}
		\hline
		\textbf{No.}&\textbf{Nama}&\textbf{NPM}\\ \hline
		1&\field{namaPemohon}&\field{npmPemohon}\\ \hline
		2&\field{namaAnggota}&\field{npmAnggota}\\ \hline
	\end{tabular}
	\end{quote}
	saat ini sedang mengambil Matakuliah \field{matkul} dengan topik "\field{judulTopik}". Ybs. memerlukan informasi sehubungan dengan hal-hal yang terkait dengan masalah tersebut yang diterapkan di \field{instansi}.\

Untuk itu kami memohonkan ijin bagi mahasiswa kami tersebut agar diperbolehkan untuk melakukan \field{tujuanKunjungan} di \field{instansi}.\

Atas perhatian dan kerjasamanya yang diberikan kami mengucapkan terima kasih.
\vspace{2cm}
		\begin{flushright}
			Bandung, \field{tanggal}\\
					Wakil Dekan III\\
			\vspace{2cm}
			\underline{\field{wdIII}}
		\end{flushright}
\thispagestyle{empty}
	\newpage
}

\end{document}

\end{lstlisting}

\begin{lstlisting}[language=tex,basicstyle=
\tiny,caption=Format untuk surat pengantar studi lapangan untuk 3 orang]
	\documentclass[12pt]{letter}
\usepackage[legalpaper, top=55mm,bottom=15mm,left=35mm,right=18mm]{geometry}
\usepackage{mailmerge}

\begin{document}
\mailfields{noSurat,namaPemohon,npmPemohon,prodi,namaAnggota1,npmAnggota1,namaAnggota2,npmAnggota2,tujuanKunjungan,kepada,alamat,matkul,judulTopik,instansi,tanggal,wdIII,kota}

\mailrepeat{
Nomor : \field{noSurat}\\
Hal\hspace{0.75cm}: Permohonan \field{tujuanKunjungan}\\
Lamp\hspace{0.35cm}: -\\


\textbf{Kepada YTH.\\
KEPALA \field{kepada}\\
\field{alamat}\\
\field{kota}}

 Dengan hormat,\\

Bersama ini kami informasikan bahwa 3 orang mahasiswa kami, yaitu:

	\begin{quote}
		\textbf{\underline{KOORDINATOR}}\\
		Nama\hspace{2.4cm}: \field{namaPemohon}\\
		NPM\hspace{2.5cm}: \field{npmPemohon}\\
		Program Studi\hspace{0.8cm}: \field{prodi}\\

		\textbf{\underline{ANGGOTA}}\\
		\begin{tabular}{|l|l|l|l|}
		\hline
		\textbf{No.}&\textbf{Nama}&\textbf{NPM}\\ \hline
		1&\field{namaAnggota1}&\field{npmAnggota1}\\ \hline
		2&\field{namaAnggota2}&\field{npmAnggota2}\\ \hline
	\end{tabular}
	\end{quote}
	saat ini sedang mengambil Matakuliah \field{matkul} dengan topik "\field{judulTopik}". Ybs. memerlukan informasi sehubungan dengan hal-hal yang terkait dengan masalah tersebut yang diterapkan di \field{instansi}.\

Untuk itu kami memohonkan ijin bagi mahasiswa kami tersebut agar diperbolehkan untuk melakukan \field{tujuanKunjungan} di \field{instansi}.\

Atas perhatian dan kerjasamanya yang diberikan kami mengucapkan terima kasih.
\vspace{2cm}
		\begin{flushright}
			Bandung, \field{tanggal}\\
					Wakil Dekan III\\
			\vspace{2cm}
			\underline{\field{wdIII}}
		\end{flushright}
\thispagestyle{empty}
	\newpage
}

\end{document}

\end{lstlisting}

\begin{lstlisting}[language=tex,basicstyle=
\tiny,caption=Format untuk surat pengantar studi lapangan untuk 4 orang]
	\documentclass[12pt]{letter}
\usepackage[legalpaper, top=55mm,bottom=15mm,left=35mm,right=18mm]{geometry}
\usepackage{mailmerge}

\begin{document}

\mailfields{noSurat,namaPemohon,npmPemohon,prodi,namaAnggota1,npmAnggota1,namaAnggota2,npmAnggota2,namaAnggota3,npmAnggota3,tujuanKunjungan,kepada,alamat,matkul,judulTopik,instansi,tanggal,wdIII,kota}

\mailrepeat{
Nomor : \field{noSurat}\\
Hal\hspace{0.75cm}: Permohonan \field{tujuanKunjungan}\\
Lamp\hspace{0.35cm}: -\\


\textbf{Kepada YTH.\\
KEPALA \field{kepada}\\
\field{alamat}\\
\field{kota}}

 Dengan hormat,\\

Bersama ini kami informasikan bahwa 4 orang mahasiswa kami, yaitu:

	\begin{quote}
		\textbf{\underline{KOORDINATOR}}\\
		Nama\hspace{2.4cm}: \field{namaPemohon}\\
		NPM\hspace{2.5cm}: \field{npmPemohon}\\
		Program Studi\hspace{0.8cm}: \field{prodi}\\

		\textbf{\underline{ANGGOTA}}\\
		\begin{tabular}{|l|l|l|l|}
		\hline
		\textbf{No.}&\textbf{Nama}&\textbf{NPM}\\ \hline
		1&\field{namaAnggota1}&\field{npmAnggota1}\\ \hline
		2&\field{namaAnggota2}&\field{npmAnggota2}\\ \hline
		3&\field{namaAnggota3}&\field{npmAnggota3}\\ \hline
	\end{tabular}
	\end{quote}
	saat ini sedang mengambil Matakuliah \field{matkul} dengan topik "\field{judulTopik}". Ybs. memerlukan informasi sehubungan dengan hal-hal yang terkait dengan masalah tersebut yang diterapkan di \field{instansi}.\

Untuk itu kami memohonkan ijin bagi mahasiswa kami tersebut agar diperbolehkan untuk melakukan \field{tujuanKunjungan} di \field{instansi}.\

Atas perhatian dan kerjasamanya yang diberikan kami mengucapkan terima kasih.
\vspace{2cm}
		\begin{flushright}
			Bandung, \field{tanggal}\\
					Wakil Dekan III\\
			\vspace{2cm}
			\underline{\field{wdIII}}
		\end{flushright}
\thispagestyle{empty}
	\newpage
}

\end{document}

\end{lstlisting}

\begin{lstlisting}[language=tex,basicstyle=
\tiny,caption=Format untuk surat pengantar studi lapangan untuk 5 orang]
\documentclass[12pt]{letter}
\usepackage[legalpaper, top=55mm,bottom=15mm,left=35mm,right=18mm]{geometry}
\usepackage{mailmerge}

\begin{document}

\mailfields{noSurat,namaPemohon,npmPemohon,prodi,namaAnggota1,npmAnggota1,namaAnggota2,npmAnggota2,namaAnggota3,npmAnggota3,namaAnggota4,npmAnggota4,tujuanKunjungan,kepada,alamat,matkul,judulTopik,instansi,tanggal,wdIII,kota}

\mailrepeat{
Nomor : \field{noSurat}\\
Hal\hspace{0.75cm}: Permohonan \field{tujuanKunjungan}\\
Lamp\hspace{0.35cm}: -\\


\textbf{Kepada YTH.\\
KEPALA \field{kepada}\\
\field{alamat}\\
\field{kota}}

 Dengan hormat,\\

Bersama ini kami informasikan bahwa 5 orang mahasiswa kami, yaitu:

	\begin{quote}
		\textbf{\underline{KOORDINATOR}}\\
		Nama\hspace{2.4cm}: \field{namaPemohon}\\
		NPM\hspace{2.5cm}: \field{npmPemohon}\\
		Program Studi\hspace{0.8cm}: \field{prodi}\\

		\textbf{\underline{ANGGOTA}}\\
		\begin{tabular}{|l|l|l|l|}
		\hline
		\textbf{No.}&\textbf{Nama}&\textbf{NPM}\\ \hline
		1&\field{namaAnggota1}&\field{npmAnggota1}\\ \hline
		2&\field{namaAnggota2}&\field{npmAnggota2}\\ \hline
		3&\field{namaAnggota3}&\field{npmAnggota3}\\ \hline
		4&\field{namaAnggota4}&\field{npmAnggota4}\\ \hline
	\end{tabular}
	\end{quote}
	saat ini sedang mengambil Matakuliah \field{matkul} dengan topik "\field{judulTopik}". Ybs. memerlukan informasi sehubungan dengan hal-hal yang terkait dengan masalah tersebut yang diterapkan di \field{instansi}.\

Untuk itu kami memohonkan ijin bagi mahasiswa kami tersebut agar diperbolehkan untuk melakukan \field{tujuanKunjungan} di \field{instansi}.\

Atas perhatian dan kerjasamanya yang diberikan kami mengucapkan terima kasih.
\vspace{2cm}
		\begin{flushright}
			Bandung, \field{tanggal}\\
					Wakil Dekan III\\
			\vspace{2cm}
			\underline{\field{wdIII}}
		\end{flushright}
\thispagestyle{empty}
	\newpage
}

\end{document}


\end{lstlisting}

\begin{lstlisting}[language=tex,basicstyle=
\tiny,caption=Format untuk surat izin cuti studi]
\documentclass[12pt]{letter}
\usepackage[legalpaper, top=55mm,bottom=7mm,left=35mm,right=18mm]{geometry}
\usepackage{mailmerge}
\usepackage{changepage} %package utk atur margin

\begin{document}

\mailfields{noSurat,nama,NPM,tanggal,semester,thnAkademik,prodi,nextSemester,nextThnAkademik,dekan}

\mailrepeat{

	
	
	Nomor : \field{noSurat}\\
	Hal\hspace{0.75cm}: Surat Ijin Berhenti Studi Sementara\\
	Lamp\hspace*{0.35cm}: Surat Permohonan\\
	\vspace{0.5cm}
	
	Kepada : Yang terhormat\\
	\hspace*{1.7cm} \field{nama}\\
	\hspace*{1.7cm} NPM \field{NPM}\\
	\hspace*{1.7cm} di tempat\\
	\vspace{0.5cm}
	
	\begin{adjustwidth}{1.8cm}{0pt}Menurut surat permohonan cuti studi Anda tertanggal \field{tanggal}, dengan ini diberitahukan bahwa Anda diijinkan cuti studi untuk:
\end{adjustwidth}
		\begin{adjustwidth}{3cm}{0pt}
			\textbf{Semester \field{semester} tahun akademik \field{thnAkademik}}
		\end{adjustwidth}
		
		\begin{adjustwidth}{1.8cm}{0pt}
		Selanjutnya, untuk dapat mengikuti kegiatan akademik kembali, Anda wajib melakukan registrasi pada Masa Perwalian dan \textbf{FRS semester \field{nextSemester} tahun akademik \field{nextThnAkademik}} sesuai kalender akademik dan melaksanakan ketentuan yang telah ditetapkan.\\
		
		Demikian harap Anda maklum.
		\end{adjustwidth}
	\vspace{2cm}	
		\begin{flushright}
			Bandung,\field{tanggal}\\ 
					Dekan\\
			\vspace{2cm}
			\underline{\field{dekan}}
		\end{flushright}
		\vspace{5cm}
		\scriptsize {Tembusan :
		\begin{enumerate}
			\item WR-I, WR-II
			\item KBAA, KBTI
			\item Kajur \field{prodi}, Dosen Wali
			\item Map Ybs
			\item Arsip
		\end{enumerate}}

	\thispagestyle{empty}
		\newpage

}

\end{document}

\end{lstlisting}

\begin{lstlisting}[language=tex,basicstyle=
\tiny,caption=Format untuk surat izin pengunduran diri]
\documentclass[12pt]{letter}
\usepackage[legalpaper, top=55mm,bottom=7mm,left=35mm,right=18mm]{geometry}
\usepackage{mailmerge}
\usepackage{changepage} %package utk ubah margin

\begin{document}

\mailfields{noSurat,nama,NPM,prodi,tanggal,semester,tanggal,dekan}

\mailrepeat{

	
	
	Nomor : \field{noSurat}\\
	Hal\hspace{0.75cm}: Pengunduran diri\\
	Lamp\hspace*{0.35cm}: Surat pengunduran diri ybs.\\
	\vspace{0.5cm}
	
	Kepada : Yang terhormat\\
	\hspace*{1.7cm} Rektor\\
	\hspace*{1.7cm} Universitas Katolik Parahyangan\\
	\hspace*{1.7cm} Di tempat\\
	\vspace{0.5cm}
	
	\hspace*{1.8cm}Berdasarkan surat pengunduran diri :

		\begin{adjustwidth}{2.5cm}{0pt}
			\textbf{\field{nama}\\
			NPM  \field{NPM}\\
			Program Studi \field{prodi}}\\
		\end{adjustwidth}
		
		\begin{adjustwidth}{1.8cm}{0pt}
		tanggal \field{tanggal}, dengan ini kami nyatakan bahwa mahasiswa tersebut telah mengundurkan diri atas permintaan sendiri dari program studi FTIS terhitung mulai \textbf{\textit{semester \field{semester}}}.
		\end{adjustwidth}
		\vspace{2cm}
		\begin{flushright}
			Bandung, \field{tanggal}\\ 
					Dekan\\
			\vspace{2cm}
			\underline{\field{dekan}}
		\end{flushright}
		\vspace{5cm}
		{\scriptsize {Tembusan :
		\begin{enumerate}
			\item WR-I, WR-II
			\item KBAA, KBTI
			\item Kajur \field{prodi}, Dosen Wali
			\item Map Ybs
			\item Arsip
		\end{enumerate}}}

	\thispagestyle{empty}
		\newpage

}

\end{document}

\end{lstlisting}

\begin{lstlisting}[language=tex,basicstyle=
\tiny,caption=Format untuk surat perwakilan perwalian (ambil 1 mata kuliah)]
\documentclass[12pt]{letter}
\usepackage[legalpaper, top=25mm,bottom=15mm,left=35mm,right=18mm]{geometry}
\usepackage{mailmerge}

\begin{document}

\mailfields{semester,thnAKademik,namaPemohon,prodiPemohon,npmPemohon,namaWakil,prodiWakil,npmWakil,alasan,kodeMK,namaMK,sks}

\mailrepeat{

	\begin{center}
	{\Large \textbf{FORMULIR\\
	PERWALIAN YANG DIWAKILKAN}}
	\end{center}
	\vspace{0.25cm}
	Semester \hspace{1.5cm}: \field{semester} \\
	Tahun Akademik : \field{thnAkademik}\\

	{\footnotesize \textbf{IDENTITAS MAHASISWA YANG PERWALIANNYA DIWAKILKAN}}\\
	Nama \hspace{2.4cm}: \field{namaPemohon} \\
	Program Studi \hspace{0.8cm}: \field{prodiPemohon}\\
	NPM \hspace{2.5cm}: \field{npmPemohon}	\\
	{\footnotesize \textbf{IDENTITAS MAHASISWA YANG DIBERI KUASA PERWALIAN}}\\
	Nama \hspace{2.4cm}: \field{namaWakil} \\
	Program Studi \hspace{0.8cm}: \field{prodiWakil}\\
	NPM \hspace{2.5cm}: \field{npmWakil}\\

	Alasan tidak bisa hadir pada saat perwalian :\\
	\field{alasan}\\
	Mata kuliah yang diambil saat FRS : \\
	\begin{tabular}{|l|l|l|l|}
		\hline
		\textbf{No.}&\textbf{Kode MK}&\textbf{Nama Mata Kuliah}&\textbf{Sks}\\ \hline
		1&\field{kodeMK}&\field{namaMK}&\field{sks}\\ \hline
	\end{tabular}

\textbf{\underline{LAMPIRAN}} :
\begin{enumerate}
	\item Fotokopi KTM mahasiswa yag menerima kuasa
\end{enumerate}

		\begin{flushright}
			Bandung, \field{tanggal}
		\end{flushright}
Tanda Tangan \hspace{7.3cm}Tanda Tangan \\
Mahasiswa yang memberi kuasa \hspace{4cm} Mahasiswa yang menerima kuasa \\
\begin{flushright}
		\end{flushright}
			\vspace{1cm}

	(...........................................)\hspace{4.6cm} (...........................................)\\

	\textbf{Mengetahui dosen wali, \hspace{4.8cm} Menyetujui Wakil Dekan I}
	\vspace{2cm}

	(...........................................)\hspace{4.8cm}(\field{wdI})
	\thispagestyle{empty}
		\newpage

}
 

\end{document}

\end{lstlisting}

\begin{lstlisting}[language=tex,basicstyle=
\tiny,caption=Format untuk surat perwakilan perwalian (ambil 2 mata kuliah)]
\documentclass[12pt]{letter}
\usepackage[legalpaper, top=25mm,bottom=15mm,left=35mm,right=18mm]{geometry}
\usepackage{mailmerge}

\begin{document}

\mailfields{semester,thnAKademik,namaPemohon,prodiPemohon,npmPemohon,namaWakil,prodiWakil,npmWakil,alasan,kodeMK1,namaMK1,sks1,kodeMK2,namaMK2,sks2}

\mailrepeat{

	\begin{center}
	{\Large \textbf{FORMULIR\\
	PERWALIAN YANG DIWAKILKAN}}
	\end{center}
	\vspace{0.25cm}
	Semester \hspace{1.5cm}: \field{semester} \\
	Tahun Akademik : \field{thnAkademik}\\

	{\footnotesize \textbf{IDENTITAS MAHASISWA YANG PERWALIANNYA DIWAKILKAN}}\\
	Nama \hspace{2.4cm}: \field{namaPemohon} \\
	Program Studi \hspace{0.8cm}: \field{prodiPemohon}\\
	NPM \hspace{2.5cm}: \field{npmPemohon}	\\
	{\footnotesize \textbf{IDENTITAS MAHASISWA YANG DIBERI KUASA PERWALIAN}}\\
	Nama \hspace{2.4cm}: \field{namaWakil} \\
	Program Studi \hspace{0.8cm}: \field{prodiWakil}\\
	NPM \hspace{2.5cm}: \field{npmWakil}\\

	Alasan tidak bisa hadir pada saat perwalian :\\
	\field{alasan}\\
	Mata kuliah yang diambil saat FRS : \\
	\begin{tabular}{|l|l|l|l|}
		\hline
		\textbf{No.}&\textbf{Kode MK}&\textbf{Nama Mata Kuliah}&\textbf{Sks}\\ \hline
		1&\field{kodeMK1}&\field{namaMK1}&\field{sks1}\\ \hline
		2&\field{kodeMK2}&\field{namaMK2}&\field{sks2}\\ \hline
	\end{tabular}

\textbf{\underline{LAMPIRAN}} :
\begin{enumerate}
	\item Fotokopi KTM mahasiswa yag menerima kuasa
\end{enumerate}

		\begin{flushright}
			Bandung, \field{tanggal}
		\end{flushright}
Tanda Tangan \hspace{7.3cm}Tanda Tangan \\
Mahasiswa yang memberi kuasa \hspace{4cm} Mahasiswa yang menerima kuasa \\
\begin{flushright}
		\end{flushright}
			\vspace{1cm}

	(...........................................)\hspace{4.6cm} (...........................................)\\

	\textbf{Mengetahui dosen wali, \hspace{4.8cm} Menyetujui Wakil Dekan I}
	\vspace{2cm}

	(...........................................)\hspace{4.8cm}(\field{wdI})
	\thispagestyle{empty}
		\newpage

}

\end{document}

\end{lstlisting}




\begin{lstlisting}[language=tex,basicstyle=
\tiny,caption=Format untuk surat perwakilan perwalian (ambil 3 mata kuliah)]
\documentclass[12pt]{letter}
\usepackage[legalpaper, top=25mm,bottom=15mm,left=35mm,right=18mm]{geometry}
\usepackage{mailmerge}

\begin{document}

\mailfields{semester,thnAKademik,namaPemohon,prodiPemohon,npmPemohon,namaWakil,prodiWakil,npmWakil,alasan,kodeMK1,namaMK1,sks1,kodeMK2,namaMK2,sks2,kodeMK3,namaMK3,sks3}

\mailrepeat{

	\begin{center}
	{\Large \textbf{FORMULIR\\
	PERWALIAN YANG DIWAKILKAN}}
	\end{center}
	\vspace{0.25cm}
	Semester \hspace{1.5cm}: \field{semester} \\
	Tahun Akademik : \field{thnAkademik}\\

	{\footnotesize \textbf{IDENTITAS MAHASISWA YANG PERWALIANNYA DIWAKILKAN}}\\
	Nama \hspace{2.4cm}: \field{namaPemohon} \\
	Program Studi \hspace{0.8cm}: \field{prodiPemohon}\\
	NPM \hspace{2.5cm}: \field{npmPemohon}	\\
	{\footnotesize \textbf{IDENTITAS MAHASISWA YANG DIBERI KUASA PERWALIAN}}\\
	Nama \hspace{2.4cm}: \field{namaWakil} \\
	Program Studi \hspace{0.8cm}: \field{prodiWakil}\\
	NPM \hspace{2.5cm}: \field{npmWakil}\\

	Alasan tidak bisa hadir pada saat perwalian :\\
	\field{alasan}\\
	Mata kuliah yang diambil saat FRS : \\
	\begin{tabular}{|l|l|l|l|}
		\hline
		\textbf{No.}&\textbf{Kode MK}&\textbf{Nama Mata Kuliah}&\textbf{Sks}\\ \hline
		1&\field{kodeMK1}&\field{namaMK1}&\field{sks1}\\ \hline
		2&\field{kodeMK2}&\field{namaMK2}&\field{sks2}\\ \hline
		3&\field{kodeMK3}&\field{namaMK3}&\field{sks3}\\ \hline
	\end{tabular}

\textbf{\underline{LAMPIRAN}} :
\begin{enumerate}
	\item Fotokopi KTM mahasiswa yag menerima kuasa
\end{enumerate}

		\begin{flushright}
			Bandung, \field{tanggal}
		\end{flushright}
Tanda Tangan \hspace{7.3cm}Tanda Tangan \\
Mahasiswa yang memberi kuasa \hspace{4cm} Mahasiswa yang menerima kuasa \\
\begin{flushright}
		\end{flushright}
			\vspace{1cm}

	(...........................................)\hspace{4.6cm} (...........................................)\\

	\textbf{Mengetahui dosen wali, \hspace{4.8cm} Menyetujui Wakil Dekan I}
	\vspace{2cm}

	(...........................................)\hspace{4.8cm}(\field{wdI})
	\thispagestyle{empty}
		\newpage

}



\end{document}

\end{lstlisting}

\begin{lstlisting}[language=tex,basicstyle=
\tiny,caption=Format untuk surat perwakilan perwalian (ambil 4 mata kuliah)]
\documentclass[12pt]{letter}
\usepackage[legalpaper, top=25mm,bottom=15mm,left=35mm,right=18mm]{geometry}
\usepackage{mailmerge}

\begin{document}

\mailfields{semester,thnAKademik,namaPemohon,prodiPemohon,npmPemohon,namaWakil,prodiWakil,npmWakil,alasan,kodeMK1,namaMK1,sks1,kodeMK2,namaMK2,sks2,kodeMK3,namaMK3,sks3,kodeMK4,namaMK4,sks4,kodeMK5,namaMK5,sks5,kodeMK6,namaMK6,sks6}

\mailrepeat{

	\begin{center}
	{\Large \textbf{FORMULIR\\
	PERWALIAN YANG DIWAKILKAN}}
	\end{center}
	\vspace{0.25cm}
	Semester \hspace{1.5cm}: \field{semester} \\
	Tahun Akademik : \field{thnAkademik}\\

	{\footnotesize \textbf{IDENTITAS MAHASISWA YANG PERWALIANNYA DIWAKILKAN}}\\
	Nama \hspace{2.4cm}: \field{namaPemohon} \\
	Program Studi \hspace{0.8cm}: \field{prodiPemohon}\\
	NPM \hspace{2.5cm}: \field{npmPemohon}	\\
	{\footnotesize \textbf{IDENTITAS MAHASISWA YANG DIBERI KUASA PERWALIAN}}\\
	Nama \hspace{2.4cm}: \field{namaWakil} \\
	Program Studi \hspace{0.8cm}: \field{prodiWakil}\\
	NPM \hspace{2.5cm}: \field{npmWakil}\\

	Alasan tidak bisa hadir pada saat perwalian :\\
	\field{alasan}\\
	Mata kuliah yang diambil saat FRS : \\
	\begin{tabular}{|l|l|l|l|}
		\hline
		\textbf{No.}&\textbf{Kode MK}&\textbf{Nama Mata Kuliah}&\textbf{Sks}\\ \hline
		1&\field{kodeMK1}&\field{namaMK1}&\field{sks1}\\ \hline
		2&\field{kodeMK2}&\field{namaMK2}&\field{sks2}\\ \hline
		3&\field{kodeMK3}&\field{namaMK3}&\field{sks3}\\ \hline
		4&\field{kodeMK4}&\field{namaMK4}&\field{sks4}\\ \hline
		5&\field{kodeMK5}&\field{namaMK5}&\field{sks5}\\ \hline
	\end{tabular}

\textbf{\underline{LAMPIRAN}} :
\begin{enumerate}
	\item Fotokopi KTM mahasiswa yag menerima kuasa
\end{enumerate}

		\begin{flushright}
			Bandung, \field{tanggal}
		\end{flushright}
Tanda Tangan \hspace{7.3cm}Tanda Tangan \\
Mahasiswa yang memberi kuasa \hspace{4cm} Mahasiswa yang menerima kuasa \\
\begin{flushright}
		\end{flushright}
			\vspace{1cm}

	(...........................................)\hspace{4.6cm} (...........................................)\\

	\textbf{Mengetahui dosen wali, \hspace{4.8cm} Menyetujui Wakil Dekan I}
	\vspace{2cm}

	(...........................................)\hspace{4.8cm}(\field{wdI})
	\thispagestyle{empty}
		\newpage

}



\end{document}

\end{lstlisting}

\begin{lstlisting}[language=tex,basicstyle=
\tiny,caption=Format untuk surat perwakilan perwalian (ambil 5 mata kuliah)]
\documentclass[12pt]{letter}
\usepackage[legalpaper, top=25mm,bottom=15mm,left=35mm,right=18mm]{geometry}
\usepackage{mailmerge}

\begin{document}

\mailfields{semester,thnAKademik,namaPemohon,prodiPemohon,npmPemohon,namaWakil,prodiWakil,npmWakil,alasan,kodeMK1,namaMK1,sks1,kodeMK2,namaMK2,sks2,kodeMK3,namaMK3,sks3,kodeMK4,namaMK4,sks4,kodeMK5,namaMK5,sks5}

\mailrepeat{

	\begin{center}
	{\Large \textbf{FORMULIR\\
	PERWALIAN YANG DIWAKILKAN}}
	\end{center}
	\vspace{0.25cm}
	Semester \hspace{1.5cm}: \field{semester} \\
	Tahun Akademik : \field{thnAkademik}\\

	{\footnotesize \textbf{IDENTITAS MAHASISWA YANG PERWALIANNYA DIWAKILKAN}}\\
	Nama \hspace{2.4cm}: \field{namaPemohon} \\
	Program Studi \hspace{0.8cm}: \field{prodiPemohon}\\
	NPM \hspace{2.5cm}: \field{npmPemohon}	\\
	{\footnotesize \textbf{IDENTITAS MAHASISWA YANG DIBERI KUASA PERWALIAN}}\\
	Nama \hspace{2.4cm}: \field{namaWakil} \\
	Program Studi \hspace{0.8cm}: \field{prodiWakil}\\
	NPM \hspace{2.5cm}: \field{npmWakil}\\

	Alasan tidak bisa hadir pada saat perwalian :\\
	\field{alasan}\\
	Mata kuliah yang diambil saat FRS : \\
	\begin{tabular}{|l|l|l|l|}
		\hline
		\textbf{No.}&\textbf{Kode MK}&\textbf{Nama Mata Kuliah}&\textbf{Sks}\\ \hline
		1&\field{kodeMK1}&\field{namaMK1}&\field{sks1}\\ \hline
		2&\field{kodeMK2}&\field{namaMK2}&\field{sks2}\\ \hline
		3&\field{kodeMK3}&\field{namaMK3}&\field{sks3}\\ \hline
		4&\field{kodeMK4}&\field{namaMK4}&\field{sks4}\\ \hline
		5&\field{kodeMK5}&\field{namaMK5}&\field{sks5}\\ \hline
	\end{tabular}

\textbf{\underline{LAMPIRAN}} :
\begin{enumerate}
	\item Fotokopi KTM mahasiswa yag menerima kuasa
\end{enumerate}

		\begin{flushright}
			Bandung, \field{tanggal}
		\end{flushright}
Tanda Tangan \hspace{7.3cm}Tanda Tangan \\
Mahasiswa yang memberi kuasa \hspace{4cm} Mahasiswa yang menerima kuasa \\
\begin{flushright}
		\end{flushright}
			\vspace{1cm}

	(...........................................)\hspace{4.6cm} (...........................................)\\

	\textbf{Mengetahui dosen wali, \hspace{4.8cm} Menyetujui Wakil Dekan I}
	\vspace{2cm}

	(...........................................)\hspace{4.8cm}(\field{wdI})
	\thispagestyle{empty}
		\newpage

}


\end{document}

\end{lstlisting}

\begin{lstlisting}[language=tex,basicstyle=
\tiny,caption=Format untuk surat perwakilan perwalian (ambil 6 mata kuliah)]
\documentclass[12pt]{letter}
\usepackage[legalpaper, top=25mm,bottom=15mm,left=35mm,right=18mm]{geometry}
\usepackage{mailmerge}

\begin{document}

\mailfields{semester,thnAKademik,namaPemohon,prodiPemohon,npmPemohon,namaWakil,prodiWakil,npmWakil,alasan,kodeMK1,namaMK1,sks1,kodeMK2,namaMK2,sks2,kodeMK3,namaMK3,sks3,kodeMK4,namaMK4,sks4,kodeMK5,namaMK5,sks5,kodeMK6,namaMK6,sks6}

\mailrepeat{

	\begin{center}
	{\Large \textbf{FORMULIR\\
	PERWALIAN YANG DIWAKILKAN}}
	\end{center}
	\vspace{0.25cm}
	Semester \hspace{1.5cm}: \field{semester} \\
	Tahun Akademik : \field{thnAkademik}\\

	{\footnotesize \textbf{IDENTITAS MAHASISWA YANG PERWALIANNYA DIWAKILKAN}}\\
	Nama \hspace{2.4cm}: \field{namaPemohon} \\
	Program Studi \hspace{0.8cm}: \field{prodiPemohon}\\
	NPM \hspace{2.5cm}: \field{npmPemohon}	\\
	{\footnotesize \textbf{IDENTITAS MAHASISWA YANG DIBERI KUASA PERWALIAN}}\\
	Nama \hspace{2.4cm}: \field{namaWakil} \\
	Program Studi \hspace{0.8cm}: \field{prodiWakil}\\
	NPM \hspace{2.5cm}: \field{npmWakil}\\

	Alasan tidak bisa hadir pada saat perwalian :\\
	\field{alasan}\\
	Mata kuliah yang diambil saat FRS : \\
	\begin{tabular}{|l|l|l|l|}
		\hline
		\textbf{No.}&\textbf{Kode MK}&\textbf{Nama Mata Kuliah}&\textbf{Sks}\\ \hline
		1&\field{kodeMK1}&\field{namaMK1}&\field{sks1}\\ \hline
		2&\field{kodeMK2}&\field{namaMK2}&\field{sks2}\\ \hline
		3&\field{kodeMK3}&\field{namaMK3}&\field{sks3}\\ \hline
		4&\field{kodeMK4}&\field{namaMK4}&\field{sks4}\\ \hline
		5&\field{kodeMK5}&\field{namaMK5}&\field{sks5}\\ \hline
		6&\field{kodeMK6}&\field{namaMK6}&\field{sks6}\\ \hline
	\end{tabular}

\textbf{\underline{LAMPIRAN}} :
\begin{enumerate}
	\item Fotokopi KTM mahasiswa yag menerima kuasa
\end{enumerate}

		\begin{flushright}
			Bandung, \field{tanggal}
		\end{flushright}
Tanda Tangan \hspace{7.3cm}Tanda Tangan \\
Mahasiswa yang memberi kuasa \hspace{4cm} Mahasiswa yang menerima kuasa \\
\begin{flushright}
		\end{flushright}
			\vspace{1cm}

	(...........................................)\hspace{4.6cm} (...........................................)\\

	\textbf{Mengetahui dosen wali, \hspace{4.8cm} Menyetujui Wakil Dekan I}
	\vspace{2cm}

	(...........................................)\hspace{4.8cm}(\field{wdI})
	\thispagestyle{empty}
		\newpage

}


\end{document}

\end{lstlisting}

\begin{lstlisting}[language=tex,basicstyle=
\tiny,caption=Format untuk surat perwakilan perwalian (ambil 7 mata kuliah)]
\documentclass[12pt]{letter}
\usepackage[legalpaper, top=25mm,bottom=15mm,left=35mm,right=18mm]{geometry}
\usepackage{mailmerge}

\begin{document}

\mailfields{semester,thnAKademik,namaPemohon,prodiPemohon,npmPemohon,namaWakil,prodiWakil,npmWakil,alasan,kodeMK1,namaMK1,sks1,kodeMK2,namaMK2,sks2,kodeMK3,namaMK3,sks3,kodeMK4,namaMK4,sks4,kodeMK5,namaMK5,sks5,kodeMK6,namaMK6,sks6,kodeMK7,namaMK7,sks7}

\mailrepeat{

	\begin{center}
	{\Large \textbf{FORMULIR\\
	PERWALIAN YANG DIWAKILKAN}}
	\end{center}
	\vspace{0.25cm}
	Semester \hspace{1.5cm}: \field{semester} \\
	Tahun Akademik : \field{thnAkademik}\\

	{\footnotesize \textbf{IDENTITAS MAHASISWA YANG PERWALIANNYA DIWAKILKAN}}\\
	Nama \hspace{2.4cm}: \field{namaPemohon} \\
	Program Studi \hspace{0.8cm}: \field{prodiPemohon}\\
	NPM \hspace{2.5cm}: \field{npmPemohon}	\\
	{\footnotesize \textbf{IDENTITAS MAHASISWA YANG DIBERI KUASA PERWALIAN}}\\
	Nama \hspace{2.4cm}: \field{namaWakil} \\
	Program Studi \hspace{0.8cm}: \field{prodiWakil}\\
	NPM \hspace{2.5cm}: \field{npmWakil}\\

	Alasan tidak bisa hadir pada saat perwalian :\\
	\field{alasan}\\
	Mata kuliah yang diambil saat FRS : \\
	\begin{tabular}{|l|l|l|l|}
		\hline
		\textbf{No.}&\textbf{Kode MK}&\textbf{Nama Mata Kuliah}&\textbf{Sks}\\ \hline
		1&\field{kodeMK1}&\field{namaMK1}&\field{sks1}\\ \hline
		2&\field{kodeMK2}&\field{namaMK2}&\field{sks2}\\ \hline
		3&\field{kodeMK3}&\field{namaMK3}&\field{sks3}\\ \hline
		4&\field{kodeMK4}&\field{namaMK4}&\field{sks4}\\ \hline
		5&\field{kodeMK5}&\field{namaMK5}&\field{sks5}\\ \hline
		6&\field{kodeMK6}&\field{namaMK6}&\field{sks6}\\ \hline
		7&\field{kodeMK7}&\field{namaMK7}&\field{sks7}\\ \hline
	\end{tabular}

\textbf{\underline{LAMPIRAN}} :
\begin{enumerate}
	\item Fotokopi KTM mahasiswa yag menerima kuasa
\end{enumerate}

		\begin{flushright}
			Bandung, \field{tanggal}
		\end{flushright}
Tanda Tangan \hspace{7.3cm}Tanda Tangan \\
Mahasiswa yang memberi kuasa \hspace{4cm} Mahasiswa yang menerima kuasa \\
\begin{flushright}
		\end{flushright}
			\vspace{1cm}

	(...........................................)\hspace{4.6cm} (...........................................)\\

	\textbf{Mengetahui dosen wali, \hspace{4.8cm} Menyetujui Wakil Dekan I}
	\vspace{2cm}

	(...........................................)\hspace{4.8cm}(\field{wdI})
	\thispagestyle{empty}
		\newpage

}



\end{document}

\end{lstlisting}

\begin{lstlisting}[language=tex,basicstyle=
\tiny,caption=Format untuk surat perwakilan perwalian (ambil 8 mata kuliah)]
\documentclass[12pt]{letter}
\usepackage[legalpaper, top=25mm,bottom=15mm,left=35mm,right=18mm]{geometry}
\usepackage{mailmerge}

\begin{document}

\mailfields{semester,thnAKademik,namaPemohon,prodiPemohon,npmPemohon,namaWakil,prodiWakil,npmWakil,alasan,kodeMK1,namaMK1,sks1,kodeMK2,namaMK2,sks2,kodeMK3,namaMK3,sks3,kodeMK4,namaMK4,sks4,kodeMK5,namaMK5,sks5,kodeMK6,namaMK6,sks6,kodeMK7,namaMK7,sks7,kodeMK8,namaMK8,sks8}

\mailrepeat{

	\begin{center}
	{\Large \textbf{FORMULIR\\
	PERWALIAN YANG DIWAKILKAN}}
	\end{center}
	\vspace{0.25cm}
	Semester \hspace{1.5cm}: \field{semester} \\
	Tahun Akademik : \field{thnAkademik}\\

	{\footnotesize \textbf{IDENTITAS MAHASISWA YANG PERWALIANNYA DIWAKILKAN}}\\
	Nama \hspace{2.4cm}: \field{namaPemohon} \\
	Program Studi \hspace{0.8cm}: \field{prodiPemohon}\\
	NPM \hspace{2.5cm}: \field{npmPemohon}	\\
	{\footnotesize \textbf{IDENTITAS MAHASISWA YANG DIBERI KUASA PERWALIAN}}\\
	Nama \hspace{2.4cm}: \field{namaWakil} \\
	Program Studi \hspace{0.8cm}: \field{prodiWakil}\\
	NPM \hspace{2.5cm}: \field{npmWakil}\\

	Alasan tidak bisa hadir pada saat perwalian :\\
	\field{alasan}\\
	Mata kuliah yang diambil saat FRS : \\
	\begin{tabular}{|l|l|l|l|}
		\hline
		\textbf{No.}&\textbf{Kode MK}&\textbf{Nama Mata Kuliah}&\textbf{Sks}\\ \hline
		1&\field{kodeMK1}&\field{namaMK1}&\field{sks1}\\ \hline
		2&\field{kodeMK2}&\field{namaMK2}&\field{sks2}\\ \hline
		3&\field{kodeMK3}&\field{namaMK3}&\field{sks3}\\ \hline
		4&\field{kodeMK4}&\field{namaMK4}&\field{sks4}\\ \hline
		5&\field{kodeMK5}&\field{namaMK5}&\field{sks5}\\ \hline
		6&\field{kodeMK6}&\field{namaMK6}&\field{sks6}\\ \hline
		7&\field{kodeMK7}&\field{namaMK7}&\field{sks7}\\ \hline
		8&\field{kodeMK8}&\field{namaMK8}&\field{sks8}\\ \hline
	\end{tabular}

\textbf{\underline{LAMPIRAN}} :
\begin{enumerate}
	\item Fotokopi KTM mahasiswa yag menerima kuasa
\end{enumerate}

		\begin{flushright}
			Bandung, \field{tanggal}
		\end{flushright}
Tanda Tangan \hspace{7.3cm}Tanda Tangan \\
Mahasiswa yang memberi kuasa \hspace{4cm} Mahasiswa yang menerima kuasa \\
\begin{flushright}
		\end{flushright}
			\vspace{1cm}

	(...........................................)\hspace{4.6cm} (...........................................)\\

	\textbf{Mengetahui dosen wali, \hspace{4.8cm} Menyetujui Wakil Dekan I}
	\vspace{2cm}

	(...........................................)\hspace{4.8cm}(\field{wdI})
	\thispagestyle{empty}
		\newpage

}


\end{document}

\end{lstlisting}

\begin{lstlisting}[language=tex,basicstyle=
\tiny,caption=Format untuk surat perwakilan perwalian (ambil 9 mata kuliah)]
\documentclass[12pt]{letter}
\usepackage[legalpaper, top=25mm,bottom=15mm,left=35mm,right=18mm]{geometry}
\usepackage{mailmerge}

\begin{document}

\mailfields{semester,thnAKademik,namaPemohon,prodiPemohon,npmPemohon,namaWakil,prodiWakil,npmWakil,alasan,kodeMK1,namaMK1,sks1,kodeMK2,namaMK2,sks2,kodeMK3,namaMK3,sks3,kodeMK4,namaMK4,sks4,kodeMK5,namaMK5,sks5,kodeMK6,namaMK6,sks6,kodeMK7,namaMK7,sks7,kodeMK8,namaMK8,sks8,kodeMK9,namaMK9,sks9}

\mailrepeat{

	\begin{center}
	{\Large \textbf{FORMULIR\\
	PERWALIAN YANG DIWAKILKAN}}
	\end{center}
	\vspace{0.25cm}
	Semester \hspace{1.5cm}: \field{semester} \\
	Tahun Akademik : \field{thnAkademik}\\

	{\footnotesize \textbf{IDENTITAS MAHASISWA YANG PERWALIANNYA DIWAKILKAN}}\\
	Nama \hspace{2.4cm}: \field{namaPemohon} \\
	Program Studi \hspace{0.8cm}: \field{prodiPemohon}\\
	NPM \hspace{2.5cm}: \field{npmPemohon}	\\
	{\footnotesize \textbf{IDENTITAS MAHASISWA YANG DIBERI KUASA PERWALIAN}}\\
	Nama \hspace{2.4cm}: \field{namaWakil} \\
	Program Studi \hspace{0.8cm}: \field{prodiWakil}\\
	NPM \hspace{2.5cm}: \field{npmWakil}\\

	Alasan tidak bisa hadir pada saat perwalian :\\
	\field{alasan}\\
	Mata kuliah yang diambil saat FRS : \\
	\begin{tabular}{|l|l|l|l|}
		\hline
		\textbf{No.}&\textbf{Kode MK}&\textbf{Nama Mata Kuliah}&\textbf{Sks}\\ \hline
		1&\field{kodeMK1}&\field{namaMK1}&\field{sks1}\\ \hline
		2&\field{kodeMK2}&\field{namaMK2}&\field{sks2}\\ \hline
		3&\field{kodeMK3}&\field{namaMK3}&\field{sks3}\\ \hline
		4&\field{kodeMK4}&\field{namaMK4}&\field{sks4}\\ \hline
		5&\field{kodeMK5}&\field{namaMK5}&\field{sks5}\\ \hline
		6&\field{kodeMK6}&\field{namaMK6}&\field{sks6}\\ \hline
		7&\field{kodeMK7}&\field{namaMK7}&\field{sks7}\\ \hline
		8&\field{kodeMK8}&\field{namaMK8}&\field{sks8}\\ \hline
		9&\field{kodeMK9}&\field{namaMK9}&\field{sks9}\\ \hline
	\end{tabular}

\textbf{\underline{LAMPIRAN}} :
\begin{enumerate}
	\item Fotokopi KTM mahasiswa yag menerima kuasa
\end{enumerate}

		\begin{flushright}
			Bandung, \field{tanggal}
		\end{flushright}
Tanda Tangan \hspace{7.3cm}Tanda Tangan \\
Mahasiswa yang memberi kuasa \hspace{4cm} Mahasiswa yang menerima kuasa \\
\begin{flushright}
		\end{flushright}
			\vspace{1cm}

	(...........................................)\hspace{4.6cm} (...........................................)\\

	\textbf{Mengetahui dosen wali, \hspace{4.8cm} Menyetujui Wakil Dekan I}
	\vspace{2cm}

	(...........................................)\hspace{4.8cm}(\field{wdI})
	\thispagestyle{empty}
		\newpage

}


\end{document}

\end{lstlisting}

\begin{lstlisting}[language=tex,basicstyle=
\tiny,caption=Format untuk surat perwakilan perwalian (ambil 10 mata kuliah)]
\documentclass[12pt]{letter}
\usepackage[legalpaper, top=25mm,bottom=15mm,left=35mm,right=18mm]{geometry}
\usepackage{mailmerge}

\begin{document}

\mailfields{semester,thnAKademik,namaPemohon,prodiPemohon,npmPemohon,namaWakil,prodiWakil,npmWakil,alasan,kodeMK1,namaMK1,sks1,kodeMK2,namaMK2,sks2,kodeMK3,namaMK3,sks3,kodeMK4,namaMK4,sks4,kodeMK5,namaMK5,sks5,kodeMK6,namaMK6,sks6,kodeMK7,namaMK7,sks7,kodeMK8,namaMK8,sks8,kodeMK9,namaMK9,sks9,kodeMK10,namaMK10,sks10}

\mailrepeat{

	\begin{center}
	{\Large \textbf{FORMULIR\\
	PERWALIAN YANG DIWAKILKAN}}
	\end{center}
	\vspace{0.25cm}
	Semester \hspace{1.5cm}: \field{semester} \\
	Tahun Akademik : \field{thnAkademik}\\

	{\footnotesize \textbf{IDENTITAS MAHASISWA YANG PERWALIANNYA DIWAKILKAN}}\\
	Nama \hspace{2.4cm}: \field{namaPemohon} \\
	Program Studi \hspace{0.8cm}: \field{prodiPemohon}\\
	NPM \hspace{2.5cm}: \field{npmPemohon}	\\
	{\footnotesize \textbf{IDENTITAS MAHASISWA YANG DIBERI KUASA PERWALIAN}}\\
	Nama \hspace{2.4cm}: \field{namaWakil} \\
	Program Studi \hspace{0.8cm}: \field{prodiWakil}\\
	NPM \hspace{2.5cm}: \field{npmWakil}\\

	Alasan tidak bisa hadir pada saat perwalian :\\
	\field{alasan}\\
	Mata kuliah yang diambil saat FRS : \\
	\begin{tabular}{|l|l|l|l|}
		\hline
		\textbf{No.}&\textbf{Kode MK}&\textbf{Nama Mata Kuliah}&\textbf{Sks}\\ \hline
		1&\field{kodeMK1}&\field{namaMK1}&\field{sks1}\\ \hline
		2&\field{kodeMK2}&\field{namaMK2}&\field{sks2}\\ \hline
		3&\field{kodeMK3}&\field{namaMK3}&\field{sks3}\\ \hline
		4&\field{kodeMK4}&\field{namaMK4}&\field{sks4}\\ \hline
		5&\field{kodeMK5}&\field{namaMK5}&\field{sks5}\\ \hline
		6&\field{kodeMK6}&\field{namaMK6}&\field{sks6}\\ \hline
		7&\field{kodeMK7}&\field{namaMK7}&\field{sks7}\\ \hline
		8&\field{kodeMK8}&\field{namaMK8}&\field{sks8}\\ \hline
		9&\field{kodeMK9}&\field{namaMK9}&\field{sks9}\\ \hline
		10&\field{kodeMK10}&\field{namaMK10}&\field{sks10}\\ \hline
	\end{tabular}

\textbf{\underline{LAMPIRAN}} :
\begin{enumerate}
	\item Fotokopi KTM mahasiswa yag menerima kuasa
\end{enumerate}

		\begin{flushright}
			Bandung, \field{tanggal}
		\end{flushright}
Tanda Tangan \hspace{7.3cm}Tanda Tangan \\
Mahasiswa yang memberi kuasa \hspace{4cm} Mahasiswa yang menerima kuasa \\
\begin{flushright}
		\end{flushright}
			\vspace{1cm}

	(...........................................)\hspace{4.6cm} (...........................................)\\

	\textbf{Mengetahui dosen wali, \hspace{4.8cm} Menyetujui Wakil Dekan I}
	\vspace{2cm}

	(...........................................)\hspace{4.8cm}(\field{wdI})
	\thispagestyle{empty}
		\newpage

}


\end{document}


\end{lstlisting}