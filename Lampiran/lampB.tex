%versi 2 (8-10-2016)
\chapter{Kode Program \textit{Controller}}
\label{lamp:B}


\begin{lstlisting}[language=tex,basicstyle=\tiny,caption=FormatsuratController.php]
<?php

namespace App\Http\Controllers;

use Illuminate\Http\Request;
use App\Repositories\FormatsuratRepository;
use App\Formatsurat;
use Illuminate\Support\Facades\Auth;
use App\Mahasiswa;
use App\User;
use App\Dosen;
use App\TU;

class FormatsuratController extends Controller
{
    //
    protected $formatsuratRepo;

    public function __construct(FormatsuratRepository $formatsuratRepo){
      // dd($formatsuratRepo);
        $this->formatsuratRepo = $formatsuratRepo;
        //dd($this->orders->getAllActive());
    }

    public function tambahFormat(){
      $loggedInUser = Auth::user();
      // dd($loggedInUser);
      $realUser = $this->getRealUser($loggedInUser);
      return view('TU/tambah_format_surat',['user' => $realUser]);
    }

    /**
	 * Menampilkan seluruh format surat di halaman pilih jenis surat saat mahasiswa hendak memilih jenis surat
	 *
	 * @return view
	 */
	public function pilihSurat(Request $request){
        $formatsurats = $this->formatsuratRepo->tampilkanFormat();
        $loggedInUser = Auth::user();
        // dd($loggedInUser);
        $realUser = $this->getRealUser($loggedInUser);
        // dd($realUser);
        $foto = $realUser->foto_mahasiswa;
        if($request->jenis_surat == "surat_keterangan"){
          return view('mahasiswa.pilih_jenis_surat_keterangan',[
              'formatsurats' => $formatsurats,
              'user' => $realUser,
              'foto' => $foto
          ]);
        }
        else if($request->jenis_surat == "surat_pengantar"){
          return view('mahasiswa.pilih_jenis_surat_pengantar',[
              'formatsurats' => $formatsurats,
              'user' => $realUser,
              'foto' => $foto
          ]);
        }
        else if($request->jenis_surat == "surat_izin"){
          return view('mahasiswa.pilih_jenis_surat_izin',[
              'formatsurats' => $formatsurats,
              'user' => $realUser,
              'foto' => $foto
          ]);
        }
        else if($request->jenis_surat == "surat_perwakilan"){
          return view('mahasiswa.pilih_jenis_surat_perwakilan',[
              'formatsurats' => $formatsurats,
              'user' => $realUser,
              'foto' => $foto
          ]);
        }

	}

  private function getRealUser($loggedInUser){
    // dd($loggedInUser);
    $realUser='';
    // dd($realUser);
    if($loggedInUser->jabatan == User::JABATAN_MHS){
      $realUser = Mahasiswa::find($loggedInUser->ref);
      // dd($realUser);
      return $realUser;
    }else if($loggedInUser->jabatan == User::JABATAN_DOS){
      $realUser = Dosen::find($loggedInUser->ref);
      // dd($realUser);
      return $realUser;
    }else{ // TU
      $realUser = TU::find($loggedInUser->ref);
      // dd($loggedInUser->jabatan);
      return $realUser;
    }
    // dd($realUser);
  }

  /**
  * Menampilkan seluruh format surat di halaman format surat milik TU
  *
  * @return view
  */
public function tampilkanSeluruhFormat(Request $request){
      //$confirmation = Confirmation::where(['id' => 2])->first();

      //dd($confirmation->order->tickets);
      //dd($confirmation);
      //--
      $formatsurats;
      if($request->kategori_format_surat == "idFormatSurat"){
        $formatsurats = $this->formatsuratRepo->findFormatsuratByIdFormatSurat($request->searchBox_format_surat);
      }
      else if($request->kategori_format_surat == "jenis_surat"){
        $formatsurats = $this->formatsuratRepo->findFormatsuratByJenisSurat($request->searchBox_format_surat);
      }
      else if($request->kategori_format_surat == "keterangan"){
        $formatsurats = $this->formatsuratRepo->findFormatsuratByKeteranganSurat($request->searchBox_format_surat);
      }
      else{
        $formatsurats = $this->formatsuratRepo->findAllFormatsurat();
      }
      // dd($formatsurats);
      $loggedInUser = Auth::user();
      // dd($loggedInUser);
      $realUser = $this->getRealUser($loggedInUser);
      return view('TU.format_surat',[
          'formatsurats' => $formatsurats,
          'user' => $realUser
      ]);
    }

    /**
    * Delete the selected data
    */
    public function destroy(Request $request){
        //
        // dd($request->deleteID);
        $formatsurat = $this->getModel($request->deleteID);
        $formatsurat->delete();
        return redirect('/format_surat')->with('success_message', 'Format surat <b>#' . $request->id . '</b> berhasil dihapus.');
    }

    /**
    * Get mahasiswa model by Id
    * @return Mahasiswa
    */
    private function getModel($id){
        $model = $this->formatsuratRepo->findById($id);
        if($model === null){
            abort(404);
        }
        return $model;
    }

   public function update(Request $request){
        $formatsurat = $this->findById($request->id);
        $format = $request->file('uploadFormat');
        // dd($format);
        $destination_path = 'format_surat_latex/';
        $filename = $format->getClientOriginalName();
        // dd($filename);
        $format->move($destination_path, $filename);

        //store to db
        $formatsurat->idFormatSurat = $request->idFormatSurat;
        $formatsurat->jenis_surat = $request->jenis_surat;
        $formatsurat->keterangan = $request->keterangan;
        $formatsurat->link_format_surat = '127.0.0.1:8000/format_surat_latex/' . $filename;
        $formatsurat->save();
   }

    public function storeFormat(Request $request){
        $formatsurat = new Formatsurat;

        //upload
        $format = $request->file('uploadFormat');
        // dd($format);
        $destination_path = 'format_surat_latex/';
        $filename = $format->getClientOriginalName();
        // dd($filename);
        $format->move($destination_path, $filename);

        //store to db
        $formatsurat->idFormatSurat = $request->idFormatSurat;
        $formatsurat->jenis_surat = $request->jenis_surat;
        $formatsurat->keterangan = $request->keterangan;
        $formatsurat->link_format_surat = '127.0.0.1:8000/format_surat_latex/' . $filename;
        $formatsurat->save();

        return redirect('/format_surat')->with('success_message', 'Surat' . $formatsurat->jenis_surat . 'berhasil dibuat');
    }

    /**
    * Untuk menampilkan formulir berdasarkan jenis surat yang dipilih oleh mahasiswa
    */
    public function tampilkanFormulir(Request $request){
      $loggedInUser = Auth::user();
      $realUser = $this->getRealUser($loggedInUser);
      $foto = $realUser->foto_mahasiswa;
      // dd($loggedInUser);
      $realUser = $this->getRealUser($loggedInUser);
      // dd($realUser);

        if($request->jenis_surat == "1"){
          return view('mahasiswa.data_keterangan_beasiswa', [
            'formatsurat_id' => $request->jenis_surat,
            'user' => $realUser,
            'foto' => $foto
          ]);
        }
        else if($request->jenis_surat == "2"){
          return view('mahasiswa.data_keterangan_mahasiswa_aktif', [
            'formatsurat_id' => $request->jenis_surat,
            'user' => $realUser,
            'foto' => $foto
          ]);
        }
        else if($request->jenis_surat == "3"){
          return view('mahasiswa.data_pembuatan_visa', [
            'formatsurat_id' => $request->jenis_surat,
            'user' => $realUser,
            'foto' => $foto
          ]);
        }
        else if($request->jenis_surat == "4"){
          // dd($request->jenis_surat);
          return view('mahasiswa.data_izin_studi_lapangan_1org', [
            'formatsurat_id' => $request->jenis_surat,
            'user' => $realUser,
            'foto' => $foto
          ]);
        }
        else if($request->jenis_surat == "5"){
          return view('mahasiswa.data_izin_studi_lapangan_2org', [
            'formatsurat_id' => $request->jenis_surat,
            'user' => $realUser,
            'foto' => $foto
          ]);
        }
        else if($request->jenis_surat == "6"){
          return view('mahasiswa.data_izin_studi_lapangan_3org', [
            'formatsurat_id' => $request->jenis_surat,
            'user' => $realUser,
            'foto' => $foto
          ]);
        }
        else if($request->jenis_surat == "7"){
          return view('mahasiswa.data_izin_studi_lapangan_4org', [
            'formatsurat_id' => $request->jenis_surat,
            'user' => $realUser,
            'foto' => $foto
          ]);
        }
        else if($request->jenis_surat == "8"){
          return view('mahasiswa.data_izin_studi_lapangan_5org', [
            'formatsurat_id' => $request->jenis_surat,
            'user' => $realUser,
            'foto' => $foto
          ]);
        }
        else if($request->jenis_surat == "9"){
          return view('mahasiswa.data_izin_cuti_studi', [
            'formatsurat_id' => $request->jenis_surat,
            'user' => $realUser,
            'foto' => $foto
          ]);
        }
        else if($request->jenis_surat == "10"){
          return view('mahasiswa.data_izin_pengunduran_diri', [
            'formatsurat_id' => $request->jenis_surat,
            'user' => $realUser,
            'foto' => $foto
          ]);
        }
        else if($request->jenis_surat == "11"){
          return view('mahasiswa.data_perwakilan_perwalian_1mk', [
            'formatsurat_id' => $request->jenis_surat,
            'user' => $realUser,
            'foto' => $foto
          ]);
        }
        else if($request->jenis_surat == "12"){
          return view('mahasiswa.data_perwakilan_perwalian_2mk', [
            'formatsurat_id' => $request->jenis_surat,
            'user' => $realUser,
            'foto' => $foto
          ]);
        }
        else if($request->jenis_surat == "13"){
          return view('mahasiswa.data_perwakilan_perwalian_3mk', [
            'formatsurat_id' => $request->jenis_surat,
            'user' => $realUser,
            'foto' => $foto
          ]);
        }
        else if($request->jenis_surat == "14"){
          return view('mahasiswa.data_perwakilan_perwalian_4mk', [
            'formatsurat_id' => $request->jenis_surat,
            'user' => $realUser,
            'foto' => $foto
          ]);
        }
        else if($request->jenis_surat == "15"){
          return view('mahasiswa.data_perwakilan_perwalian_5mk', [
            'formatsurat_id' => $request->jenis_surat,
            'user' => $realUser,
            'foto' => $foto
          ]);
        }
        else if($request->jenis_surat == "16"){
          return view('mahasiswa.data_perwakilan_perwalian_6mk', [
            'formatsurat_id' => $request->jenis_surat,
            'user' => $realUser,
            'foto' => $foto
          ]);
        }
        else if($request->jenis_surat == "17"){
          return view('mahasiswa.data_perwakilan_perwalian_7mk', [
            'formatsurat_id' => $request->jenis_surat,
            'user' => $realUser,
            'foto' => $foto
          ]);
        }
        else if($request->jenis_surat == "18"){
          return view('mahasiswa.data_perwakilan_perwalian_8mk', [
            'formatsurat_id' => $request->jenis_surat,
            'user' => $realUser,
            'foto' => $foto
          ]);
        }
        else if($request->jenis_surat == "19"){
          return view('mahasiswa.data_perwakilan_perwalian_9mk', [
            'formatsurat_id' => $request->jenis_surat,
            'user' => $realUser,
            'foto' => $foto
          ]);
        }
        else if($request->jenis_surat == "20"){
          return view('mahasiswa.data_perwakilan_perwalian_10mk', [
            'formatsurat_id' => $request->jenis_surat,
            'user' => $realUser,
            'foto' => $foto
          ]);
        }
    }
}
\end{lstlisting}

\begin{lstlisting}[language=tex,basicstyle=\tiny,caption=HistorysuratController.php]
<?php

namespace App\Http\Controllers;

use Illuminate\Http\Request;
use App\Repositories\PesanansuratRepository;
use App\Repositories\HistorysuratRepository;
use App\Repositories\JurusanRepository;
use App\Repositories\FormatsuratRepository;
use App\Mahasiswa;
use App\Formatsurat;
use App\Historysurat;
use Storage;
use Illuminate\Support\Facades\Auth;
use App\Dosen;
use App\User;
use App\TU;
class HistorysuratController extends Controller
{
    //
    protected $historysuratRepo;
    protected $pesananSuratRepo;
    protected $formatSuratRepo;
    protected $jurusanRepo;

    public function __construct(HistorysuratRepository $historysuratRepo, PesanansuratRepository $pesananSuratRepo, FormatsuratRepository $formatsuratRepo, JurusanRepository $jurusanRepo){
      // dd($formatsuratRepo);
        $this->historysuratRepo = $historysuratRepo;
        $this->pesananSuratRepo = $pesananSuratRepo;
        $this->formatsuratRepo = $formatsuratRepo;
        $this->jurusanRepo = $jurusanRepo;
        //dd($this->orders->getAllActive());
    }

    public function tampilkanHistoryDiPejabat(Request $request){
          $historysurats;
          if($request->kategori_history_surat == "noSurat"){
            $historysurats = $this->historysuratRepo->findHistorySuratByNomorSurat($request->searchBox);
          }
          else if($request->kategori_history_surat == "jenis_surat"){
            $historysurats = $this->historysuratRepo->findHistoryByJenisSurat($request->searchBox);
          }
          else if($request->kategori_history_surat == "perihal"){
            $historysurats = $this->historysuratRepo->findHistoryByPerihal($request->searchBox);
          }
          else if($request->kategori_history_surat == "penerima_surat"){
            $historysurats = $this->historysuratRepo->findHistorySuratByPenerimaSurat($request->searchBox);
          }
          else if($request->kategori_history_surat == "pengirimSurat"){
            $historysurats = $this->historysuratRepo->findHistoryByPengirimSurat($request->searchBox);
          }
          else if($request->kategori_history_surat == "tanggalPembuatan"){
            $historysurats = $this->historysuratRepo->findHistoryByTanggalPembuatan($request->searchBox);
          }
          else{
            $historysurats = $this->historysuratRepo->findAllHistorysurat();
          }
          // dd($formatsurats);
          $loggedInUser = Auth::user();
          $realUser = $this->getRealUser($loggedInUser);
          return view('pejabat.history_pejabat', [
            'historysurats' => $historysurats,
            'user' => $realUser
          ]);
  	}

    public function tampilkanProfil(){
      $loggedInUser = Auth::user();
      // dd($loggedInUser);
      $realUser = $this->getRealUser($loggedInUser);
      // dd($realUser);
      // dd($realUser->historysurats);
      $histories = $realUser->historysurats;
      // dd($histories);
      // $jenis = $this->historysuratRepo->findHistoryById($history->id)
      // dd($history);
      $foto = $realUser->foto_mahasiswa;
      // dd($foto);
      return view('mahasiswa.home_mahasiswa',[
        'user' => $realUser,
        'historysurats' => $histories
      ]);
    }

    public function ubahStatusPengambilan(Request $request){
      $history = $this->historysuratRepo->findHistoryById($request->id);
      // dd($history);
      $history->pengambilan = true;
      $history->save();
      return redirect('/history_TU');
    }

    public function ubahStatusPenandatanganan(Request $request){
      $history = $this->historysuratRepo->findHistoryById($request->id);
      $history->penandatanganan = true;
      $history->save();
      return redirect('/history_pejabat');
    }

    private function getRealUser($loggedInUser){
      // dd($loggedInUser);
      $realUser='';
      // dd($realUser);
      if($loggedInUser->jabatan == User::JABATAN_MHS){
        $realUser = Mahasiswa::find($loggedInUser->ref);
        // dd($realUser);
        return $realUser;
      }else if($loggedInUser->jabatan == User::JABATAN_DOS){
        $realUser = Dosen::find($loggedInUser->ref);
        // dd($realUser);
        return $realUser;
      }else{ // TU
        $realUser = TU::find($loggedInUser->ref);
        // dd($loggedInUser->jabatan);
        return $realUser;
      }
      // dd($realUser);
    }

    /**
	 * Display a listing of the resource.
	 *
	 * @return Response
	 */
	public function pilihHistorySurat(Request $request){
        $historysurats;
        if($request->kategori_history_surat == "noSurat"){
          $historysurats = $this->historysuratRepo->findHistorySuratByNomorSurat($request->searchBox);
        }
        else if($request->kategori_history_surat == "jenis_surat"){
          $historysurats = $this->historysuratRepo->findHistoryByJenisSurat($request->searchBox);
        }
        else if($request->kategori_history_surat == "perihal"){
          $historysurats = $this->historysuratRepo->findHistoryByPerihal($request->searchBox);
        }
        else if($request->kategori_history_surat == "penerima_surat"){
          $historysurats = $this->historysuratRepo->findHistorySuratByPenerimaSurat($request->searchBox);
        }
        else if($request->kategori_history_surat == "pengirimSurat"){
          $historysurats = $this->historysuratRepo->findHistoryByPengirimSurat($request->searchBox);
        }
        else if($request->kategori_history_surat == "tanggalPembuatan"){
          $historysurats = $this->historysuratRepo->findHistoryByTanggalPembuatan($request->searchBox);
        }
        else{
          $historysurats = $this->historysuratRepo->findAllHistorysurat();
        }
        // dd($formatsurats);
        $loggedInUser = Auth::user();
        $realUser = $this->getRealUser($loggedInUser);
        $foto = $realUser->foto_mahasiswa;
        return view('TU.history_TU', [
          'historysurats' => $historysurats,
          'user' => $realUser,
          'foto' => $foto
        ]);
	}

  public function buatPDF(Request $request){
      $loggedInUser = Auth::user();
      $realUser = $this->getRealUser($loggedInUser);
      if($request->idFormatSurat == "1"){
        $dataSurat = $request->data;
        $json = json_decode($dataSurat);
        $noSurat = $request->noSurat;
        $nama = $json->nama;
        $prodi = $this->jurusanRepo->findJurusanById($json->prodi);
        $npm = $json->npm;
        $semester = $json->semester;
        $thnAkademik = $json->thnAkademik;
        $penyediabeasiswa = $json->penyediabeasiswa;
        $pemesan = $request->pemesan;
        // $tanggal = $this->pesananSuratRepo->findHistorySuratById($request->id)->created_at;

        $entry = '\mailentry{' . $noSurat . ',' . $nama . ',' . $prodi . ',' . $npm . ',' . $semester . ',' . $thnAkademik . ',' . $penyediabeasiswa . '}';
        // dd($entry);
        $fileTemplate = file('format_surat_latex/surat_keterangan_beasiswa.tex');
        $stringFormat = "";
        $baris = count($fileTemplate);
        // dd($baris);
        foreach ($fileTemplate as $line_num => $line) {
            // dd($line);
            $stringFormat .= $line;
            if($line_num == $baris-3){
                $stringFormat .= $entry;
            }
        }
        // dd($stringFormat);
        $file = fopen("arsip_surat/" . $noSurat. "_" . $npm . "_surat_keterangan_beasiswa.tex", "w");
        fwrite($file, $stringFormat);
        fclose($file);
        shell_exec('pdflatex -output-directory arsip_surat arsip_surat/' . $noSurat. '_' . $npm . '_surat_keterangan_beasiswa.tex');

        //store to db
        $historysurat = new Historysurat;
        $historysurat->no_surat = $noSurat;
        $historysurat->perihal = '-';
        $historysurat->penerimaSurat = $json->penyediabeasiswa;
        $historysurat->mahasiswa_id = $pemesan;
        $historysurat->formatsurats_id = $request->idFormatSurat;
        $historysurat->link_arsip_surat = '127.0.0.1:8000/arsip_surat/' . $noSurat. '_' . $npm . '_surat_keterangan_beasiswa.pdf';
        $historysurat->penandatanganan = false;
        $historysurat->pengambilan = false;
        $historysurat->save();
        return redirect('/history_TU');//->with('success_message', 'Surat berhasil dibuat!');
      }
      else if($request->idFormatSurat == "2"){
        $dataSurat = $request->data;
        $json = json_decode($dataSurat);
        $noSurat = $request->noSurat;
        $nama = $json->nama;
        $prodi = $json->prodi;
        $npm = $json->npm;
        $kota_lahir = $json->kota_lahir;
        $tglLahir = $json->tglLahir;
        $alamat = $json->alamat;
        $semester = $json->semester;
        $pemesan = $request->pemesan;

        //input entry
        $entry = '\mailentry{' . $noSurat . ',' . $nama . ',' . $prodi . ',' . $npm . ',' . $kota_lahir . ',' . $tglLahir . ',' . $alamat . ',' . $semester . '}';
        // dd($entry);
        $fileTemplate = file('format_surat_latex/surat_keterangan_mahasiswa_aktif.tex');
        $stringFormat = "";
        $baris = count($fileTemplate);
        // dd($baris);
        foreach ($fileTemplate as $line_num => $line) {
            // dd($line);
            $stringFormat .= $line;
            if($line_num == $baris-3){
                $stringFormat .= $entry;
            }
        }
        // dd($stringFormat);
        //inject ke file baru
        $file = fopen("arsip_surat/" . $noSurat. "_" . $npm . "_surat_keterangan_mahasiswa_aktif.tex", "w");
        fwrite($file, $stringFormat);
        fclose($file);
        shell_exec('pdflatex -output-directory arsip_surat arsip_surat/' . $noSurat . '_' . $npm . '_surat_keterangan_mahasiswa_aktif.tex');

        //store to db
        $historysurat = new Historysurat;
        $historysurat->no_surat = $noSurat;
        $historysurat->perihal = '-';
        $historysurat->penerimaSurat = $nama;
        $historysurat->mahasiswa_id = $pemesan;
        $historysurat->formatsurats_id = $request->idFormatSurat;
        $historysurat->link_arsip_surat = '127.0.0.1:8000/arsip_surat/' . $noSurat. '_' . $npm . '_surat_keterangan_mahasiswa_aktif.pdf';
        $historysurat->penandatanganan = false;
        $historysurat->pengambilan = false;
        $historysurat->save();
        return redirect('/history_TU');
      }
      else if($request->idFormatSurat == "3"){
        $dataSurat = $request->data;
        $json = json_decode($dataSurat);
        $noSurat = $request->noSurat;
        $nama = $json->nama;
        $tglLahir = $json->tglLahir;
        $organisasiTujuan = $json->organisasiTujuan;
        $thnAkademik = $json->thnAkademik;
        $negaraTujuan = $json->negaraTujuan;
        $tanggalKunjungan = $json->tanggalKunjungan;
        $pemesan = $request->pemesan;
        $npm = $this->mahasiswaRepo->findMahasiswaById($pemesan);

        $entry = '\mailentry{' . $noSurat . ',' . $nama . ',' . $tglLahir . ',' . $organisasiTujuan . ',' . $thnAkademik . ',' . $negaraTujuan . ',' . $tanggalKunjungan . '}';
        $fileTemplate = file('format_surat_latex/surat_pengantar_pembuatan_visa.tex');
        $stringFormat = "";
        $baris = count($fileTemplate);
        // dd($baris);
        foreach ($fileTemplate as $line_num => $line) {
            // dd($line);
            $stringFormat .= $line;
            if($line_num == $baris-3){
                $stringFormat .= $entry;
            }
        }
        // dd($stringFormat);

        $file = fopen("arsip_surat/" . $noSurat. "_" . $npm . "_surat_pengantar_pembuatan_visa.tex", "w");
        fwrite($file, $stringFormat);
        fclose($file);
        shell_exec('pdflatex -output-directory arsip_surat arsip_surat/' . $noSurat . '_' . $npm . '_surat_pengantar_pembuatan_visa.tex');

        //store to db
        $historysurat = new Historysurat;
        $historysurat->no_surat = $noSurat;
        $historysurat->perihal = 'APPLICATION FOR VISA SCHENGEN';
        $historysurat->penerimaSurat = $organisasiTujuan;
        $historysurat->mahasiswa_id = $pemesan;
        $historysurat->formatsurats_id = $request->idFormatSurat;
        $historysurat->link_arsip_surat = '127.0.0.1:8000/arsip_surat/' . $noSurat. '_' . $npm . '_surat_pengantar_pembuatan_visa.pdf';
        $historysurat->penandatanganan = false;
        $historysurat->pengambilan = false;
        $historysurat->save();
        return redirect('/history_TU');
      }
      else if($request->idFormatSurat == "4"){
        $dataSurat = $request->data;
        $json = json_decode($dataSurat);
        $noSurat = $request->noSurat;
        $nama = $json->nama;
        $npm = $json->npm;
        $prodi = $json->prodi;
        $matkul = $json->matkul;
        $topik = $json->topik;
        $organisasi = $json->organisasi;
        $alamatOrganisasi = $json->alamatOrganisasi;
        $kepada = $json->kepada;
        $kota = $json->kota;
        $keperluanKunjungan = $json->keperluanKunjungan;
        $pemesan = $request->pemesan;

        $entry = '\mailentry{' .
          $noSurat . ',' . $nama . ',' . $npm . ',' . $prodi . ',' . $matkul . ',' . $topik . ',' . $organisasi . ',' . $alamatOrganisasi . ',' . $keperluanKunjungan . ',' .$kota . ',' . $kepada . ',' . '}';
        $fileTemplate = file('format_surat_latex/surat_pengantar_studi_lapangan_1orang.tex');
        $stringFormat = "";
        $baris = count($fileTemplate);
        // dd($baris);
        foreach ($fileTemplate as $line_num => $line) {
            // dd($line);
            $stringFormat .= $line;
            if($line_num == $baris-3){
                $stringFormat .= $entry;
            }
        }
        $file = fopen("arsip_surat/" . $noSurat. "_" . $npm . "_surat_pengantar_studi_lapangan_1orang.tex", "w");
        fwrite($file, $stringFormat);
        fclose($file);
        shell_exec('pdflatex -output-directory arsip_surat arsip_surat/' . $noSurat . '_' . $npm . '_surat_pengantar_studi_lapangan_1orang.tex');

        //store to db
        $historysurat = new Historysurat;
        $historysurat->no_surat = $noSurat;
        $historysurat->perihal = 'Permohonan' . $keperluanKunjungan;
        $historysurat->penerimaSurat = $kepada;
        $historysurat->mahasiswa_id = $pemesan;
        $historysurat->formatsurats_id = $request->idFormatSurat;
        $historysurat->link_arsip_surat = '127.0.0.1:8000/arsip_surat/' . $noSurat. '_' . $npm . '_surat_pengantar_studi_lapangan_1orang.pdf';
        $historysurat->penandatanganan = false;
        $historysurat->pengambilan = false;
        $historysurat->save();
        return redirect('/history_TU');
      }
      else if($request->idFormatSurat == "5"){
        $dataSurat = $request->data;
        $json = json_decode($dataSurat);
        $noSurat = $request->noSurat;
        $nama = $json->nama;
        $npm = $json->npm;
        $prodi = $json->prodi;
        $matkul = $json->matkul;
        $topik = $json->topik;
        $organisasi = $json->organisasi;
        $alamatOrganisasi = $json->alamatOrganisasi;
        $keperluanKunjungan = $json->keperluanKunjungan;
        $kepada = $json->kepada;
        $kota = $json->kota;
        $namaAnggota = $json->namaAnggota;
        $npmAnggota = $json->npmAnggota;
        $pemesan = $request->pemesan;

        $entry = '\mailentry{' .
          $noSurat . ',' .
          $nama . ',' .
          $npm . ',' .
          $prodi . ',' .
          $matkul . ',' .
          $topik . ',' .
          $organisasi . ',' .
          $alamatOrganisasi . ',' .
          $keperluanKunjungan . ',' .
          $kota . ',' .
          $kepada . ',' .
          $namaAnggota . ',' .
          $npmAnggota . ',' .
          '}';
        $fileTemplate = file('format_surat_latex/surat_pengantar_studi_lapangan_2orang.tex');
        $stringFormat = "";
        $baris = count($fileTemplate);
        // dd($baris);
        foreach ($fileTemplate as $line_num => $line) {
            // dd($line);
            $stringFormat .= $line;
            if($line_num == $baris-3){
                $stringFormat .= $entry;
            }
        }
        // dd($stringFormat);
        $file = fopen("arsip_surat/" . $noSurat. "_" . $npm . "_surat_pengantar_studi_lapangan_2orang.tex", "w");
        fwrite($file, $stringFormat);
        fclose($file);
        shell_exec('pdflatex -output-directory arsip_surat arsip_surat/' . $noSurat . '_' . $npm . '_surat_pengantar_studi_lapangan_2orang.tex');

        //store to db
        $historysurat = new Historysurat;
        $historysurat->no_surat = $noSurat;
        $historysurat->perihal = 'Permohonan' . $keperluanKunjungan;
        $historysurat->penerimaSurat = $kepada;
        $historysurat->mahasiswa_id = $pemesan;
        $historysurat->formatsurats_id = $request->idFormatSurat;
        $historysurat->link_arsip_surat = '127.0.0.1:8000/arsip_surat/' . $noSurat. '_' . $npm . '_surat_pengantar_studi_lapangan_2orang.pdf';
        $historysurat->penandatanganan = false;
        $historysurat->pengambilan = false;
        $historysurat->save();
        return redirect('/history_TU');
      }
      else if($request->idFormatSurat == "6"){
        $dataSurat = $request->data;
        $json = json_decode($dataSurat);
        $noSurat = $request->noSurat;
        $nama = $json->nama;
        $npm = $json->npm;
        $prodi = $json->prodi;
        $matkul = $json->matkul;
        $topik = $json->topik;
        $organisasi = $json->organisasi;
        $alamatOrganisasi = $json->alamatOrganisasi;
        $keperluanKunjungan = $json->keperluanKunjungan;
        $kepada = $json->kepada;
        $kota = $json->kota;
        $namaAnggota1 = $json->namaAnggota1;
        $npmAnggota1 = $json->npmAnggota1;
        $namaAnggota2 = $json->namaAnggota2;
        $npmAnggota2 = $json->npmAnggota2;
        $pemesan = $request->pemesan;

        $entry = '\mailentry{' .
          $noSurat . ',' .
          $nama . ',' .
          $npm . ',' .
          $prodi . ',' .
          $matkul . ',' .
          $topik . ',' .
          $organisasi . ',' .
          $alamatOrganisasi . ',' .
          $keperluanKunjungan . ',' .
          $kota . ',' .
          $kepada . ',' .
          $namaAnggota1 . ',' .
          $npmAnggota1 . ',' .
          $namaAnggota2 . ',' .
          $npmAnggota2 . ',' .
          '}';
        $fileTemplate = file('format_surat_latex/surat_pengantar_studi_lapangan_3orang.tex');
        $stringFormat = "";
        $baris = count($fileTemplate);
        // dd($baris);
        foreach ($fileTemplate as $line_num => $line) {
            // dd($line);
            $stringFormat .= $line;
            if($line_num == $baris-3){
                $stringFormat .= $entry;
            }
        }
        // dd($stringFormat);
        $file = fopen("arsip_surat/" . $noSurat. "_" . $npm . "_surat_pengantar_studi_lapangan_3orang.tex", "w");
        fwrite($file, $stringFormat);
        fclose($file);
        shell_exec('pdflatex -output-directory arsip_surat arsip_surat/' . $noSurat . '_' . $npm . '_surat_pengantar_studi_lapangan_3orang.tex');

        //store to db
        $historysurat = new Historysurat;
        $historysurat->no_surat = $noSurat;
        $historysurat->perihal = 'Permohonan' . $keperluanKunjungan;
        $historysurat->penerimaSurat = $kepada;
        $historysurat->mahasiswa_id = $pemesan;
        $historysurat->formatsurats_id = $request->idFormatSurat;
        $historysurat->link_arsip_surat = '127.0.0.1:8000/arsip_surat/' . $noSurat. '_' . $npm . '_surat_pengantar_studi_lapangan_3orang.pdf';
        $historysurat->penandatanganan = false;
        $historysurat->pengambilan = false;
        $historysurat->save();
        return redirect('/history_TU');
      }
      else if($request->idFormatSurat == "7"){
        $dataSurat = $request->data;
        $json = json_decode($dataSurat);
        $noSurat = $request->noSurat;
        $nama = $json->nama;
        $npm = $json->npm;
        $prodi = $json->prodi;
        $matkul = $json->matkul;
        $topik = $json->topik;
        $organisasi = $json->organisasi;
        $alamatOrganisasi = $json->alamatOrganisasi;
        $keperluanKunjungan = $json->keperluanKunjungan;
        $kepada = $json->kepada;
        $kota = $json->kota;
        $namaAnggota1 = $json->namaAnggota1;
        $npmAnggota1 = $json->npmAnggota1;
        $namaAnggota2 = $json->namaAnggota2;
        $npmAnggota2 = $json->npmAnggota2;
        $namaAnggota3 = $json->namaAnggota3;
        $npmAnggota3 = $json->npmAnggota3;
        $pemesan = $request->pemesan;

        $entry = '\mailentry{' .
          $noSurat . ',' .
          $nama . ',' .
          $npm . ',' .
          $prodi . ',' .
          $matkul . ',' .
          $topik . ',' .
          $organisasi . ',' .
          $alamatOrganisasi . ',' .
          $keperluanKunjungan . ',' .
          $kota . ',' .
          $kepada . ',' .
          $namaAnggota1 . ',' .
          $npmAnggota1 . ',' .
          $namaAnggota2 . ',' .
          $npmAnggota2 . ',' .
          $namaAnggota3 . ',' .
          $npmAnggota3 . ',' .
          '}';
        $fileTemplate = file('format_surat_latex/surat_pengantar_studi_lapangan_4orang.tex');
        $stringFormat = "";
        $baris = count($fileTemplate);
        // dd($baris);
        foreach ($fileTemplate as $line_num => $line) {
            // dd($line);
            $stringFormat .= $line;
            if($line_num == $baris-3){
                $stringFormat .= $entry;
            }
        }
        // dd($stringFormat);
        $file = fopen("arsip_surat/" . $noSurat. "_" . $npm . "_surat_pengantar_studi_lapangan_4orang.tex", "w");
        fwrite($file, $stringFormat);
        fclose($file);
        shell_exec('pdflatex -output-directory arsip_surat arsip_surat/' . $noSurat . '_' . $npm . '_surat_pengantar_studi_lapangan_4orang.tex');

        //store to db
        $historysurat = new Historysurat;
        $historysurat->no_surat = $noSurat;
        $historysurat->perihal = 'Permohonan' . $keperluanKunjungan;
        $historysurat->penerimaSurat = $kepada;
        $historysurat->mahasiswa_id = $pemesan;
        $historysurat->formatsurats_id = $request->idFormatSurat;
        $historysurat->link_arsip_surat = '127.0.0.1:8000/arsip_surat/' . $noSurat. '_' . $npm . '_surat_pengantar_studi_lapangan_4orang.pdf';
        $historysurat->penandatanganan = false;
        $historysurat->pengambilan = false;
        $historysurat->save();
        return redirect('/history_TU');
      }
      else if($request->idFormatSurat == "8"){
        $dataSurat = $request->data;
        $json = json_decode($dataSurat);
        $noSurat = $request->noSurat;
        $nama = $json->nama;
        $npm = $json->npm;
        $prodi = $json->prodi;
        $matkul = $json->matkul;
        $topik = $json->topik;
        $organisasi = $json->organisasi;
        $alamatOrganisasi = $json->alamatOrganisasi;
        $keperluanKunjungan = $json->keperluanKunjungan;
        $kepada = $json->kepada;
        $kota = $json->kota;
        $namaAnggota1 = $json->namaAnggota1;
        $npmAnggota1 = $json->npmAnggota1;
        $namaAnggota2 = $json->namaAnggota2;
        $npmAnggota2 = $json->npmAnggota2;
        $namaAnggota3 = $json->namaAnggota3;
        $npmAnggota3 = $json->npmAnggota3;
        $namaAnggota4 = $json->namaAnggota4;
        $npmAnggota4 = $json->npmAnggota4;
        $pemesan = $request->pemesan;

        $entry = '\mailentry{' .
          $noSurat . ',' .
          $nama . ',' .
          $npm . ',' .
          $prodi . ',' .
          $matkul . ',' .
          $topik . ',' .
          $organisasi . ',' .
          $alamatOrganisasi . ',' .
          $keperluanKunjungan . ',' .
          $kota . ',' .
          $kepada . ',' .
          $namaAnggota1 . ',' .
          $npmAnggota1 . ',' .
          $namaAnggota2 . ',' .
          $npmAnggota2 . ',' .
          $namaAnggota3 . ',' .
          $npmAnggota3 . ',' .
          $namaAnggota4 . ',' .
          $npmAnggota4 . ',' .
          '}';
        $fileTemplate = file('format_surat_latex/surat_pengantar_studi_lapangan_5orang.tex');
        $stringFormat = "";
        $baris = count($fileTemplate);
        // dd($baris);
        foreach ($fileTemplate as $line_num => $line) {
            // dd($line);
            $stringFormat .= $line;
            if($line_num == $baris-3){
                $stringFormat .= $entry;
            }
        }
        // dd($stringFormat);
        $file = fopen("arsip_surat/" . $noSurat. "_" . $npm . "_surat_pengantar_studi_lapangan_5orang.tex", "w");
        fwrite($file, $stringFormat);
        fclose($file);
        shell_exec('pdflatex -output-directory arsip_surat arsip_surat/' . $noSurat . '_' . $npm . '_surat_pengantar_studi_lapangan_5orang.tex');

        //store to db
        $historysurat = new Historysurat;
        $historysurat->no_surat = $noSurat;
        $historysurat->perihal = 'Permohonan' . $keperluanKunjungan;
        $historysurat->penerimaSurat = $kepada;
        $historysurat->mahasiswa_id = $pemesan;
        $historysurat->formatsurats_id = $request->idFormatSurat;
        $historysurat->link_arsip_surat = '127.0.0.1:8000/arsip_surat/' . $noSurat. '_' . $npm . '_surat_pengantar_studi_lapangan_5orang.pdf';
        $historysurat->penandatanganan = false;
        $historysurat->pengambilan = false;
        $historysurat->save();
        return redirect('/history_TU');
      }
      else if($request->idFormatSurat == "9"){
        $dataSurat = $request->data;
        $json = json_decode($dataSurat);
        $noSurat = $request->noSurat;
        $nama = $json->nama;
        $npm = $json->npm;
        $prodi = $json->prodi;
        $fakultas = $json->fakultas;
        $alamat = $json->alamat;
        $cutiStudiKe = $json->cutiStudiKe;
        $alasanCutiStudi = $json->alasanCutiStudi;
        $dosenWali = $json->dosenWali;
        $semester = $json->semester;
        $thnAkademik = $json->thnAkademik;
        $pemesan = $request->pemesan;

        $entry = '\mailentry{' . $noSurat . ',' . $nama . ',' . $npm . ',' . $prodi . ',' . $fakultas . ',' . $alamat . ',' . $cutiStudiKe . ',' . $alasanCutiStudi . ',' . $dosenWali . ',' . $semester . ',' . $thnAkademik . '}';
        $fileTemplate = file('format_surat_latex/surat_izin_cuti_studi.tex');
        $stringFormat = "";
        $baris = count($fileTemplate);
        // dd($baris);
        foreach ($fileTemplate as $line_num => $line) {
            // dd($line);
            $stringFormat .= $line;
            if($line_num == $baris-3){
                $stringFormat .= $entry;
            }
        }
        // dd($stringFormat);
        $file = fopen("arsip_surat/" . $noSurat. "_" . $npm . "_surat_izin_cuti_studi.tex", "w");
        fwrite($file, $stringFormat);
        fclose($file);
        shell_exec('pdflatex -output-directory arsip_surat arsip_surat/' . $noSurat . '_' . $npm . '_surat_izin_cuti_studi.tex');

        //store to db
        $historysurat = new Historysurat;
        $historysurat->no_surat = $noSurat;
        $historysurat->perihal = 'Surat Ijin Berhenti Studi Sementara';
        $historysurat->penerimaSurat = $nama;
        $historysurat->mahasiswa_id = $pemesan;
        $historysurat->formatsurats_id = $request->idFormatSurat;
        $historysurat->link_arsip_surat = '127.0.0.1:8000/arsip_surat/' . $noSurat. '_' . $nama . '_surat_izin_cuti_studi.pdf';
        $historysurat->penandatanganan = false;
        $historysurat->pengambilan = false;
        $historysurat->save();
        return redirect('/history_TU');
      }
      else if($request->idFormatSurat == "10"){
        $dataSurat = $request->data;
        $json = json_decode($dataSurat);
        $noSurat = $request->noSurat;
        $nama = $json->nama;
        $npm = $json->npm;
        $alamat = $json->alamat;
        $noTelepon = $json->noTelepon;
        $namaOrtu = $json->namaOrtu;
        $dosenWali = $json->dosenWali;
        $semester = $json->semester;
        $pemesan = $request->pemesan;

        $entry = '\mailentry{' . $noSurat . ',' . $nama . ',' . $npm . ',' . $alamat . ',' . $noTelepon . ',' . $namaOrtu . ',' . $dosenWali . ',' . $semester . ',' . '}';
        $fileTemplate = file('format_surat_latex/surat_pengunduran_diri.tex');
        $stringFormat = "";
        $baris = count($fileTemplate);
        // dd($baris);
        foreach ($fileTemplate as $line_num => $line) {
            // dd($line);
            $stringFormat .= $line;
            if($line_num == $baris-3){
                $stringFormat .= $entry;
            }
        }
        // dd($stringFormat);
        $file = fopen("arsip_surat/" . $noSurat. "_" . $npm . "_surat_pengunduran_diri.tex", "w");
        fwrite($file, $stringFormat);
        fclose($file);
        shell_exec('pdflatex -output-directory arsip_surat arsip_surat/' . $noSurat . '_' . $npm . '_surat_pengunduran_diri.tex');

        //store to db
        $historysurat = new Historysurat;
        $historysurat->no_surat = $noSurat;
        $historysurat->perihal = 'Pengunduran diri';
        $historysurat->penerimaSurat = 'Rektor';
        $historysurat->mahasiswa_id = $pemesan;
        $historysurat->formatsurats_id = $request->idFormatSurat;
        $historysurat->link_arsip_surat = '127.0.0.1:8000/arsip_surat/' . $noSurat. '_' . $npm . '_surat_pengunduran_diri.pdf';
        $historysurat->penandatanganan = false;
        $historysurat->pengambilan = false;
        $historysurat->save();
        return redirect('/history_TU');
      }
      else if($request->idFormatSurat == "11"){
        $dataSurat = $request->data;
        $json = json_decode($dataSurat);
        $noSurat = $request->noSurat;
        $nama = $json->nama;
        $prodi = $json->prodi;
        $npm = $json->npm;
        $namaWakil = $json->namaWakil;
        $prodiWakil = $json->prodiWakil;
        $npmWakil = $json->npmWakil;
        $dosenWali = $json->dosenWali;
        $alasan = $json->alasan;
        $kodeMK = $json->kodeMK;
        $matkul = $json->matkul;
        $sks = $json->sks;
        $pemesan = $request->pemesan;

        $entry = '\mailentry{' .
          $noSurat . ',' .
          $nama . ',' .
          $prodi . ',' .
          $npm . ',' .
          $namaWakil . ',' .
          $prodiWakil . ',' .
          $npmWakil . ',' .
          $dosenWali . ',' .
          $alasan . ',' .
          $kodeMK . ',' .
          $matkul . ',' .
          $sks . ',' .
          '}';
        $fileTemplate = file('format_surat_latex/surat_perwakilan_perwalian_1mk.tex');
        $stringFormat = "";
        $baris = count($fileTemplate);
        // dd($baris);
        foreach ($fileTemplate as $line_num => $line) {
            // dd($line);
            $stringFormat .= $line;
            if($line_num == $baris-3){
                $stringFormat .= $entry;
            }
        }
        // dd($stringFormat);
        $file = fopen("arsip_surat/" . $noSurat. "_" . $npm . "_surat_perwakilan_perwalian_1mk.tex", "w");
        fwrite($file, $stringFormat);
        fclose($file);
        shell_exec('pdflatex -output-directory arsip_surat arsip_surat/' . $noSurat . '_' . $npm . '_surat_perwakilan_perwalian_1mk.tex');

        //store to db
        $historysurat = new Historysurat;
        $historysurat->no_surat = $noSurat;
        $historysurat->perihal = '-';
        $historysurat->penerimaSurat = '-';
        $historysurat->mahasiswa_id = $pemesan;
        $historysurat->formatsurats_id = $request->idFormatSurat;
        $historysurat->link_arsip_surat = '127.0.0.1:8000/arsip_surat/' . $noSurat. '_' . $npm . '_surat_perwakilan_perwalian_1mk.pdf';
        $historysurat->penandatanganan = false;
        $historysurat->pengambilan = false;
        $historysurat->save();
        return redirect('/history_TU');
      }
      else if($request->idFormatSurat == "12"){
        $dataSurat = $request->data;
        $json = json_decode($dataSurat);
        $noSurat = $request->noSurat;
        $nama = $json->nama;
        $prodi = $json->prodi;
        $npm = $json->npm;
        $namaWakil = $json->namaWakil;
        $prodiWakil = $json->prodiWakil;
        $npmWakil = $json->npmWakil;
        $dosenWali = $json->dosenWali;
        $alasan = $json->alasan;
        $matkul1 = $json->matkul1;
        $sks1 = $json->sks1;
        $matkul2 = $json->matkul2;
        $sks2 = $json->sks2;
        $pemesan = $request->pemesan;

        $entry = '\mailentry{' .
          $noSurat . ',' .
          $nama . ',' .
          $prodi . ',' .
          $npm . ',' .
          $namaWakil . ',' .
          $prodiWakil . ',' .
          $npmWakil . ',' .
          $dosenWali . ',' .
          $alasan . ',' .
          $kodeMK1 = $json->kodeMK1;
          $matkul1 = $json->matkul1;
          $sks1 = $json->sks1;
          $kodeMK2 = $json->kodeMK2;
          $matkul2 = $json->matkul2;
          $sks2 = $json->sks2;
          $pemesan = $request->pemesan;

          $entry = '\mailentry{' .
            $noSurat . ',' .
            $nama . ',' .
            $prodi . ',' .
            $npm . ',' .
            $namaWakil . ',' .
            $prodiWakil . ',' .
            $npmWakil . ',' .
            $dosenWali . ',' .
            $alasan . ',' .
            $kodeMK1 . ',' .
            $matkul1 . ',' .
            $sks1 . ',' .
            $kodeMK2 . ',' .
            $matkul2 . ',' .
            $sks2 . ',' .
          '}';
        $fileTemplate = file('format_surat_latex/surat_perwakilan_perwalian_2mk.tex');
        $stringFormat = "";
        $baris = count($fileTemplate);
        // dd($baris);
        foreach ($fileTemplate as $line_num => $line) {
            // dd($line);
            $stringFormat .= $line;
            if($line_num == $baris-3){
                $stringFormat .= $entry;
            }
        }
        // dd($stringFormat);
        $file = fopen("arsip_surat/" . $noSurat. "_" . $npm . "_surat_perwakilan_perwalian_2mk.tex", "w");
        fwrite($file, $stringFormat);
        fclose($file);
        shell_exec('pdflatex -output-directory arsip_surat arsip_surat/' . $noSurat . '_' . $npm . '_surat_perwakilan_perwalian_2mk.tex');

        //store to db
        $historysurat = new Historysurat;
        $historysurat->no_surat = $noSurat;
        $historysurat->perihal = '-';
        $historysurat->penerimaSurat = '-';
        $historysurat->mahasiswa_id = $pemesan;
        $historysurat->formatsurats_id = $request->idFormatSurat;
        $historysurat->link_arsip_surat = '127.0.0.1:8000/arsip_surat/' . $noSurat. '_' . $npm . '_surat_perwakilan_perwalian_2mk.pdf';
        $historysurat->penandatanganan = false;
        $historysurat->pengambilan = false;
        $historysurat->save();
        return redirect('/history_TU');
      }
      else if($request->idFormatSurat == "13"){
        $dataSurat = $request->data;
        $json = json_decode($dataSurat);
        $noSurat = $request->noSurat;
        $semester = $json->semester;
        $thnAkademik = $json->thnAkademik;
        $nama = $json->nama;
        $prodi = $json->prodi;
        $npm = $json->npm;
        $namaWakil = $json->namaWakil;
        $prodiWakil = $json->prodiWakil;
        $npmWakil = $json->npmWakil;
        $dosenWali = $json->dosenWali;
        $alasan = $json->alasan;
        $kodeMK1 = $json->kodeMK1;
        $matkul1 = $json->matkul1;
        $sks1 = $json->sks1;
        $kodeMK2 = $json->kodeMK2;
        $matkul2 = $json->matkul2;
        $sks2 = $json->sks2;
        $kodeMK3 = $json->kodeMK3;
        $matkul3 = $json->matkul3;
        $sks3 = $json->sks3;
        $pemesan = $request->pemesan;

        $entry = '\mailentry{' . $semester . ',' .$thnAkademik . ','. $nama . ',' . $prodi . ',' . $npm . ',' . $namaWakil . ',' . $prodiWakil . ',' . $npmWakil . ',' . $alasan . ',' . $kodeMK1 . ',' . $matkul1 . ',' . $sks1 . ',' . $kodeMK2 . ',' . $matkul2 . ',' . $sks2 . ',' . $kodeMK3 . ',' . $matkul3 . ','. $sks3 .'}';
        // dd($entry);
        $fileTemplate = file('format_surat_latex/surat_perwakilan_perwalian_3mk.tex');
        $stringFormat = "";
        $baris = count($fileTemplate);
        // dd($baris);
        foreach ($fileTemplate as $line_num => $line) {
            // dd($line);
            $stringFormat .= $line;
            if($line_num == $baris-3){
                $stringFormat .= $entry;
            }
        }
        // dd($stringFormat);
        $file = fopen("arsip_surat/" . $noSurat. "_" . $npm . "_surat_perwakilan_perwalian_3mk.tex", "w");
        fwrite($file, $stringFormat);
        fclose($file);
        shell_exec('pdflatex -output-directory arsip_surat arsip_surat/' . $noSurat . '_' . $npm . '_surat_perwakilan_perwalian_3mk.tex');

        //store to db
        $historysurat = new Historysurat;
        $historysurat->no_surat = $noSurat;
        $historysurat->perihal = '-';
        $historysurat->penerimaSurat = '-';
        $historysurat->mahasiswa_id = $pemesan;
        $historysurat->formatsurat_id = $request->idFormatSurat;
        $historysurat->link_arsip_surat = '127.0.0.1:8000/arsip_surat/' . $noSurat. '_' . $npm . '_surat_perwakilan_perwalian_3mk.pdf';
        $historysurat->penandatanganan = false;
        $historysurat->pengambilan = false;
        $historysurat->save();
        return redirect('/history_TU');
      }
      else if($request->idFormatSurat == "14"){
        $dataSurat = $request->data;
        $json = json_decode($dataSurat);
        $noSurat = $request->noSurat;
        $nama = $json->nama;
        $prodi = $json->prodi;
        $npm = $json->npm;
        $namaWakil = $json->namaWakil;
        $prodiWakil = $json->prodiWakil;
        $npmWakil = $json->npmWakil;
        $dosenWali = $json->dosenWali;
        $alasan = $json->alasan;
        $kodeMK1 = $json->kodeMK1;
        $matkul1 = $json->matkul1;
        $sks1 = $json->sks1;
        $kodeMK2 = $json->kodeMK2;
        $matkul2 = $json->matkul2;
        $sks2 = $json->sks2;
        $kodeMK3 = $json->kodeMK3;
        $matkul3 = $json->matkul3;
        $sks3 = $json->sks3;
        $kodeMK4 = $json->kodeMK4;
        $matkul4 = $json->matkul4;
        $sks4 = $json->sks4;
        $pemesan = $request->pemesan;

        $entry = '\mailentry{' .
          $noSurat . ',' .
          $nama . ',' .
          $prodi . ',' .
          $npm . ',' .
          $namaWakil . ',' .
          $prodiWakil . ',' .
          $npmWakil . ',' .
          $dosenWali . ',' .
          $alasan . ',' .
          $kodeMK1 . ',' .
          $matkul1 . ',' .
          $sks1 . ',' .
          $kodeMK2 . ',' .
          $matkul2 . ',' .
          $sks2 . ',' .
          $kodeMK3 . ',' .
          $matkul3 . ',' .
          $sks3 . ',' .
          $kodeMK4 . ',' .
          $matkul4 . ',' .
          $sks4 . ',' .
          '}';
        $fileTemplate = file('format_surat_latex/surat_perwakilan_perwalian_4mk.tex');
        $stringFormat = "";
        $baris = count($fileTemplate);
        // dd($baris);
        foreach ($fileTemplate as $line_num => $line) {
            // dd($line);
            $stringFormat .= $line;
            if($line_num == $baris-3){
                $stringFormat .= $entry;
            }
        }
        // dd($stringFormat);
        $file = fopen("arsip_surat/" . $noSurat. "_" . $npm . "_surat_perwakilan_perwalian_4mk.tex", "w");
        fwrite($file, $stringFormat);
        fclose($file);
        shell_exec('pdflatex -output-directory arsip_surat arsip_surat/' . $noSurat . '_' . $npm . '_surat_perwakilan_perwalian_4mk.tex');

        //store to db
        $historysurat = new Historysurat;
        $historysurat->no_surat = $noSurat;
        $historysurat->perihal = '-';
        $historysurat->penerimaSurat = '-';
        $historysurat->mahasiswa_id = $pemesan;
        $historysurat->formatsurats_id = $request->idFormatSurat;
        $historysurat->link_arsip_surat = '127.0.0.1:8000/arsip_surat/' . $noSurat. '_' . $npm . '_surat_perwakilan_perwalian_4mk.pdf';
        $historysurat->penandatanganan = false;
        $historysurat->pengambilan = false;
        $historysurat->save();
        return redirect('/history_TU');
      }
      else if($request->idFormatSurat == "15"){
        $dataSurat = $request->data;
        $json = json_decode($dataSurat);
        $noSurat = $request->noSurat;
        $nama = $json->nama;
        $prodi = $json->prodi;
        $npm = $json->npm;
        $namaWakil = $json->namaWakil;
        $prodiWakil = $json->prodiWakil;
        $npmWakil = $json->npmWakil;
        $dosenWali = $json->dosenWali;
        $alasan = $json->alasan;
        $kodeMK1 = $json->kodeMK1;
        $matkul1 = $json->matkul1;
        $sks1 = $json->sks1;
        $kodeMK2 = $json->kodeMK2;
        $matkul2 = $json->matkul2;
        $sks2 = $json->sks2;
        $kodeMK3 = $json->kodeMK3;
        $matkul3 = $json->matkul3;
        $sks3 = $json->sks3;
        $kodeMK4 = $json->kodeMK4;
        $matkul4 = $json->matkul4;
        $sks4 = $json->sks4;
        $kodeMK5 = $json->kodeMK5;
        $matkul5 = $json->matkul5;
        $sks5 = $json->sks5;
        $pemesan = $request->pemesan;

        $entry = '\mailentry{' .
          $noSurat . ',' .
          $nama . ',' .
          $prodi . ',' .
          $npm . ',' .
          $namaWakil . ',' .
          $prodiWakil . ',' .
          $npmWakil . ',' .
          $dosenWali . ',' .
          $alasan . ',' .
          $kodeMK1 . ',' .
          $matkul1 . ',' .
          $sks1 . ',' .
          $kodeMK2 . ',' .
          $matkul2 . ',' .
          $sks2 . ',' .
          $kodeMK3 . ',' .
          $matkul3 . ',' .
          $sks3 . ',' .
          $kodeMK4 . ',' .
          $matkul4 . ',' .
          $sks4 . ',' .
          $kodeMK5 . ',' .
          $matkul5 . ',' .
          $sks5 . ',' .
          '}';
        $fileTemplate = file('format_surat_latex/surat_perwakilan_perwalian_5mk.tex');
        $stringFormat = "";
        $baris = count($fileTemplate);
        // dd($baris);
        foreach ($fileTemplate as $line_num => $line) {
            // dd($line);
            $stringFormat .= $line;
            if($line_num == $baris-3){
                $stringFormat .= $entry;
            }
        }
        // dd($stringFormat);
        $file = fopen("arsip_surat/" . $noSurat. "_" . $npm . "_surat_perwakilan_perwalian_5mk.tex", "w");
        fwrite($file, $stringFormat);
        fclose($file);
        shell_exec('pdflatex -output-directory arsip_surat arsip_surat/' . $noSurat . '_' . $npm . '_surat_perwakilan_perwalian_5mk.tex');

        //store to db
        $historysurat = new Historysurat;
        $historysurat->no_surat = $noSurat;
        $historysurat->perihal = '-';
        $historysurat->penerimaSurat = '-';
        $historysurat->mahasiswa_id = $pemesan;
        $historysurat->formatsurats_id = $request->idFormatSurat;
        $historysurat->link_arsip_surat = '127.0.0.1:8000/arsip_surat/' . $noSurat. '_' . $npm . '_surat_perwakilan_perwalian_5mk.pdf';
        $historysurat->penandatanganan = false;
        $historysurat->pengambilan = false;
        $historysurat->save();
        return redirect('/history_TU');
      }
      else if($request->idFormatSurat == "16"){
        $dataSurat = $request->data;
        $json = json_decode($dataSurat);
        $noSurat = $request->noSurat;
        $nama = $json->nama;
        $prodi = $json->prodi;
        $npm = $json->npm;
        $namaWakil = $json->namaWakil;
        $prodiWakil = $json->prodiWakil;
        $npmWakil = $json->npmWakil;
        $dosenWali = $json->dosenWali;
        $alasan = $json->alasan;
        $kodeMK1 = $json->kodeMK1;
        $matkul1 = $json->matkul1;
        $sks1 = $json->sks1;
        $kodeMK2 = $json->kodeMK2;
        $matkul2 = $json->matkul2;
        $sks2 = $json->sks2;
        $kodeMK3 = $json->kodeMK3;
        $matkul3 = $json->matkul3;
        $sks3 = $json->sks3;
        $kodeMK4 = $json->kodeMK4;
        $matkul4 = $json->matkul4;
        $sks4 = $json->sks4;
        $kodeMK5 = $json->kodeMK5;
        $matkul5 = $json->matkul5;
        $sks5 = $json->sks5;
        $kodeMK6 = $json->kodeMK6;
        $matkul6 = $json->matkul6;
        $sks6 = $json->sks6;
        $pemesan = $request->pemesan;

        $entry = '\mailentry{' .
          $noSurat . ',' .
          $nama . ',' .
          $prodi . ',' .
          $npm . ',' .
          $namaWakil . ',' .
          $prodiWakil . ',' .
          $npmWakil . ',' .
          $dosenWali . ',' .
          $alasan . ',' .
          $kodeMK1 . ',' .
          $matkul1 . ',' .
          $sks1 . ',' .
          $kodeMK2 . ',' .
          $matkul2 . ',' .
          $sks2 . ',' .
          $kodeMK3 . ',' .
          $matkul3 . ',' .
          $sks3 . ',' .
          $kodeMK4 . ',' .
          $matkul4 . ',' .
          $sks4 . ',' .
          $kodeMK5 . ',' .
          $matkul5 . ',' .
          $sks5 . ',' .
          $kodeMK6 . ',' .
          $matkul6 . ',' .
          $sks6 . ',' .
          '}';
        $fileTemplate = file('format_surat_latex/surat_perwakilan_perwalian_6mk.tex');
        $stringFormat = "";
        $baris = count($fileTemplate);
        // dd($baris);
        foreach ($fileTemplate as $line_num => $line) {
            // dd($line);
            $stringFormat .= $line;
            if($line_num == $baris-3){
                $stringFormat .= $entry;
            }
        }
        // dd($stringFormat);
        $file = fopen("arsip_surat/" . $noSurat. "_" . $npm . "_surat_perwakilan_perwalian_6mk.tex", "w");
        fwrite($file, $stringFormat);
        fclose($file);
        shell_exec('pdflatex -output-directory arsip_surat arsip_surat/' . $noSurat . '_' . $npm . '_surat_perwakilan_perwalian_6mk.tex');

        //store to db
        $historysurat = new Historysurat;
        $historysurat->no_surat = $noSurat;
        $historysurat->perihal = '-';
        $historysurat->penerimaSurat = '-';
        $historysurat->mahasiswa_id = $pemesan;
        $historysurat->formatsurats_id = $request->idFormatSurat;
        $historysurat->link_arsip_surat = '127.0.0.1:8000/arsip_surat/' . $noSurat. '_' . $npm . '_surat_perwakilan_perwalian_6mk.pdf';
        $historysurat->penandatanganan = false;
        $historysurat->pengambilan = false;
        $historysurat->save();
        return redirect('/history_TU');
      }
      else if($request->idFormatSurat == "17"){
        $dataSurat = $request->data;
        $json = json_decode($dataSurat);
        $noSurat = $request->noSurat;
        $nama = $json->nama;
        $prodi = $json->prodi;
        $npm = $json->npm;
        $namaWakil = $json->namaWakil;
        $prodiWakil = $json->prodiWakil;
        $npmWakil = $json->npmWakil;
        $dosenWali = $json->dosenWali;
        $alasan = $json->alasan;
        $kodeMK1 = $json->kodeMK1;
        $matkul1 = $json->matkul1;
        $sks1 = $json->sks1;
        $kodeMK2 = $json->kodeMK2;
        $matkul2 = $json->matkul2;
        $sks2 = $json->sks2;
        $kodeMK3 = $json->kodeMK3;
        $matkul3 = $json->matkul3;
        $sks3 = $json->sks3;
        $kodeMK4 = $json->kodeMK4;
        $matkul4 = $json->matkul4;
        $sks4 = $json->sks4;
        $kodeMK5 = $json->kodeMK5;
        $matkul5 = $json->matkul5;
        $sks5 = $json->sks5;
        $kodeMK6 = $json->kodeMK6;
        $matkul6 = $json->matkul6;
        $sks6 = $json->sks6;
        $kodeMK7 = $json->kodeMK7;
        $matkul7 = $json->matkul7;
        $sks7 = $json->sks7;
        $pemesan = $request->pemesan;

        $entry = '\mailentry{' .
          $noSurat . ',' .
          $nama . ',' .
          $prodi . ',' .
          $npm . ',' .
          $namaWakil . ',' .
          $prodiWakil . ',' .
          $npmWakil . ',' .
          $dosenWali . ',' .
          $alasan . ',' .
          $kodeMK1 . ',' .
          $matkul1 . ',' .
          $sks1 . ',' .
          $kodeMK2 . ',' .
          $matkul2 . ',' .
          $sks2 . ',' .
          $kodeMK3 . ',' .
          $matkul3 . ',' .
          $sks3 . ',' .
          $kodeMK4 . ',' .
          $matkul4 . ',' .
          $sks4 . ',' .
          $kodeMK5 . ',' .
          $matkul5 . ',' .
          $sks5 . ',' .
          $kodeMK6 . ',' .
          $matkul6 . ',' .
          $sks6 . ',' .
          $kodeMK7 . ',' .
          $matkul7 . ',' .
          $sks7 . ',' .
          '}';
        $fileTemplate = file('format_surat_latex/surat_perwakilan_perwalian_7mk.tex');
        $stringFormat = "";
        $baris = count($fileTemplate);
        // dd($baris);
        foreach ($fileTemplate as $line_num => $line) {
            // dd($line);
            $stringFormat .= $line;
            if($line_num == $baris-3){
                $stringFormat .= $entry;
            }
        }
        // dd($stringFormat);
        $file = fopen("arsip_surat/" . $noSurat. "_" . $npm . "_surat_perwakilan_perwalian_7mk.tex", "w");
        fwrite($file, $stringFormat);
        fclose($file);
        shell_exec('pdflatex -output-directory arsip_surat arsip_surat/' . $noSurat . '_' . $npm . '_surat_perwakilan_perwalian_7mk.tex');

        //store to db
        $historysurat = new Historysurat;
        $historysurat->no_surat = $noSurat;
        $historysurat->perihal = '-';
        $historysurat->penerimaSurat = '-';
        $historysurat->mahasiswa_id = $pemesan;
        $historysurat->formatsurats_id = $request->idFormatSurat;
        $historysurat->link_arsip_surat = '127.0.0.1:8000/arsip_surat/' . $noSurat. '_' . $npm . '_surat_perwakilan_perwalian_7mk.pdf';
        $historysurat->penandatanganan = false;
        $historysurat->pengambilan = false;
        $historysurat->save();
        return redirect('/history_TU');
      }
      else if($request->idFormatSurat == "18"){
        $dataSurat = $request->data;
        $json = json_decode($dataSurat);
        $noSurat = $request->noSurat;
        $nama = $json->nama;
        $prodi = $json->prodi;
        $npm = $json->npm;
        $namaWakil = $json->namaWakil;
        $prodiWakil = $json->prodiWakil;
        $npmWakil = $json->npmWakil;
        $dosenWali = $json->dosenWali;
        $alasan = $json->alasan;
        $kodeMK1 = $json->kodeMK1;
        $matkul1 = $json->matkul1;
        $sks1 = $json->sks1;
        $kodeMK2 = $json->kodeMK2;
        $matkul2 = $json->matkul2;
        $sks2 = $json->sks2;
        $kodeMK3 = $json->kodeMK3;
        $matkul3 = $json->matkul3;
        $sks3 = $json->sks3;
        $kodeMK4 = $json->kodeMK4;
        $matkul4 = $json->matkul4;
        $sks4 = $json->sks4;
        $kodeMK5 = $json->kodeMK5;
        $matkul5 = $json->matkul5;
        $sks5 = $json->sks5;
        $kodeMK6 = $json->kodeMK6;
        $matkul6 = $json->matkul6;
        $sks6 = $json->sks6;
        $kodeMK7 = $json->kodeMK7;
        $matkul7 = $json->matkul7;
        $sks7 = $json->sks7;
        $kodeMK8 = $json->kodeMK8;
        $matkul8 = $json->matkul8;
        $sks8 = $json->sks8;
        $pemesan = $request->pemesan;

        $entry = '\mailentry{' .
          $noSurat . ',' .
          $nama . ',' .
          $prodi . ',' .
          $npm . ',' .
          $namaWakil . ',' .
          $prodiWakil . ',' .
          $npmWakil . ',' .
          $dosenWali . ',' .
          $alasan . ',' .
          $kodeMK1 . ',' .
          $matkul1 . ',' .
          $sks1 . ',' .
          $kodeMK2 . ',' .
          $matkul2 . ',' .
          $sks2 . ',' .
          $kodeMK3 . ',' .
          $matkul3 . ',' .
          $sks3 . ',' .
          $kodeMK4 . ',' .
          $matkul4 . ',' .
          $sks4 . ',' .
          $kodeMK5 . ',' .
          $matkul5 . ',' .
          $sks5 . ',' .
          $kodeMK6 . ',' .
          $matkul6 . ',' .
          $sks6 . ',' .
          $kodeMK7 . ',' .
          $matkul7 . ',' .
          $sks7 . ',' .
          $kodeMK8 . ',' .
          $matkul8 . ',' .
          $sks8 . ',' .
          '}';
        $fileTemplate = file('format_surat_latex/surat_perwakilan_perwalian_8mk.tex');
        $stringFormat = "";
        $baris = count($fileTemplate);
        // dd($baris);
        foreach ($fileTemplate as $line_num => $line) {
            // dd($line);
            $stringFormat .= $line;
            if($line_num == $baris-3){
                $stringFormat .= $entry;
            }
        }
        // dd($stringFormat);
        $file = fopen("arsip_surat/" . $noSurat. "_" . $npm . "_surat_perwakilan_perwalian_8mk.tex", "w");
        fwrite($file, $stringFormat);
        fclose($file);
        shell_exec('pdflatex -output-directory arsip_surat arsip_surat/' . $noSurat . '_' . $npm . '_surat_perwakilan_perwalian_8mk.tex');

        //store to db
        $historysurat = new Historysurat;
        $historysurat->no_surat = $noSurat;
        $historysurat->perihal = '-';
        $historysurat->penerimaSurat = '-';
        $historysurat->mahasiswa_id = $pemesan;
        $historysurat->formatsurats_id = $request->idFormatSurat;
        $historysurat->link_arsip_surat = '127.0.0.1:8000/arsip_surat/' . $noSurat. '_' . $npm . '_surat_perwakilan_perwalian_8mk.pdf';
        $historysurat->penandatanganan = false;
        $historysurat->pengambilan = false;
        $historysurat->save();
        return redirect('/history_TU');
      }
      else if($request->idFormatSurat == "19"){
        $dataSurat = $request->data;
        $json = json_decode($dataSurat);
        $noSurat = $request->noSurat;
        $nama = $json->nama;
        $prodi = $json->prodi;
        $npm = $json->npm;
        $namaWakil = $json->namaWakil;
        $prodiWakil = $json->prodiWakil;
        $npmWakil = $json->npmWakil;
        $dosenWali = $json->dosenWali;
        $alasan = $json->alasan;
        $kodeMK1 = $json->kodeMK1;
        $matkul1 = $json->matkul1;
        $sks1 = $json->sks1;
        $kodeMK2 = $json->kodeMK2;
        $matkul2 = $json->matkul2;
        $sks2 = $json->sks2;
        $kodeMK3 = $json->kodeMK3;
        $matkul3 = $json->matkul3;
        $sks3 = $json->sks3;
        $kodeMK4 = $json->kodeMK4;
        $matkul4 = $json->matkul4;
        $sks4 = $json->sks4;
        $kodeMK5 = $json->kodeMK5;
        $matkul5 = $json->matkul5;
        $sks5 = $json->sks5;
        $kodeMK6 = $json->kodeMK6;
        $matkul6 = $json->matkul6;
        $sks6 = $json->sks6;
        $kodeMK7 = $json->kodeMK7;
        $matkul7 = $json->matkul7;
        $sks7 = $json->sks7;
        $kodeMK8 = $json->kodeMK8;
        $matkul8 = $json->matkul8;
        $sks8 = $json->sks8;
        $kodeMK9 = $json->kodeMK9;
        $matkul9 = $json->matkul9;
        $sks9 = $json->sks9;
        $pemesan = $request->pemesan;

        $entry = '\mailentry{' .
          $noSurat . ',' .
          $nama . ',' .
          $prodi . ',' .
          $npm . ',' .
          $namaWakil . ',' .
          $prodiWakil . ',' .
          $npmWakil . ',' .
          $dosenWali . ',' .
          $alasan . ',' .
          $kodeMK1 . ',' .
          $matkul1 . ',' .
          $sks1 . ',' .
          $kodeMK2 . ',' .
          $matkul2 . ',' .
          $sks2 . ',' .
          $kodeMK3 . ',' .
          $matkul3 . ',' .
          $sks3 . ',' .
          $kodeMK4 . ',' .
          $matkul4 . ',' .
          $sks4 . ',' .
          $kodeMK5 . ',' .
          $matkul5 . ',' .
          $sks5 . ',' .
          $kodeMK6 . ',' .
          $matkul6 . ',' .
          $sks6 . ',' .
          $kodeMK7 . ',' .
          $matkul7 . ',' .
          $sks7 . ',' .
          $kodeMK8 . ',' .
          $matkul8 . ',' .
          $sks8 . ',' .
          $kodeMK9 . ',' .
          $matkul9 . ',' .
          $sks9 . ',' .
          '}';
        $fileTemplate = file('format_surat_latex/surat_perwakilan_perwalian_9mk.tex');
        $stringFormat = "";
        $baris = count($fileTemplate);
        // dd($baris);
        foreach ($fileTemplate as $line_num => $line) {
            // dd($line);
            $stringFormat .= $line;
            if($line_num == $baris-3){
                $stringFormat .= $entry;
            }
        }
        // dd($stringFormat);
        $file = fopen("arsip_surat/" . $noSurat. "_" . $npm . "_surat_perwakilan_perwalian_9mk.tex", "w");
        fwrite($file, $stringFormat);
        fclose($file);
        shell_exec('pdflatex -output-directory arsip_surat arsip_surat/' . $noSurat . '_' . $npm . '_surat_perwakilan_perwalian_9mk.tex');

        //store to db
        $historysurat = new Historysurat;
        $historysurat->no_surat = $noSurat;
        $historysurat->perihal = '-';
        $historysurat->penerimaSurat = '-';
        $historysurat->mahasiswa_id = $pemesan;
        $historysurat->formatsurats_id = $request->idFormatSurat;
        $historysurat->link_arsip_surat = '127.0.0.1:8000/arsip_surat/' . $noSurat. '_' . $npm . '_surat_perwakilan_perwalian_9mk.pdf';
        $historysurat->penandatanganan = false;
        $historysurat->pengambilan = false;
        $historysurat->save();
        return redirect('/history_TU');
      }
      else if($request->idFormatSurat == "20"){
        $dataSurat = $request->data;
        $json = json_decode($dataSurat);
        $noSurat = $request->noSurat;
        $nama = $json->nama;
        $prodi = $json->prodi;
        $npm = $json->npm;
        $namaWakil = $json->namaWakil;
        $prodiWakil = $json->prodiWakil;
        $npmWakil = $json->npmWakil;
        $dosenWali = $json->dosenWali;
        $alasan = $json->alasan;
        $kodeMK1 = $json->kodeMK1;
        $matkul1 = $json->matkul1;
        $sks1 = $json->sks1;
        $kodeMK2 = $json->kodeMK2;
        $matkul2 = $json->matkul2;
        $sks2 = $json->sks2;
        $kodeMK3 = $json->kodeMK3;
        $matkul3 = $json->matkul3;
        $sks3 = $json->sks3;
        $kodeMK4 = $json->kodeMK4;
        $matkul4 = $json->matkul4;
        $sks4 = $json->sks4;
        $kodeMK5 = $json->kodeMK5;
        $matkul5 = $json->matkul5;
        $sks5 = $json->sks5;
        $kodeMK6 = $json->kodeMK6;
        $matkul6 = $json->matkul6;
        $sks6 = $json->sks6;
        $kodeMK7 = $json->kodeMK7;
        $matkul7 = $json->matkul7;
        $sks7 = $json->sks7;
        $kodeMK8 = $json->kodeMK8;
        $matkul8 = $json->matkul8;
        $sks8 = $json->sks8;
        $kodeMK9 = $json->kodeMK9;
        $matkul9 = $json->matkul9;
        $sks9 = $json->sks9;
        $kodeMK10 = $json->kodeMK10;
        $matkul10 = $json->matkul10;
        $sks10 = $json->sks10;
        $pemesan = $request->pemesan;

        $entry = '\mailentry{' .
          $noSurat . ',' .
          $nama . ',' .
          $prodi . ',' .
          $npm . ',' .
          $namaWakil . ',' .
          $prodiWakil . ',' .
          $npmWakil . ',' .
          $dosenWali . ',' .
          $alasan . ',' .
          $kodeMK1 . ',' .
          $matkul1 . ',' .
          $sks1 . ',' .
          $kodeMK2 . ',' .
          $matkul2 . ',' .
          $sks2 . ',' .
          $kodeMK3 . ',' .
          $matkul3 . ',' .
          $sks3 . ',' .
          $kodeMK4 . ',' .
          $matkul4 . ',' .
          $sks4 . ',' .
          $kodeMK5 . ',' .
          $matkul5 . ',' .
          $sks5 . ',' .
          $kodeMK6 . ',' .
          $matkul6 . ',' .
          $sks6 . ',' .
          $kodeMK7 . ',' .
          $matkul7 . ',' .
          $sks7 . ',' .
          $kodeMK8 . ',' .
          $matkul8 . ',' .
          $sks8 . ',' .
          $kodeMK9 . ',' .
          $matkul9 . ',' .
          $sks9 . ',' .
          $kodeMK10 . ',' .
          $matkul10 . ',' .
          $sks10 . ',' .
          '}';
        $fileTemplate = file('format_surat_latex/surat_perwakilan_perwalian_10mk.tex');
        $stringFormat = "";
        $baris = count($fileTemplate);
        // dd($baris);
        foreach ($fileTemplate as $line_num => $line) {
            // dd($line);
            $stringFormat .= $line;
            if($line_num == $baris-3){
                $stringFormat .= $entry;
            }
        }
        // dd($stringFormat);
        $file = fopen("arsip_surat/" . $noSurat. "_" . $npm . "_surat_perwakilan_perwalian_10mk.tex", "w");
        fwrite($file, $stringFormat);
        fclose($file);
        shell_exec('pdflatex -output-directory arsip_surat arsip_surat/' . $noSurat . '_' . $npm . '_surat_perwakilan_perwalian_10mk.tex');

        //store to db
        $historysurat = new Historysurat;
        $historysurat->no_surat = $noSurat;
        $historysurat->perihal = '-';
        $historysurat->penerimaSurat = '-';
        $historysurat->mahasiswa_id = $pemesan;
        $historysurat->formatsurats_id = $request->idFormatSurat;
        $historysurat->link_arsip_surat = '127.0.0.1:8000/arsip_surat/' . $noSurat. '_' . $npm . '_surat_perwakilan_perwalian_10mk.pdf';
        $historysurat->penandatanganan = false;
        $historysurat->pengambilan = false;
        $historysurat->save();
        return redirect('/history_TU');
      }
  }
}

\end{lstlisting}

\begin{lstlisting}[language=tex,basicstyle=\tiny,caption=MahasiswasuratController.php]
<?php

namespace App\Http\Controllers;

use Illuminate\Http\Request;

use App\Repositories\MahasiswaRepository;
use Illuminate\Support\Facades\Auth;
use App\Mahasiswa;
use App\User;
use App\Dosen;
use App\TU;
class MahasiswaController extends Controller
{
    //
    protected $mahasiswaRepo;

    public function __construct(MahasiswaRepository $mahasiswaRepo){
      // dd($formatsuratRepo);
        $this->mahasiswaRepo = $mahasiswaRepo;
    }

    public function tambahDataMahasiswa(){
      $loggedInUser = Auth::user();
      // dd($loggedInUser);
      $realUser = $this->getRealUser($loggedInUser);
      return view('TU.tambah_data_mahasiswa', [
        'user' => $realUser
      ]);
    }

    public function kategoriSurat(){
      $loggedInUser = Auth::user();
      // dd($loggedInUser);
      $realUser = $this->getRealUser($loggedInUser);
      return view('mahasiswa.pilih_kategori_surat', [
        'user' => $realUser
      ]);
    }

    public function setting(){
      $loggedInUser = Auth::user();
      // dd($loggedInUser);
      $realUser = $this->getRealUser($loggedInUser);
      return view('tu.setting_semester_thnAjaran', [
        'user' => $realUser
      ]);
    }

    public function updateSemester(Request $request){
      $mahasiswas = $this->mahasiswaRepo->findAllMhs();
      foreach ($mahasiswas as $mhs) {
        $mhs->semester = $request->semester;
        $mhs->thnAkademik = $request->thnAkademik;
        $mhs->save();
      }
    }

    public function tampilkanSeluruhSurat(Request $request){
      $loggedInUser = Auth::user();
      // dd($loggedInUser);
      $realUser = $this->getRealUser($loggedInUser);
      // dd($realUser);
      return view('mahasiswa.home_mahasiswa',[
        'user' => $realUser
      ]);
    }

    private function getRealUser($loggedInUser){
      // dd($loggedInUser);
      $realUser='';
      // dd($realUser);
      if($loggedInUser->jabatan == User::JABATAN_MHS){
        $realUser = Mahasiswa::find($loggedInUser->ref);
        // dd($realUser);
        return $realUser;
      }else if($loggedInUser->jabatan == User::JABATAN_DOS){
        $realUser = Dosen::find($loggedInUser->ref);
        // dd($realUser);
        return $realUser;
      }else{ // TU
        $realUser = TU::find($loggedInUser->ref);
        // dd($loggedInUser->jabatan);
        return $realUser;
      }
      // dd($realUser);
    }

    public function pilihMahasiswa(Request $request){
        //$confirmation = Confirmation::where(['id' => 2])->first();

        //dd($confirmation->order->tickets);
        //dd($confirmation);
        // //--
        $mahasiswas;
        if($request->kategori_mahasiswa == "nirm"){
          $mahasiswas = $this->mahasiswaRepo->findMahasiswaByNIRM($request->searchBox);
        }
        else if($request->kategori_mahasiswa == "npm"){
          $mahasiswas = $this->mahasiswaRepo->findMahasiswaByNPM($request->searchBox);
        }
        else if($request->kategori_mahasiswa == "nama_mahasiswa"){
          $mahasiswas = $this->mahasiswaRepo->findMahasiswaByNama($request->searchBox);
        }
        else if($request->kategori_mahasiswa == "prodi"){
          $mahasiswas = $this->mahasiswaRepo->findMahasiswaByProdi($request->searchBox);
        }
        else if($request->kategori_mahasiswa == "angkatan"){
          $mahasiswas = $this->mahasiswaRepo->findMahasiswaByAngkatan($request->searchBox);
        }
        else if($request->kategori_mahasiswa == "kota_lahir"){
          $mahasiswas = $this->mahasiswaRepo->findMahasiswaByKotaLahir($request->searchBox);
        }
        else if($request->kategori_mahasiswa == "tanggal_lahir"){
          $mahasiswas = $this->mahasiswaRepo->findMahasiswaByTanggalLahir($request->searchBox);
        }
        else if($request->kategori_mahasiswa == "fakultas"){
          $mahasiswas = $this->mahasiswaRepo->findMahasiswaByFakultas($request->searchBox);
        }
        else if($request->kategori_mahasiswa == "dosenWali"){
          $mahasiswas = $this->mahasiswaRepo->findMahasiswaByDosenWali($request->searchBox);
        }
        else{
          $mahasiswas = $this->mahasiswaRepo->findAllMahasiswa();
        }
        // dd($mahasiswas);
        $loggedInUser = Auth::user();
        $realUser = $this->getRealUser($loggedInUser);
        return view('TU.data_mahasiswa',[
        // dd($formatsurats);
            'mahasiswas' => $mahasiswas,
            'user' => $realUser
        ]);
  	}

    /**
    * Delete the selected data
    */
    public function destroy(Request $request){
        //
        // dd($request->deleteID);
        $mahasiswa = $this->getModel($request->deleteID);
        $mahasiswa->delete();
        return redirect('/data_mahasiswa')->with('success_message', 'Mahasiswa <b>#' . $request->id . '</b> berhasil dihapus.');;
    }

    /**
    * Get mahasiswa model by Id
    * @return Mahasiswa
    */
    private function getModel($id){
        $model = $this->mahasiswaRepo->findById($id);
        if($model === null){
            abort(404);
        }
        return $model;
    }

    public function uploadMahasiswa(Request $request){
      // dd($request);
      $dataMhs = $request->file('uploadDataMhs');
      $mhs = file($dataMhs);
      // dd($mhs);
      $baris = '';
      foreach ($mhs as $line_num => $line) {
        $baris .= $line;

        $data = explode(",", $baris);  //ubah pemisahnya dengan yg ada di sql
        // dd($data);

        $mahasiswa = new Mahasiswa;

        $mahasiswa->nirm = $data[0];
        $mahasiswa->npm = $data[1];
        $mahasiswa->nama_mahasiswa = $data[2];
        $mahasiswa->jurusan_id = $data[3];
        $mahasiswa->fakultas_id = $data[4];
        $mahasiswa->angkatan = $data[5];
        $mahasiswa->kota_lahir = $data[6];
        $mahasiswa->tanggal_lahir = $data[7];
        $mahasiswa->foto_mahasiswa = $data[8];
        $mahasiswa->dosen_id = $data[9];
        $mahasiswa->username = $data[10];
        $mahasiswa->save();

        $user = new User;
        $user->name = $data[2];
        $user->username = $data[10];
        $user->password = $data[11];
        $user->save();
      }
      return redirect('/data_mahasiswa')->with('success_message', 'Data mahasiswa telah di upload');
    }
}

\end{lstlisting}

\begin{lstlisting}[language=tex,basicstyle=\tiny,caption=PesanansuratController.php]
<?php
namespace App\Http\Controllers;
use Illuminate\Http\Request;
use App\Repositories\PesanansuratRepository;
use App\Repositories\FormatsuratRepository;
use App\Repositories\MahasiswaRepository;
use App\Pesanansurat;
use App\Formatsurat;
use Illuminate\Support\Facades\Auth;
use App\Mahasiswa;
use App\User;
use App\Dosen;
use App\TU;
class PesanansuratController extends Controller
{
    //
    protected $pesanansuratRepo;
    protected $formatsuratRepo;
    protected $mahasiswaRepo;
    public function __construct(PesanansuratRepository $pesanansuratRepo, FormatsuratRepository $formatsuratRepo, MahasiswaRepository $mahasiswaRepo){
      // dd($pesanansuratRepo);
        $this->pesanansuratRepo = $pesanansuratRepo;
        $this->formatsuratRepo = $formatsuratRepo;
        $this->mahasiswaRepo = $mahasiswaRepo;
        //dd($this->orders->getAllActive());
    }
    public function tampilkanPesananDiPejabat(Request $request){
      // $pesanansurats = $this->pesanansuratRepo->findAllPesananSurat();

      $loggedInUser = Auth::user();
      $realUser = $this->getRealUser($loggedInUser);

      $results = [];

      //CEK USER DEKAN
      if($realUser->id == $realUser->fakultas->id_dekan){
        // dd("s");
        $tempPesananSurats = PesananSurat::where('count','=',4)->get();
        foreach ($tempPesananSurats as $key => $surat) {
          array_push($results,$surat);
        }
        return view('pejabat.home_pejabat', [
          'pesanansurats' => $results,
          'user' => $realUser
        ]);
      }

      //CEK USER WD1
      if($realUser->id == $realUser->fakultas->id_WD_I){
        $tempPesananSurats = PesananSurat::where('count','=',3)->get();
        foreach ($tempPesananSurats as $key => $surat) {
          array_push($results,$surat);
        }
      }

      //CEK USER WD2
      // dd($realUser->id.' - '.$realUser->fakultas);
      if($realUser->id == $realUser->fakultas->id_WD_II){
        // dd("s");
        $tempPesananSurats = PesananSurat::where('count','=',2)->get();
        // dd($tempPesananSurats);
        foreach ($tempPesananSurats as $key => $surat) {
          // dd('asd');
          array_push($results,$surat);
        }
      }

      // CEK KALO USER ADALAH KETUA JURUSAN
      if($realUser->id == $realUser->jurusan->dosen->id){
        $tempPesananSurats = PesananSurat::where('count','=',1)->get();
        foreach ($tempPesananSurats as $key => $surat) {
          array_push($results,$surat);
        }
      }

      foreach ($realUser->mahasiswas as $key => $mhs) {
        foreach ($mhs->pesanansurats as $key => $surat) {
          array_push($results,$surat);
        }
      }

      return view('pejabat.home_pejabat', [
        'pesanansurats' => $results,
        'user' => $realUser
      ]);
    }

    public function updateFormulir(Request $request){
      // dd($request);
      $jsonArray = json_decode($request->dataSurat);
      $surat = $this->pesanansuratRepo->findPesananSuratById($request->idPesanansurat);
      // dd($surat);
      if($surat->count == 0){
        // DOSEN WALI = 0
        // dd($jsonArray);
        $dosenWali = explode('|',$request->dosenWali);
        $jsonArray->persetujuanDosenWali = $dosenWali[0];
        $jsonArray->catatanDosenWali = $dosenWali[1];
        $surat->persetujuanDosenWali = true;
        // dd($jsonArray);
      }else if($surat->count == 1){
      // KAPRODI = 1
        $kaprodi = explode('|',$request->kaprodi);
        $jsonArray->persetujuanKaprodi = $kaprodi[0];
        $jsonArray->catatanKaprodi = $kaprodi[1];
        $surat->persetujuanKaprodi = true;
      }else if($surat->count == 2){
        // WD2 = 2
        $wd2 = explode('|',$request->wd2);
        $jsonArray->persetujuanWDII = $wd2[0];
        $jsonArray->catatanWDII = $wd2[1];
        $surat->persetujuanWDII = true;
      }else if($surat->count == 3){
        // WD1 = 3
        $wd1 = explode('|',$request->wd1);
        $jsonArray->persetujuanWDI = $wd1[0];
        $jsonArray->catatanWDI = $wd1[1];
        $surat->persetujuanWDI = true;
      }else if($surat->count == 4){
        // DEKAN = 4
        $jsonArray->persetujuanDekan = $request->dekan;
        $surat->persetujuanDekan = true;
      }else{
        dd("count lebih dari 4 "+$surat->count);
      }
      $dataSuratUpdated = json_encode($jsonArray);
      // dd($dataSuratUpdated);
      $surat->dataSurat = $dataSuratUpdated;
      $surat->count += 1;
      $surat->save();
      // dd($surat);
      return redirect('/home_pejabat');
    }

    public function tambahPersetujuan(Request $request){
      $loggedInUser = Auth::user();
      $realUser = $this->getRealUser($loggedInUser);
      // dd($request->idPesananSurat);
      // $surat = PesananSurat::find($request->idPesananSurat);
      // dd($surat);
      // $surat->count += 1;
      // dd($surat);
      // $surat->save();

      $dataSurat = $request->dataSurat;
      $formatsurat_id = $request->idFormatSurat;
      return view('pejabat.tambah_persetujuan',[
        'dataSurat' => $dataSurat,
        'formatsurat_id' => $formatsurat_id,
        'user' => $realUser,
        'idPesananSurat' => $request->idPesananSurat
      ]);
    }

    public function previewDosen(Request $request){
      // dd($request);
        $loggedInUser = Auth::user();
      // dd($loggedInUser);
        $idPesanansurat = $request->idPesananSurat;
        // dd($idPesanansurat);
        $realUser = Dosen::find($loggedInUser->ref);
        // dd($realUser);
        if($request->formatsurat_id == "9"){
          $dataSurat = $request->dataSurat;
          $json = json_decode($dataSurat);
          $nama = $json->nama;
          $npm = $json->npm;
          $prodi = $json->prodi;
          $fakultas = $json->fakultas;
          $alamat = $json->alamat;
          $cutiStudiKe = $json->cutiStudiKe;
          $alasanCutiStudi = $json->alasanCutiStudi;
          $dosenWali = $json->dosenWali;
          $semester = $json->semester;
          $thnAkademik = $json->thnAkademik;
          $formatsurat_id = $request->idFormatSurat;
          // $dataSurat = $this->buatJSON($request);
          // dd($request->formatsurat_id);
          // dd("asd "+$request->idPesananSurat);
          // dd("asd");
          if($realUser->id == "5"){
              $persetujuanDekan = $request->persetujuan;
              $arrayJson = json_decode(PesananSurat::find($idPesanansurat)->dataSurat);
              return view('pejabat.preview_izin_cuti_studi', [
                'idPesanansurat' => $idPesanansurat,
                'nama' => $nama,
                'npm' => $npm,
                'prodi' => $prodi,
                'fakultas' => $fakultas,
                'alamat' => $alamat,
                'cutiStudiKe' => $cutiStudiKe,
                'alasanCutiStudi' => $alasanCutiStudi,
                'dosenWali' => $dosenWali,
                'semester' => $semester,
                'thnAkademik' => $thnAkademik,
                'formatsurat_id' => $formatsurat_id,
                'dataSurat' => $dataSurat,
                'user' => $realUser,
                'persetujuanDosenWali' => $arrayJson->persetujuanDosenWali,
                'catatanDosenWali' => $arrayJson->catatanDosenWali,
                'persetujuanKaprodi' => $arrayJson->persetujuanKaprodi,
                'catatanKaprodi' => $arrayJson->catatanKaprodi,
                'persetujuanWDII' => $arrayJson->persetujuanWDII,
                'catatanWDII' => $arrayJson->catatanWDII,
                'persetujuanWDI' => $arrayJson->persetujuanWDI,
                'catatanWDI' => $arrayJson->catatanWDI,
                'persetujuanDekan' => $persetujuanDekan,
                'idPesananSurat' => $request->idPesananSurat
              ]);
          }
          else if($realUser->id == "4"){
              $persetujuanWDI = $request->persetujuan;
              $catatanWDI = $request->catatan;
              $arrayJson = json_decode(PesananSurat::find($idPesanansurat)->dataSurat);
              return view('pejabat.preview_izin_cuti_studi', [
                'idPesanansurat' => $idPesanansurat,
                'nama' => $nama,
                'npm' => $npm,
                'prodi' => $prodi,
                'fakultas' => $fakultas,
                'alamat' => $alamat,
                'cutiStudiKe' => $cutiStudiKe,
                'alasanCutiStudi' => $alasanCutiStudi,
                'dosenWali' => $dosenWali,
                'semester' => $semester,
                'thnAkademik' => $thnAkademik,
                'formatsurat_id' => $formatsurat_id,
                'dataSurat' => $dataSurat,
                'user' => $realUser,
                'persetujuanDosenWali' => $arrayJson->persetujuanDosenWali,
                'catatanDosenWali' => $arrayJson->catatanDosenWali,
                'persetujuanKaprodi' => $arrayJson->persetujuanKaprodi,
                'catatanKaprodi' => $arrayJson->catatanKaprodi,
                'persetujuanWDII' => $arrayJson->persetujuanWDII,
                'catatanWDII' => $arrayJson->catatanWDII,
                'persetujuanWDI' => $persetujuanWDI,
                'catatanWDI' => $catatanWDI,
                'persetujuanDekan' => '-'

              ]);
          }
          else if($realUser->id == "3"){
              $persetujuanWDII = $request->persetujuan;
              $catatanWDII = $request->catatan;
              $arrayJson = json_decode(PesananSurat::find($idPesanansurat)->dataSurat);
              // dd($arrayJson);
              return view('pejabat.preview_izin_cuti_studi', [
                'idPesanansurat' => $idPesanansurat,
                'nama' => $nama,
                'npm' => $npm,
                'prodi' => $prodi,
                'fakultas' => $fakultas,
                'alamat' => $alamat,
                'cutiStudiKe' => $cutiStudiKe,
                'alasanCutiStudi' => $alasanCutiStudi,
                'dosenWali' => $dosenWali,
                'semester' => $semester,
                'thnAkademik' => $thnAkademik,
                'formatsurat_id' => $formatsurat_id,
                'dataSurat' => $dataSurat,
                'user' => $realUser,
                'persetujuanDosenWali' => $arrayJson->persetujuanDosenWali,
                'catatanDosenWali' => $arrayJson->catatanDosenWali,
                'persetujuanKaprodi' => $arrayJson->persetujuanKaprodi,
                'catatanKaprodi' => $arrayJson->catatanKaprodi,
                'persetujuanWDII' => $persetujuanWDII,
                'catatanWDII' => $catatanWDII,
                'persetujuanWDI'  => '-',
                'catatanWDI' => '-',
                'persetujuanDekan' => '-'

              ]);
          }
          else if($realUser->id == "6"){


              $surat = PesananSurat::find($idPesanansurat);
              // dd($surat);
              if($surat->count == 0){
                $persetujuanDosenWali = $request->persetujuan;
                $catatanDosenWali = $request->catatan;
                return view('pejabat.preview_izin_cuti_studi', [
                  'idPesanansurat' => $idPesanansurat,
                  'nama' => $nama,
                  'npm' => $npm,
                  'prodi' => $prodi,
                  'fakultas' => $fakultas,
                  'alamat' => $alamat,
                  'cutiStudiKe' => $cutiStudiKe,
                  'alasanCutiStudi' => $alasanCutiStudi,
                  'dosenWali' => $dosenWali,
                  'semester' => $semester,
                  'thnAkademik' => $thnAkademik,
                  'formatsurat_id' => $formatsurat_id,
                  'dataSurat' => $dataSurat,
                  'user' => $realUser,
                  'persetujuanDosenWali' => $persetujuanDosenWali,
                  'catatanDosenWali' => $catatanDosenWali,
                  'persetujuanKaprodi' => '-',
                  'catatanKaprodi' => '-',
                  'persetujuanWDII' => '-',
                  'catatanWDII' => '-',
                  'persetujuanWDI'  => '-',
                  'catatanWDI' => '-',
                  'persetujuanDekan' => '-'
                ]);
              }else{
                // dd("asd");
                $persetujuanKaprodi = $request->persetujuan;
                $catatanKaprodi = $request->catatan;
                $jsonArray = json_decode(PesananSurat::find($idPesanansurat)->dataSurat);
                // dd($jsonArray);
                return view('pejabat.preview_izin_cuti_studi', [
                  'idPesanansurat' => $idPesanansurat,
                  'nama' => $nama,
                  'npm' => $npm,
                  'prodi' => $prodi,
                  'fakultas' => $fakultas,
                  'alamat' => $alamat,
                  'cutiStudiKe' => $cutiStudiKe,
                  'alasanCutiStudi' => $alasanCutiStudi,
                  'dosenWali' => $dosenWali,
                  'semester' => $semester,
                  'thnAkademik' => $thnAkademik,
                  'formatsurat_id' => $formatsurat_id,
                  'dataSurat' => $dataSurat,
                  'user' => $realUser,
                  'persetujuanDosenWali' => $jsonArray->persetujuanDosenWali,
                  'catatanDosenWali' => $jsonArray->catatanDosenWali,
                  'persetujuanKaprodi' => $persetujuanKaprodi,
                  'catatanKaprodi' => $catatanKaprodi,
                  'persetujuanWDII' => '-',
                  'catatanWDII' => '-',
                  'persetujuanWDI'  => '-',
                  'catatanWDI' => '-',
                  'persetujuanDekan' => '-'
                ]);
              }
          }
          else{
              $persetujuanDosenWali = $request->persetujuan;
              $catatanDosenWali = $request->catatan;
              return view('pejabat.preview_izin_cuti_studi', [
                'idPesanansurat' => $idPesanansurat,
                'nama' => $nama,
                'npm' => $npm,
                'prodi' => $prodi,
                'fakultas' => $fakultas,
                'alamat' => $alamat,
                'cutiStudiKe' => $cutiStudiKe,
                'alasanCutiStudi' => $alasanCutiStudi,
                'dosenWali' => $dosenWali,
                'semester' => $semester,
                'thnAkademik' => $thnAkademik,
                'formatsurat_id' => $formatsurat_id,
                'dataSurat' => $dataSurat,
                'user' => $realUser,
                'persetujuanDosenWali' => $persetujuanDosenWali,
                'catatanDosenWali' => $catatanDosenWali,
                'persetujuanKaprodi' => '-',
                'catatanKaprodi' => '-',
                'persetujuanWDII' => '-',
                'catatanWDII' => '-',
                'persetujuanWDI'  => '-',
                'catatanWDI' => '-',
                'persetujuanDekan' => '-'
              ]);
          }

          // dd($request);

        }
        else if($request->formatsurat_id == "10"){
          $nama = $request->nama;
          $npm = $request->npm;
          $alamat = $request->alamat;
          $noTelepon = $request->noTelepon;
          $namaOrtu = $request->namaOrtu;
          $dosenWali = $request->dosenWali;
          $semester = $request->semester;
          if($realUser->id == "5"){
              $persetujuanDekan = $request->persetujuan;
          }
          else if($realUser->id == "4"){
              $persetujuanWDI = $request->persetujuan;
              $catatanWDI = $request->catatan;
          }
          else if($realUser->id == "3"){
              $persetujuanWDII = $request->persetujuan;
              $catatanWDII = $request->catatan;
          }
          else if($realUser->id == "6"){
              $persetujuanKaprodi = $request->persetujuan;
              $catatanKaprodi = $request->catatan;
          }
          else{
            $persetujuanDosenWali = $request->persetujuan;
            $catatanDosenWali = $request->catatan;
          }
          $formatsurat_id = $request->jenis_surat;
          $dataSurat = $this->buatJSON($request);
          return view('mahasiswa.preview_izin_pengunduran_diri', [
              'idPesanansurat' => $idPesanansurat,
              'nama' => $nama,
              'npm' => $npm,
              'alamat' => $alamat,
              'noTelepon' => $noTelepon,
              'namaOrtu' => $namaOrtu,
              'dosenWali' => $dosenWali,
              'semester' => $semester,
              'persetujuanDosenWali' => $persetujuanDosenWali,
              'catatanDosenWali' => $catatanDosenWali,
              'persetujuanKaprodi' => $persetujuanKaprodi,
              'catatanKaprodi' => $catatanKaprodi,
              'persetujuanWDII' => $persetujuanWDII,
              'catatanWDII' => $catatanWDII,
              'persetujuanWDI' => $persetujuanWDI,
              'catatanWDI' => $catatanWDI,
              'persetujuanDekan' => $persetujuanDekan,
              'formatsurat_id' => $formatsurat_id,
              'dataSurat' => $dataSurat,
              'user' => $realUser
          ]);
      }
    }

    private function getRealUser($loggedInUser){
      $realUser="";
      if($loggedInUser->jabatan == User::JABATAN_MHS){
        // dd($loggedInUser);
        $realUser = Mahasiswa::find($loggedInUser->ref);
        return $realUser;
      }else if($loggedInUser->jabatan == User::JABATAN_DOS){
        $realUser = Dosen::find($loggedInUser->ref);
        // dd($realUser);
        return $realUser;
      }else{ // TU
        $realUser = TU::find($loggedInUser->ref);
        // dd($loggedInUser->jabatan);
        return $realUser;
      }
    }

    public function tampilkanPesananSurat(Request $request){
          $loggedInUser = Auth::user();
          // dd($loggedInUser);
          $realUser = $this->getRealUser($loggedInUser);

          $pesanansurats;
          if($request->kategori == "jenis_surat"){
            $pesanansurats = $this->pesanansuratRepo->findMahasiswaByJenisSurat($request->searchBox);
          }
          else if($request->kategori == "perihal"){
            $pesanansurats = $this->pesanansuratRepo->findMahasiswaByPerihal($request->searchBox);
          }
          else if($request->kategori == "penerima_surat"){
            $pesanansurats = $this->pesanansuratRepo->findPesananSuratByPenerimaSurat($request->searchBox);
          }
          else if($request->kategori == "pengirimSurat"){
            $pesanansurats = $this->pesanansuratRepo->findMahasiswaByPengirimSurat($request->searchBox);
          }
          else if($request->kategori == "tanggalPembuatan"){
            $pesanansurats = $this->pesanansuratRepo->findMahasiswaByTanggalPembuatan($request->searchBox);
          }
          else{
            $pesanansurats = $this->pesanansuratRepo->findAllPesananSurat();
          }
          // dd($pesanansurats[0]);
          return view('TU.home_TU',[
              'pesanansurats' => $pesanansurats,
              'user' => $realUser
          ]);
  	}
    public function sendDataSurat(Request $request){
      $loggedInUser = Auth::user();
      // dd($loggedInUser);
      $realUser = $this->getRealUser($loggedInUser);

      if($request->idFormatSurat == "1"){
        $dataSurat = $request->prosesSurat;
        $json = json_decode($dataSurat);
        $nama = $json->nama;
        $prodi = $json->prodi;
        $npm = $json->npm;
        $user = Mahasiswa::where('npm',$json->npm)->first();
        // dd($user);
        $semester = $json->semester;
        $thnAkademik = $json->thnAkademik;
        $penyediabeasiswa = $json->penyediabeasiswa;
        $formatsurat_id = $request->idFormatSurat;
        $pesananID = $this->pesanansuratRepo->findPesananSuratById($request->id);
        $pemesan = $pesananID->mahasiswa_id;
        // dd($request->id);
        // dd($pemesan);
        return view('TU.proses_surat_keterangan_beasiswa', [
            'nama' => $nama,
            'prodi' => $prodi,
            'npm' => $npm,
            'semester' => $semester,
            'thnAkademik' => $thnAkademik,
            'penyediabeasiswa' => $penyediabeasiswa,
            'formatsurat_id' => $formatsurat_id,
            'dataSurat' => $dataSurat,
            'user' => $user,
            'pemesan' => $pemesan
        ]);
      }
      else if($request->idFormatSurat == "2"){
        $dataSurat = $request->prosesSurat;
        $json = json_decode($dataSurat);
        $nama = $json->nama;
        $prodi = $json->prodi;
        $npm = $json->npm;
        $kota_lahir = $json->kota_lahir;
        $tglLahir = $json->tglLahir;
        $semester = $json->semester;
        $alamat = $json->alamat;
        $formatsurat_id = $request->idFormatSurat;
        $pesananID = $this->pesanansuratRepo->findPesananSuratById($request->id);
        $pemesan = $pesananID->mahasiswa_id;
        $user = Mahasiswa::where('npm',$json->npm)->first();
        // dd($dataSurat);
        return view('TU.proses_surat_keterangan_mahasiswa_aktif', [
            'nama' => $nama,
            'prodi' => $prodi,
            'npm' => $npm,
            'kota_lahir' => $kota_lahir,
            'tglLahir' => $tglLahir,
            'alamat' => $alamat,
            'semester' => $semester,
            'formatsurat_id' => $formatsurat_id,
            'dataSurat' => $dataSurat,
            'user' => $user,
            'pemesan' => $pemesan
        ]);
      }
      else if($request->idFormatSurat == "3"){
        $dataSurat = $request->prosesSurat;
        $json = json_decode($dataSurat);
        $nama = $json->nama;
        $tglLahir = $json->tglLahir;
        $kewarganegaraan = $json->kewarganegaraan;
        $organisasiTujuan = $json->organisasiTujuan;
        $thnAkademik = $json->thnAkademik;
        $negaraTujuan = $json->negaraTujuan;
        $tanggalKunjungan = $json->tanggalKunjungan;
        $formatsurat_id = $request->idFormatSurat;
        $pesananID = $this->pesanansuratRepo->findPesananSuratById($request->id);
        $pemesan = $pesananID->mahasiswa_id;
        $user = Mahasiswa::where('npm',$json->npm)->first();
        // dd($dataSurat);
        return view('TU.proses_surat_pembuatan_visa', [
            'nama' => $nama,
            'tglLahir' => $tglLahir,
            'kewarganegaraan' => $kewarganegaraan,
            'organisasiTujuan' => $organisasiTujuan,
            'thnAkademik' => $thnAkademik,
            'negaraTujuan' => $negaraTujuan,
            'tanggalKunjungan' => $tanggalKunjungan,
            'formatsurat_id' => $formatsurat_id,
            'dataSurat' => $dataSurat,
            'user' => $user,
            'pemesan' => $pemesan
        ]);
      }
      else if($request->idFormatSurat == "4"){
        $dataSurat = $request->prosesSurat;
        $json = json_decode($dataSurat);
        $nama = $json->nama;
        $npm = $json->npm;
        $prodi = $json->prodi;
        $matkul = $json->matkul;
        $topik = $json->topik;
        $organisasi = $json->organisasi;
        $alamatOrganisasi = $json->alamatOrganisasi;
        $keperluanKunjungan = $json->keperluanKunjungan;
        $kota = $json->kota;
        $kepada = $json->kepada;
        $formatsurat_id = $request->idFormatSurat;
        $pesananID = $this->pesanansuratRepo->findPesananSuratById($request->id);
        $pemesan = $pesananID->mahasiswa_id;
        $user = Mahasiswa::where('npm',$json->npm)->first();
        // dd($request);
        return view('TU.proses_surat_izin_studi_lapangan_1org', [
            'nama' => $nama,
            'npm' => $npm,
            'prodi' => $prodi,
            'matkul' => $matkul,
            'topik' => $topik,
            'organisasi' => $organisasi,
            'alamatOrganisasi' => $alamatOrganisasi,
            'keperluanKunjungan' => $keperluanKunjungan,
            'kota' => $kota,
            'kepada' => $kepada,
            'formatsurat_id' => $formatsurat_id,
            'dataSurat' => $dataSurat,
            'user' => $user,
            'pemesan' => $pemesan
        ]);
      }
      else if($request->idFormatSurat == "5"){
        $dataSurat = $request->prosesSurat;
        $json = json_decode($dataSurat);
        $nama = $json->nama;
        $npm = $json->npm;
        $prodi = $json->prodi;
        $matkul = $json->matkul;
        $topik = $json->topik;
        $organisasi = $json->organisasi;
        $alamatOrganisasi = $json->alamatOrganisasi;
        $keperluanKunjungan = $json->keperluanKunjungan;
        $kota = $json->kota;
        $kepada = $json->kepada;
        $namaAnggota = $json->namaAnggota;
        $npmAnggota = $json->npmAnggota;
        $formatsurat_id = $request->idFormatSurat;
        $pesananID = $this->pesanansuratRepo->findPesananSuratById($request->id);
        $pemesan = $pesananID->mahasiswa_id;
        $user = Mahasiswa::where('npm',$json->npm)->first();
        return view('TU.proses_surat_izin_studi_lapangan_2org', [
            'nama' => $nama,
            'npm' => $npm,
            'prodi' => $prodi,
            'matkul' => $matkul,
            'topik' => $topik,
            'organisasi' => $organisasi,
            'alamatOrganisasi' => $alamatOrganisasi,
            'keperluanKunjungan' => $keperluanKunjungan,
            'kota' => $kota,
            'kepada' => $kepada,
            'namaAnggota' => $namaAnggota,
            'npmAnggota' => $npmAnggota,
            'formatsurat_id' => $formatsurat_id,
            'dataSurat' => $dataSurat,
            'user' => $user,
            'pemesan' => $pemesan
        ]);
      }
      else if($request->idFormatSurat == "6"){
        $dataSurat = $request->prosesSurat;
        $json = json_decode($dataSurat);
        $nama = $json->nama;
        $npm = $json->npm;
        $prodi = $json->prodi;
        $matkul = $json->matkul;
        $topik = $json->topik;
        $organisasi = $json->organisasi;
        $alamatOrganisasi = $json->alamatOrganisasi;
        $keperluanKunjungan = $json->keperluanKunjungan;
        $kota = $json->kota;
        $kepada = $json->kepada;
        $namaAnggota1 = $json->namaAnggota1;
        $npmAnggota1 = $json->npmAnggota1;
        $namaAnggota2 = $json->namaAnggota2;
        $npmAnggota2 = $json->npmAnggota2;
        $formatsurat_id = $request->idFormatSurat;
        $pesananID = $this->pesanansuratRepo->findPesananSuratById($request->id);
        $pemesan = $pesananID->mahasiswa_id;
        $user = Mahasiswa::where('npm',$json->npm)->first();
        return view('TU.proses_surat_izin_studi_lapangan_3org', [
            'nama' => $nama,
            'npm' => $npm,
            'prodi' => $prodi,
            'matkul' => $matkul,
            'topik' => $topik,
            'organisasi' => $organisasi,
            'alamatOrganisasi' => $alamatOrganisasi,
            'keperluanKunjungan' => $keperluanKunjungan,
            'kota' => $kota,
            'kepada' => $kepada,
            'namaAnggota1' => $namaAnggota1,
            'npmAnggota1' => $npmAnggota1,
            'namaAnggota2' => $namaAnggota2,
            'npmAnggota2' => $npmAnggota2,
            'formatsurat_id' => $formatsurat_id,
            'dataSurat' => $dataSurat,
            'user' => $user,
            'pemesan' => $pemesan
        ]);
      }
      else if($request->idFormatSurat == "7"){
        $dataSurat = $request->prosesSurat;
        $json = json_decode($dataSurat);
        $nama = $json->nama;
        $npm = $json->npm;
        $prodi = $json->prodi;
        $matkul = $json->matkul;
        $topik = $json->topik;
        $organisasi = $json->organisasi;
        $alamatOrganisasi = $json->alamatOrganisasi;
        $keperluanKunjungan = $json->keperluanKunjungan;
        $kota = $json->kota;
        $kepada = $json->kepada;
        $namaAnggota1 = $json->namaAnggota1;
        $npmAnggota1 = $json->npmAnggota1;
        $namaAnggota2 = $json->namaAnggota2;
        $npmAnggota2 = $json->npmAnggota2;
        $namaAnggota3 = $json->namaAnggota3;
        $npmAnggota3 = $json->npmAnggota3;
        $formatsurat_id = $request->idFormatSurat;
        $pesananID = $this->pesanansuratRepo->findPesananSuratById($request->id);
        $pemesan = $pesananID->mahasiswa_id;
        $user = Mahasiswa::where('npm',$json->npm)->first();
        return view('TU.proses_surat_izin_studi_lapangan_4org', [
            'nama' => $nama,
            'npm' => $npm,
            'prodi' => $prodi,
            'matkul' => $matkul,
            'topik' => $topik,
            'organisasi' => $organisasi,
            'alamatOrganisasi' => $alamatOrganisasi,
            'keperluanKunjungan' => $keperluanKunjungan,
            'kota' => $kota,
            'kepada' => $kepada,
            'namaAnggota1' => $namaAnggota1,
            'npmAnggota1' => $npmAnggota1,
            'namaAnggota2' => $namaAnggota2,
            'npmAnggota2' => $npmAnggota2,
            'namaAnggota3' => $namaAnggota3,
            'npmAnggota3' => $npmAnggota3,
            'formatsurat_id' => $formatsurat_id,
            'dataSurat' => $dataSurat,
            'user' => $user,
            'pemesan' => $pemesan
        ]);
      }
      else if($request->idFormatSurat == "8"){
        $dataSurat = $request->prosesSurat;
        $json = json_decode($dataSurat);
        $nama = $json->nama;
        $npm = $json->npm;
        $prodi = $json->prodi;
        $matkul = $json->matkul;
        $topik = $json->topik;
        $organisasi = $json->organisasi;
        $alamatOrganisasi = $json->alamatOrganisasi;
        $keperluanKunjungan = $json->keperluanKunjungan;
        $kota = $json->kota;
        $kepada = $json->kepada;
        $namaAnggota1 = $json->namaAnggota1;
        $npmAnggota1 = $json->npmAnggota1;
        $namaAnggota2 = $json->namaAnggota2;
        $npmAnggota2 = $json->npmAnggota2;
        $namaAnggota3 = $json->namaAnggota3;
        $npmAnggota3 = $json->npmAnggota3;
        $namaAnggota4 = $json->namaAnggota4;
        $npmAnggota4 = $json->npmAnggota4;
        $formatsurat_id = $request->idFormatSurat;
        $pesananID = $this->pesanansuratRepo->findPesananSuratById($request->id);
        $pemesan = $pesananID->mahasiswa_id;
        $user = Mahasiswa::where('npm',$json->npm)->first();
        return view('TU.proses_surat_izin_studi_lapangan_5org', [
            'nama' => $nama,
            'npm' => $npm,
            'prodi' => $prodi,
            'matkul' => $matkul,
            'topik' => $topik,
            'organisasi' => $organisasi,
            'alamatOrganisasi' => $alamatOrganisasi,
            'keperluanKunjungan' => $keperluanKunjungan,
            'kota' => $kota,
            'kepada' => $kepada,
            'namaAnggota1' => $namaAnggota1,
            'npmAnggota1' => $npmAnggota1,
            'namaAnggota2' => $namaAnggota2,
            'npmAnggota2' => $npmAnggota2,
            'namaAnggota3' => $namaAnggota3,
            'npmAnggota3' => $npmAnggota3,
            'namaAnggota4' => $namaAnggota4,
            'npmAnggota4' => $npmAnggota4,
            'formatsurat_id' => $formatsurat_id,
            'dataSurat' => $dataSurat,
            'user' => $user,
            'pemesan' => $pemesan
        ]);
      }
      else if($request->idFormatSurat == "9"){
        $dataSurat = $request->prosesSurat;
        $json = json_decode($dataSurat);
        $nama = $json->nama;
        $npm = $json->npm;
        $prodi = $json->prodi;
        $fakultas = $json->fakultas;
        $alamat = $json->alamat;
        $cutiStudiKe = $json->cutiStudiKe;
        $alasanCutiStudi = $json->alasanCutiStudi;
        $dosenWali = $json->dosenWali;
        $semester = $json->semester;
        $thnAkademik = $json->thnAkademik;
        $pesananID = $this->pesanansuratRepo->findPesananSuratById($request->id);
        $pemesan = $pesananID->mahasiswa_id;
        $persetujuanDosenWali = $json->persetujuanDosenWali;
        $catatanDosenWali = $json->catatanDosenWali;
        $persetujuanKaprodi = $json->persetujuanKaprodi;
        $catatanKaprodi = $json->catatanKaprodi;
        $persetujuanWDII = $json->persetujuanWDII;
        $catatanWDII = $json->catatanWDII;
        $persetujuanWDI = $json->persetujuanWDI;
        $catatanWDI = $json->catatanWDI;
        $persetujuanDekan = $json->persetujuanDekan;
        $formatsurat_id = $request->idFormatSurat;
        $user = Mahasiswa::where('npm',$json->npm)->first();
        return view('TU.proses_surat_izin_cuti_studi', [
            'nama' => $nama,
            'npm' => $npm,
            'prodi' => $prodi,
            'fakultas' => $fakultas,
            'alamat' => $alamat,
            'cutiStudiKe' => $cutiStudiKe,
            'alasanCutiStudi' => $alasanCutiStudi,
            'dosenWali' => $dosenWali,
            'semester' => $semester,
            'thnAkademik' => $thnAkademik,
            'persetujuanDosenWali' => $persetujuanDosenWali,
            'catatanDosenWali' => $catatanDosenWali,
            'persetujuanKaprodi' => $persetujuanKaprodi,
            'catatanKaprodi' => $catatanKaprodi,
            'persetujuanWDII' => $persetujuanWDII,
            'catatanWDII' => $catatanWDII,
            'persetujuanWDI' => $persetujuanWDI,
            'catatanWDI' => $catatanWDI,
            'persetujuanDekan' => $persetujuanDekan,
            'formatsurat_id' => $formatsurat_id,
            'dataSurat' => $dataSurat,
            'user' => $user,
            'pemesan' => $pemesan
        ]);
      }
      else if($request->idFormatSurat == "10"){
        $dataSurat = $request->prosesSurat;
        $json = json_decode($dataSurat);
        $nama = $json->nama;
        $npm = $json->npm;
        $alamat = $json->alamat;
        $noTelepon = $json->noTelepon;
        $namaOrtu = $json->namaOrtu;
        $dosenWali = $json->dosenWali;
        $semester = $json->semester;
        $pesananID = $this->pesanansuratRepo->findPesananSuratById($request->id);
        $pemesan = $pesananID->mahasiswa_id;
        $user = Mahasiswa::where('npm',$json->npm)->first();
        //upload
        // $lampiran = $request->file('lampiran_CutiStudi');
        // $destination_path = ('lampiran/cuti_studi/');
        // $filename = $lampiran->getClientOriginalName();
        // $namaDepan = explode(" ", $nama);
        // $savedLampiran = ($namaDepan[0] . '_' . $namaDepan[1] . '_' .$filename);
        // $lampiran->move($destination_path, $savedLampiran);

        // $link = '127.0.0.1:8000/format_surat_latex/' . $filename;
        $persetujuanDosenWali = '-';
        $catatanDosenWali = '-';
        $persetujuanKaprodi = '-';
        $catatanKaprodi = '-';
        $persetujuanWDII = '-';
        $catatanWDII = '-';
        $persetujuanWDI = '-';
        $catatanWDI = '-';
        $persetujuanDekan = '-';
        $formatsurat_id = $request->idFormatSurat;
        return view('TU.proses_surat_izin_pengunduran_diri', [
            'nama' => $nama,
            'npm' => $npm,
            'alamat' => $alamat,
            'noTelepon' => $noTelepon,
            'namaOrtu' => $namaOrtu,
            'dosenWali' => $dosenWali,
            'semester' => $semester,
            'persetujuanDosenWali' => $persetujuanDosenWali,
            'catatanDosenWali' => $catatanDosenWali,
            'persetujuanKaprodi' => $persetujuanKaprodi,
            'catatanKaprodi' => $catatanKaprodi,
            'persetujuanWDII' => $persetujuanWDII,
            'catatanWDII' => $catatanWDII,
            'persetujuanWDI' => $persetujuanWDI,
            'catatanWDI' => $catatanWDI,
            'persetujuanDekan' => $persetujuanDekan,
            'formatsurat_id' => $formatsurat_id,
            'dataSurat' => $dataSurat,
            'user' => $user,
            'pemesan' => $pemesan
        ]);
      }
      else if($request->idFormatSurat == "11"){
        $dataSurat = $request->prosesSurat;
        $json = json_decode($dataSurat);
        $semester = $json->semester;
        $thnAkademik = $json->thnAkademik;
        $nama = $json->nama;
        $prodi = $json->prodi;
        $npm = $json->npm;
        $namaWakil = $json->namaWakil;
        $prodiWakil = $json->prodiWakil;
        $npmWakil = $json->npmWakil;
        $dosenWali = $json->dosenWali;
        $alasan = $json->alasan;
        $kodeMK =$json->kodeMK;
        $matkul = $json->matkul;
        $sks = $json->sks;
        $formatsurat_id = $request->idFormatSurat;
        $pesananID = $this->pesanansuratRepo->findPesananSuratById($request->id);
        $pemesan = $pesananID->mahasiswa_id;
        $user = Mahasiswa::where('npm',$json->npm)->first();
        return view('TU.proses_surat_perwakilan_perwalian_1mk', [
            'semester' => $semester,
            'thnAkademik' => $thnAkademik,
            'nama' => $nama,
            'prodi' => $prodi,
            'npm' => $npm,
            'namaWakil' => $namaWakil,
            'prodiWakil' => $prodiWakil,
            'npmWakil' => $npmWakil,
            'dosenWali' => $dosenWali,
            'alasan' => $alasan,
            'kodeMK' => $kodeMK,
            'matkul' => $matkul,
            'sks' => $sks,
            'formatsurat_id' => $formatsurat_id,
            'dataSurat' => $dataSurat,
            'user' => $user,
            'pemesan' => $pemesan
        ]);
      }
      else if($request->idFormatSurat == "12"){
        $dataSurat = $request->prosesSurat;
        $json = json_decode($dataSurat);
        $semester = $json->semester;
        $thnAkademik = $json->thnAkademik;
        $nama = $json->nama;
        $prodi = $json->prodi;
        $npm = $json->npm;
        $namaWakil = $json->namaWakil;
        $prodiWakil = $json->prodiWakil;
        $npmWakil = $json->npmWakil;
        $dosenWali = $json->dosenWali;
        $alasan = $json->alasan;
        $kodeMK1 = $json->kodeMK1;
        $matkul1 = $json->matkul1;
        $sks1 = $json->sks1;
        $kodeMK2 = $json->kodeMK2;
        $matkul2 = $json->matkul2;
        $sks2 = $json->sks2;
        $formatsurat_id = $request->idFormatSurat;
        $pesananID = $this->pesanansuratRepo->findPesananSuratById($request->id);
        $pemesan = $pesananID->mahasiswa_id;
        $user = Mahasiswa::where('npm',$json->npm)->first();
        return view('TU.proses_surat_perwakilan_perwalian_2mk', [
            'semester' => $semester,
            'thnAkademik' => $thnAkademik,
            'nama' => $nama,
            'prodi' => $prodi,
            'npm' => $npm,
            'namaWakil' => $namaWakil,
            'prodiWakil' => $prodiWakil,
            'npmWakil' => $npmWakil,
            'dosenWali' => $dosenWali,
            'alasan' => $alasan,
            'kodeMK1' => $kodeMK1,
            'matkul1' => $matkul1,
            'sks1' => $sks1,
            'kodeMK2' => $kodeMK2,
            'matkul2' => $matkul2,
            'sks2' => $sks2,
            'formatsurat_id' => $formatsurat_id,
            'dataSurat' => $dataSurat,
            'user' => $user,
            'pemesan' => $pemesan
        ]);
      }
      else if($request->idFormatSurat == "13"){
        $dataSurat = $request->prosesSurat;
        $json = json_decode($dataSurat);
        $semester = $json->semester;
        $thnAkademik = $json->thnAkademik;
        $nama = $json->nama;
        $prodi = $json->prodi;
        $npm = $json->npm;
        $namaWakil = $json->namaWakil;
        $prodiWakil = $json->prodiWakil;
        $npmWakil = $json->npmWakil;
        $dosenWali = $json->dosenWali;
        $alasan = $json->alasan;
        $kodeMK1 = $json->kodeMK1;
        $matkul1 = $json->matkul1;
        $sks1 = $json->sks1;
        $kodeMK2 = $json->kodeMK2;
        $matkul2 = $json->matkul2;
        $sks2 = $json->sks2;
        $kodeMK3 = $json->kodeMK3;
        $matkul3 = $json->matkul3;
        $sks3 = $json->sks3;
        $formatsurat_id = $request->idFormatSurat;
        $pesananID = $this->pesanansuratRepo->findPesananSuratById($request->id);
        $pemesan = $pesananID->mahasiswa_id;
        $user = Mahasiswa::where('npm',$json->npm)->first();
        return view('TU.proses_surat_perwakilan_perwalian_3mk', [
            'semester' => $semester,
            'thnAkademik' => $thnAkademik,
            'nama' => $nama,
            'prodi' => $prodi,
            'npm' => $npm,
            'namaWakil' => $namaWakil,
            'prodiWakil' => $prodiWakil,
            'npmWakil' => $npmWakil,
            'dosenWali' => $dosenWali,
            'alasan' => $alasan,
            'kodeMK1' => $kodeMK1,
            'matkul1' => $matkul1,
            'sks1' => $sks1,
            'kodeMK2' => $kodeMK2,
            'matkul2' => $matkul2,
            'sks2' => $sks2,
            'kodeMK3' => $kodeMK3,
            'matkul3' => $matkul3,
            'sks3' => $sks3,
            'formatsurat_id' => $formatsurat_id,
            'dataSurat' => $dataSurat,
            'user' => $user,
            'pemesan' => $pemesan
        ]);
      }
      else if($request->idFormatSurat == "14"){
        $dataSurat = $request->prosesSurat;
        $json = json_decode($dataSurat);
        $semester = $json->semester;
        $thnAkademik = $json->thnAkademik;
        $nama = $json->nama;
        $prodi = $json->prodi;
        $npm = $json->npm;
        $namaWakil = $json->namaWakil;
        $prodiWakil = $json->prodiWakil;
        $npmWakil = $json->npmWakil;
        $dosenWali = $json->dosenWali;
        $alasan = $json->alasan;
        $kodeMK1 = $json->kodeMK1;
        $matkul1 = $json->matkul1;
        $sks1 = $json->sks1;
        $kodeMK2 = $json->kodeMK2;
        $matkul2 = $json->matkul2;
        $sks2 = $json->sks2;
        $kodeMK3 = $json->kodeMK3;
        $matkul3 = $json->matkul3;
        $sks3 = $json->sks3;
        $kodeMK4 = $json->kodeMK4;
        $matkul4 = $json->matkul4;
        $sks4 = $json->sks4;
        $formatsurat_id = $request->idFormatSurat;
        $pesananID = $this->pesanansuratRepo->findPesananSuratById($request->id);
        $pemesan = $pesananID->mahasiswa_id;
        $user = Mahasiswa::where('npm',$json->npm)->first();
        return view('TU.proses_surat_perwakilan_perwalian_4mk', [
            'semester' => $semester,
            'thnAkademik' => $thnAkademik,
            'nama' => $nama,
            'prodi' => $prodi,
            'npm' => $npm,
            'namaWakil' => $namaWakil,
            'prodiWakil' => $prodiWakil,
            'npmWakil' => $npmWakil,
            'dosenWali' => $dosenWali,
            'alasan' => $alasan,
            'kodeMK1' => $kodeMK1,
            'matkul1' => $matkul1,
            'sks1' => $sks1,
            'kodeMK2' => $kodeMK2,
            'matkul2' => $matkul2,
            'sks2' => $sks2,
            'kodeMK3' => $kodeMK3,
            'matkul3' => $matkul3,
            'sks3' => $sks3,
            'kodeMK4' => $kodeMK4,
            'matkul4' => $matkul4,
            'sks4' => $sks4,
            'formatsurat_id' => $formatsurat_id,
            'dataSurat' => $dataSurat,
            'user' => $user,
            'pemesan' => $pemesan
        ]);
      }
      else if($request->idFormatSurat == "15"){
        $dataSurat = $request->prosesSurat;
        $json = json_decode($dataSurat);
        $semester = $json->semester;
        $thnAkademik = $json->thnAkademik;
        $nama = $json->nama;
        $prodi = $json->prodi;
        $npm = $json->npm;
        $namaWakil = $json->namaWakil;
        $prodiWakil = $json->prodiWakil;
        $npmWakil = $json->npmWakil;
        $dosenWali = $json->dosenWali;
        $alasan = $json->alasan;
        $kodeMK1 = $json->kodeMK1;
        $matkul1 = $json->matkul1;
        $sks1 = $json->sks1;
        $kodeMK2 = $json->kodeMK2;
        $matkul2 = $json->matkul2;
        $sks2 = $json->sks2;
        $kodeMK3 = $json->kodeMK3;
        $matkul3 = $json->matkul3;
        $sks3 = $json->sks3;
        $kodeMK4 = $json->kodeMK4;
        $matkul4 = $json->matkul4;
        $sks4 = $json->sks4;
        $kodeMK5 = $json->kodeMK5;
        $matkul5 = $json->matkul5;
        $sks5 = $json->sks5;
        $formatsurat_id = $request->idFormatSurat;
        $pesananID = $this->pesanansuratRepo->findPesananSuratById($request->id);
        $pemesan = $pesananID->mahasiswa_id;
        $user = Mahasiswa::where('npm',$json->npm)->first();
        return view('TU.proses_surat_perwakilan_perwalian_5mk', [
            'semester' => $semester,
            'thnAkademik' => $thnAkademik,
            'nama' => $nama,
            'prodi' => $prodi,
            'npm' => $npm,
            'namaWakil' => $namaWakil,
            'prodiWakil' => $prodiWakil,
            'npmWakil' => $npmWakil,
            'dosenWali' => $dosenWali,
            'alasan' => $alasan,
            'kodeMK1' => $kodeMK1,
            'matkul1' => $matkul1,
            'sks1' => $sks1,
            'kodeMK2' => $kodeMK2,
            'matkul2' => $matkul2,
            'sks2' => $sks2,
            'kodeMK3' => $kodeMK3,
            'matkul3' => $matkul3,
            'sks3' => $sks3,
            'kodeMK4' => $kodeMK4,
            'matkul4' => $matkul4,
            'sks4' => $sks4,
            'kodeMK5' => $kodeMK5,
            'matkul5' => $matkul5,
            'sks5' => $sks5,
            'formatsurat_id' => $formatsurat_id,
            'dataSurat' => $dataSurat,
            'user' => $user,
            'pemesan' => $pemesan
        ]);
      }
      else if($request->idFormatSurat == "16"){
        $dataSurat = $request->prosesSurat;
        $json = json_decode($dataSurat);
        $semester = $json->semester;
        $thnAkademik = $json->thnAkademik;
        $nama = $json->nama;
        $prodi = $json->prodi;
        $npm = $json->npm;
        $namaWakil = $json->namaWakil;
        $prodiWakil = $json->prodiWakil;
        $npmWakil = $json->npmWakil;
        $dosenWali = $json->dosenWali;
        $alasan = $json->alasan;
        $kodeMK1 = $json->kodeMK1;
        $matkul1 = $json->matkul1;
        $sks1 = $json->sks1;
        $kodeMK2 = $json->kodeMK2;
        $matkul2 = $json->matkul2;
        $sks2 = $json->sks2;
        $kodeMK3 = $json->kodeMK3;
        $matkul3 = $json->matkul3;
        $sks3 = $json->sks3;
        $kodeMK4 = $json->kodeMK4;
        $matkul4 = $json->matkul4;
        $sks4 = $json->sks4;
        $kodeMK5 = $json->kodeMK5;
        $matkul5 = $json->matkul5;
        $sks5 = $json->sks5;
        $kodeMK6 = $json->kodeMK6;
        $matkul6 = $json->matkul6;
        $sks6 = $json->sks6;
        $formatsurat_id = $request->idFormatSurat;
        $pesananID = $this->pesanansuratRepo->findPesananSuratById($request->id);
        $pemesan = $pesananID->mahasiswa_id;
        $user = Mahasiswa::where('npm',$json->npm)->first();
        return view('TU.proses_surat_perwakilan_perwalian_6mk', [
            'semester' => $semester,
            'thnAkademik' => $thnAkademik,
            'nama' => $nama,
            'prodi' => $prodi,
            'npm' => $npm,
            'namaWakil' => $namaWakil,
            'prodiWakil' => $prodiWakil,
            'npmWakil' => $npmWakil,
            'dosenWali' => $dosenWali,
            'alasan' => $alasan,
            'kodeMK1' => $kodeMK1,
            'matkul1' => $matkul1,
            'sks1' => $sks1,
            'kodeMK2' => $kodeMK2,
            'matkul2' => $matkul2,
            'sks2' => $sks2,
            'kodeMK3' => $kodeMK3,
            'matkul3' => $matkul3,
            'sks3' => $sks3,
            'kodeMK4' => $kodeMK4,
            'matkul4' => $matkul4,
            'sks4' => $sks4,
            'kodeMK5' => $kodeMK5,
            'matkul5' => $matkul5,
            'sks5' => $sks5,
            'kodeMK6' => $kodeMK6,
            'matkul6' => $matkul6,
            'sks6' => $sks6,
            'formatsurat_id' => $formatsurat_id,
            'dataSurat' => $dataSurat,
            'user' => $user,
            'pemesan' => $pemesan
        ]);
      }
      else if($request->idFormatSurat == "17"){
        $dataSurat = $request->prosesSurat;
        $json = json_decode($dataSurat);
        $semester = $json->semester;
        $thnAkademik = $json->thnAkademik;
        $nama = $json->nama;
        $prodi = $json->prodi;
        $npm = $json->npm;
        $namaWakil = $json->namaWakil;
        $prodiWakil = $json->prodiWakil;
        $npmWakil = $json->npmWakil;
        $dosenWali = $json->dosenWali;
        $alasan = $json->alasan;
        $kodeMK1 = $json->kodeMK1;
        $matkul1 = $json->matkul1;
        $sks1 = $json->sks1;
        $kodeMK2 = $json->kodeMK2;
        $matkul2 = $json->matkul2;
        $sks2 = $json->sks2;
        $kodeMK3 = $json->kodeMK3;
        $matkul3 = $json->matkul3;
        $sks3 = $json->sks3;
        $kodeMK4 = $json->kodeMK4;
        $matkul4 = $json->matkul4;
        $sks4 = $json->sks4;
        $kodeMK5 = $json->kodeMK5;
        $matkul5 = $json->matkul5;
        $sks5 = $json->sks5;
        $kodeMK6 = $json->kodeMK6;
        $matkul6 = $json->matkul6;
        $sks6 = $json->sks6;
        $kodeMK7 = $json->kodeMK7;
        $matkul7 = $json->matkul7;
        $sks7 = $json->sks7;
        $formatsurat_id = $request->idFormatSurat;
        $pesananID = $this->pesanansuratRepo->findPesananSuratById($request->id);
        $pemesan = $pesananID->mahasiswa_id;
        $user = Mahasiswa::where('npm',$json->npm)->first();
        return view('TU.proses_surat_perwakilan_perwalian_7mk', [
            'semester' => $semester,
            'thnAkademik' => $thnAkademik,
            'nama' => $nama,
            'prodi' => $prodi,
            'npm' => $npm,
            'namaWakil' => $namaWakil,
            'prodiWakil' => $prodiWakil,
            'npmWakil' => $npmWakil,
            'dosenWali' => $dosenWali,
            'alasan' => $alasan,
            'kodeMK1' => $kodeMK1,
            'matkul1' => $matkul1,
            'sks1' => $sks1,
            'kodeMK2' => $kodeMK2,
            'matkul2' => $matkul2,
            'sks2' => $sks2,
            'kodeMK3' => $kodeMK3,
            'matkul3' => $matkul3,
            'sks3' => $sks3,
            'kodeMK4' => $kodeMK4,
            'matkul4' => $matkul4,
            'sks4' => $sks4,
            'kodeMK5' => $kodeMK5,
            'matkul5' => $matkul5,
            'sks5' => $sks5,
            'kodeMK6' => $kodeMK6,
            'matkul6' => $matkul6,
            'sks6' => $sks6,
            'kodeMK7' => $kodeMK7,
            'matkul7' => $matkul7,
            'sks7' => $sks7,
            'formatsurat_id' => $formatsurat_id,
            'dataSurat' => $dataSurat,
            'user' => $user,
            'pemesan' => $pemesan
        ]);
      }
      else if($request->idFormatSurat == "18"){
        $dataSurat = $request->prosesSurat;
        $json = json_decode($dataSurat);
        $semester = $json->semester;
        $thnAkademik = $json->thnAkademik;
        $nama = $json->nama;
        $prodi = $json->prodi;
        $npm = $json->npm;
        $namaWakil = $json->namaWakil;
        $prodiWakil = $json->prodiWakil;
        $npmWakil = $json->npmWakil;
        $dosenWali = $json->dosenWali;
        $alasan = $json->alasan;
        $kodeMK1 = $json->kodeMK1;
        $matkul1 = $json->matkul1;
        $sks1 = $json->sks1;
        $kodeMK2 = $json->kodeMK2;
        $matkul2 = $json->matkul2;
        $sks2 = $json->sks2;
        $kodeMK3 = $json->kodeMK3;
        $matkul3 = $json->matkul3;
        $sks3 = $json->sks3;
        $kodeMK4 = $json->kodeMK4;
        $matkul4 = $json->matkul4;
        $sks4 = $json->sks4;
        $kodeMK5 = $json->kodeMK5;
        $matkul5 = $json->matkul5;
        $sks5 = $json->sks5;
        $kodeMK6 = $json->kodeMK6;
        $matkul6 = $json->matkul6;
        $sks6 = $json->sks6;
        $kodeMK7 = $json->kodeMK7;
        $matkul7 = $json->matkul7;
        $sks7 = $json->sks7;
        $kodeMK8 = $json->kodeMK8;
        $matkul8 = $json->matkul8;
        $sks8 = $json->sks8;
        $formatsurat_id = $request->idFormatSurat;
        $pesananID = $this->pesanansuratRepo->findPesananSuratById($request->id);
        $pemesan = $pesananID->mahasiswa_id;
        $user = Mahasiswa::where('npm',$json->npm)->first();
        return view('TU.proses_surat_perwakilan_perwalian_8mk', [
            'semester' => $semester,
            'thnAkademik' => $thnAkademik,
            'nama' => $nama,
            'prodi' => $prodi,
            'npm' => $npm,
            'namaWakil' => $namaWakil,
            'prodiWakil' => $prodiWakil,
            'npmWakil' => $npmWakil,
            'dosenWali' => $dosenWali,
            'alasan' => $alasan,
            'kodeMK1' => $kodeMK1,
            'matkul1' => $matkul1,
            'sks1' => $sks1,
            'kodeMK2' => $kodeMK2,
            'matkul2' => $matkul2,
            'sks2' => $sks2,
            'kodeMK3' => $kodeMK3,
            'matkul3' => $matkul3,
            'sks3' => $sks3,
            'kodeMK4' => $kodeMK4,
            'matkul4' => $matkul4,
            'sks4' => $sks4,
            'kodeMK5' => $kodeMK5,
            'matkul5' => $matkul5,
            'sks5' => $sks5,
            'kodeMK6' => $kodeMK6,
            'matkul6' => $matkul6,
            'sks6' => $sks6,
            'kodeMK7' => $kodeMK7,
            'matkul7' => $matkul7,
            'sks7' => $sks7,
            'kodeMK8' => $kodeMK8,
            'matkul8' => $matkul8,
            'sks8' => $sks8,
            'formatsurat_id' => $formatsurat_id,
            'dataSurat' => $dataSurat,
            'user' => $user,
            'pemesan' => $pemesan
        ]);
      }
      else if($request->idFormatSurat == "19"){
        $dataSurat = $request->prosesSurat;
        $json = json_decode($dataSurat);
        $semester = $json->semester;
        $thnAkademik = $json->thnAkademik;
        $nama = $json->nama;
        $prodi = $json->prodi;
        $npm = $json->npm;
        $namaWakil = $json->namaWakil;
        $prodiWakil = $json->prodiWakil;
        $npmWakil = $json->npmWakil;
        $dosenWali = $json->dosenWali;
        $alasan = $json->alasan;
        $kodeMK1 = $json->kodeMK1;
        $matkul1 = $json->matkul1;
        $sks1 = $json->sks1;
        $kodeMK2 = $json->kodeMK2;
        $matkul2 = $json->matkul2;
        $sks2 = $json->sks2;
        $kodeMK3 = $json->kodeMK3;
        $matkul3 = $json->matkul3;
        $sks3 = $json->sks3;
        $kodeMK4 = $json->kodeMK4;
        $matkul4 = $json->matkul4;
        $sks4 = $json->sks4;
        $kodeMK5 = $json->kodeMK5;
        $matkul5 = $json->matkul5;
        $sks5 = $json->sks5;
        $kodeMK6 = $json->kodeMK6;
        $matkul6 = $json->matkul6;
        $sks6 = $json->sks6;
        $kodeMK7 = $json->kodeMK7;
        $matkul7 = $json->matkul7;
        $sks7 = $json->sks7;
        $kodeMK8 = $json->kodeMK8;
        $matkul8 = $json->matkul8;
        $sks8 = $json->sks8;
        $kodeMK9 = $json->kodeMK9;
        $matkul9 = $json->matkul9;
        $sks9 = $json->sks9;
        $formatsurat_id = $request->idFormatSurat;
        $pesananID = $this->pesanansuratRepo->findPesananSuratById($request->id);
        $pemesan = $pesananID->mahasiswa_id;
        $user = Mahasiswa::where('npm',$json->npm)->first();
        return view('TU.proses_surat_perwakilan_perwalian_9mk', [
            'semester' => $semester,
            'thnAkademik' => $thnAkademik,
            'nama' => $nama,
            'prodi' => $prodi,
            'npm' => $npm,
            'namaWakil' => $namaWakil,
            'prodiWakil' => $prodiWakil,
            'npmWakil' => $npmWakil,
            'dosenWali' => $dosenWali,
            'alasan' => $alasan,
            'kodeMK1' => $kodeMK1,
            'matkul1' => $matkul1,
            'sks1' => $sks1,
            'kodeMK2' => $kodeMK2,
            'matkul2' => $matkul2,
            'sks2' => $sks2,
            'kodeMK3' => $kodeMK3,
            'matkul3' => $matkul3,
            'sks3' => $sks3,
            'kodeMK4' => $kodeMK4,
            'matkul4' => $matkul4,
            'sks4' => $sks4,
            'kodeMK5' => $kodeMK5,
            'matkul5' => $matkul5,
            'sks5' => $sks5,
            'kodeMK6' => $kodeMK6,
            'matkul6' => $matkul6,
            'sks6' => $sks6,
            'kodeMK7' => $kodeMK7,
            'matkul7' => $matkul7,
            'sks7' => $sks7,
            'kodeMK8' => $kodeMK8,
            'matkul8' => $matkul8,
            'sks8' => $sks8,
            'kodeMK9' => $kodeMK9,
            'matkul9' => $matkul9,
            'sks9' => $sks9,
            'formatsurat_id' => $formatsurat_id,
            'dataSurat' => $dataSurat,
            'user' => $user,
            'pemesan' => $pemesan
        ]);
      }
      else if($request->idFormatSurat == "20"){
        $dataSurat = $request->prosesSurat;
        $json = json_decode($dataSurat);
        $semester = $json->semester;
        $thnAkademik = $json->thnAkademik;
        $nama = $json->nama;
        $prodi = $json->prodi;
        $npm = $json->npm;
        $namaWakil = $json->namaWakil;
        $prodiWakil = $json->prodiWakil;
        $npmWakil = $json->npmWakil;
        $dosenWali = $json->dosenWali;
        $alasan = $json->alasan;
        $kodeMK1 = $json->kodeMK1;
        $matkul1 = $json->matkul1;
        $sks1 = $json->sks1;
        $kodeMK2 = $json->kodeMK2;
        $matkul2 = $json->matkul2;
        $sks2 = $json->sks2;
        $kodeMK3 = $json->kodeMK3;
        $matkul3 = $json->matkul3;
        $sks3 = $json->sks3;
        $kodeMK4 = $json->kodeMK4;
        $matkul4 = $json->matkul4;
        $sks4 = $json->sks4;
        $kodeMK5 = $json->kodeMK5;
        $matkul5 = $json->matkul5;
        $sks5 = $json->sks5;
        $kodeMK6 = $json->kodeMK6;
        $matkul6 = $json->matkul6;
        $sks6 = $json->sks6;
        $kodeMK7 = $json->kodeMK7;
        $matkul7 = $json->matkul7;
        $sks7 = $json->sks7;
        $kodeMK8 = $json->kodeMK8;
        $matkul8 = $json->matkul8;
        $sks8 = $json->sks8;
        $kodeMK9 = $json->kodeMK9;
        $matkul9 = $json->matkul9;
        $sks9 = $json->sks9;
        $kodeMK10 = $json->kodeMK10;
        $matkul10 = $json->matkul10;
        $sks10 = $json->sks10;
        $formatsurat_id = $request->idFormatSurat;
        $pesananID = $this->pesanansuratRepo->findPesananSuratById($request->id);
        $pemesan = $pesananID->mahasiswa_id;
        $user = Mahasiswa::where('npm',$json->npm)->first();
        return view('TU.proses_surat_perwakilan_perwalian_10mk', [
            'semester' => $semester,
            'thnAkademik' => $thnAkademik,
            'nama' => $nama,
            'prodi' => $prodi,
            'npm' => $npm,
            'namaWakil' => $namaWakil,
            'prodiWakil' => $prodiWakil,
            'npmWakil' => $npmWakil,
            'dosenWali' => $dosenWali,
            'alasan' => $alasan,
            'kodeMK1' => $kodeMK1,
            'matkul1' => $matkul1,
            'sks1' => $sks1,
            'kodeMK2' => $kodeMK2,
            'matkul2' => $matkul2,
            'sks2' => $sks2,
            'kodeMK3' => $kodeMK3,
            'matkul3' => $matkul3,
            'sks3' => $sks3,
            'kodeMK4' => $kodeMK4,
            'matkul4' => $matkul4,
            'sks4' => $sks4,
            'kodeMK5' => $kodeMK5,
            'matkul5' => $matkul5,
            'sks5' => $sks5,
            'kodeMK6' => $kodeMK6,
            'matkul6' => $matkul6,
            'sks6' => $sks6,
            'kodeMK7' => $kodeMK7,
            'matkul7' => $matkul7,
            'sks7' => $sks7,
            'kodeMK8' => $kodeMK8,
            'matkul8' => $matkul8,
            'sks8' => $sks8,
            'kodeMK9' => $kodeMK9,
            'matkul9' => $matkul9,
            'sks9' => $sks9,
            'kodeMK10' => $kodeMK10,
            'matkul10' => $matkul10,
            'sks10' => $sks10,
            'formatsurat_id' => $formatsurat_id,
            'dataSurat' => $dataSurat,
            'user' => $user,
            'pemesan' => $pemesan
        ]);
      }
    }
    public function store(Request $request){
      $loggedInUser = Auth::user();
      // dd($request->idFormat);
      // dd("DI LINE 1452 PesananSuratController.php | "+$request->idPesanansurat);
      $realUser = $this->getRealUser($loggedInUser);
        if($request->idFormat == "1"){
          $pesanansurat = new PesananSurat;
          $pesanansurat->perihal = '-';
          $pesanansurat->mahasiswa_id = $realUser->id;
          $pesanansurat->formatsurat_id = $request->idFormat;
          $pesanansurat->penerimaSurat = $request->provider;
          $pesanansurat->dataSurat = $request->dataSurat;
          $pesanansurat->persetujuanDosenWali = true ;
          $pesanansurat->persetujuanKaprodi = true;
          $pesanansurat->persetujuanWDII = true;
          $pesanansurat->persetujuanWDI = true;
          $pesanansurat->persetujuanDekan = true;
          $pesanansurat->save();
        }
        else if($request->idFormat == "2"){
          $pesanansurat = new PesananSurat;
          $pesanansurat->perihal = '-';
          $pesanansurat->mahasiswa_id = $realUser->id;
          $pesanansurat->formatsurat_id = $request->idFormat;
          $pesanansurat->penerimaSurat = '-';;
          $pesanansurat->dataSurat = $request->dataSurat;
          $pesanansurat->persetujuanDosenWali = true ;
          $pesanansurat->persetujuanKaprodi = true;
          $pesanansurat->persetujuanWDII = true;
          $pesanansurat->persetujuanWDI = true;
          $pesanansurat->persetujuanDekan = true;
          $pesanansurat->save();
        }
        else if($request->idFormat == "3"){
          $pesanansurat = new PesananSurat;
          $pesanansurat->perihal = '-';
          $pesanansurat->mahasiswa_id = $realUser->id;
          $pesanansurat->formatsurat_id = $request->idFormat;
          $pesanansurat->penerimaSurat = $request->organisasiTujuan;
          $pesanansurat->dataSurat = $request->dataSurat;
          $pesanansurat->persetujuanDosenWali = true ;
          $pesanansurat->persetujuanKaprodi = true;
          $pesanansurat->persetujuanWDII = true;
          $pesanansurat->persetujuanWDI = true;
          $pesanansurat->persetujuanDekan = true;
          $pesanansurat->save();
        }
        else if($request->idFormat == "4"){
          $pesanansurat = new PesananSurat;
          $pesanansurat->perihal = '-';
          $pesanansurat->mahasiswa_id = $realUser->id;
          $pesanansurat->formatsurat_id = $request->idFormat;
          $pesanansurat->penerimaSurat = $request->organisasiTujuan;
          $pesanansurat->dataSurat = $request->dataSurat;
          $pesanansurat->persetujuanDosenWali = true ;
          $pesanansurat->persetujuanKaprodi = true;
          $pesanansurat->persetujuanWDII = true;
          $pesanansurat->persetujuanWDI = true;
          $pesanansurat->persetujuanDekan = true;
          $pesanansurat->save();
        }
        else if($request->idFormat == "5"){
          $pesanansurat = new PesananSurat;
          $pesanansurat->perihal = '-';
          $pesanansurat->mahasiswa_id = $realUser->id;
          $pesanansurat->formatsurat_id = $request->idFormat;
          $pesanansurat->penerimaSurat = $request->organisasiTujuan;
          $pesanansurat->dataSurat = $request->dataSurat;
          $pesanansurat->persetujuanDosenWali = true ;
          $pesanansurat->persetujuanKaprodi = true;
          $pesanansurat->persetujuanWDII = true;
          $pesanansurat->persetujuanWDI = true;
          $pesanansurat->persetujuanDekan = true;
          $pesanansurat->save();
        }
        else if($request->idFormat == "6"){
          $pesanansurat = new PesananSurat;
          $pesanansurat->perihal = '-';
          $pesanansurat->mahasiswa_id = $realUser->id;
          $pesanansurat->formatsurat_id = $request->idFormat;
          $pesanansurat->penerimaSurat = $request->organisasiTujuan;
          $pesanansurat->dataSurat = $request->dataSurat;
          $pesanansurat->persetujuanDosenWali = true ;
          $pesanansurat->persetujuanKaprodi = true;
          $pesanansurat->persetujuanWDII = true;
          $pesanansurat->persetujuanWDI = true;
          $pesanansurat->persetujuanDekan = true;
          $pesanansurat->save();
        }
        else if($request->idFormat == "7"){
          $pesanansurat = new PesananSurat;
          $pesanansurat->perihal = '-';
          $pesanansurat->mahasiswa_id = $realUser->id;
          $pesanansurat->formatsurat_id = $request->idFormat;
          $pesanansurat->penerimaSurat = $request->organisasiTujuan;
          $pesanansurat->dataSurat = $request->dataSurat;
          $pesanansurat->persetujuanDosenWali = true ;
          $pesanansurat->persetujuanKaprodi = true;
          $pesanansurat->persetujuanWDII = true;
          $pesanansurat->persetujuanWDI = true;
          $pesanansurat->persetujuanDekan = true;
          $pesanansurat->save();
        }
        else if($request->idFormat == "8"){
          $pesanansurat = new PesananSurat;
          $pesanansurat->perihal = '-';
          $pesanansurat->mahasiswa_id = $realUser->id;
          $pesanansurat->formatsurat_id = $request->idFormat;
          $pesanansurat->penerimaSurat = $request->organisasiTujuan;
          $pesanansurat->dataSurat = $request->dataSurat;
          $pesanansurat->persetujuanDosenWali = true ;
          $pesanansurat->persetujuanKaprodi = true;
          $pesanansurat->persetujuanWDII = true;
          $pesanansurat->persetujuanWDI = true;
          $pesanansurat->persetujuanDekan = true;
          $pesanansurat->save();
        }
        else if($request->idFormat == "9"){
          $pesanansurat = new PesananSurat;
          $pesanansurat->perihal = 'Surat Ijin Berhenti Studi Sementara';
          $pesanansurat->mahasiswa_id = $realUser->id;
          $pesanansurat->formatsurat_id = $request->idFormat;
          $pesanansurat->penerimaSurat = $realUser->nama_mahasiswa;
          // dd($realUser);
          $pesanansurat->dataSurat = $request->dataSurat;
          $pesanansurat->persetujuanDosenWali = false ;
          $pesanansurat->persetujuanKaprodi = false;
          $pesanansurat->persetujuanWDII = false;
          $pesanansurat->persetujuanWDI = false;
          $pesanansurat->persetujuanDekan = false;
          $pesanansurat->save();
        }
        else if($request->idFormat == "10"){
          $pesanansurat = new PesananSurat;
          $pesanansurat->perihal = 'Pengunduran Diri';
          $pesanansurat->mahasiswa_id = $realUser->id;
          $pesanansurat->formatsurat_id = $request->idFormat;
          $pesanansurat->penerimaSurat = 'Rektor';
          // $pesanansurat->dataSurat = $request->data;
          $pesanansurat->dataSurat = $request->dataSurat;
          $pesanansurat->persetujuanDosenWali = false ;
          $pesanansurat->persetujuanKaprodi = false;
          $pesanansurat->persetujuanWDII = false;
          $pesanansurat->persetujuanWDI = false;
          $pesanansurat->persetujuanDekan = false;
          $pesanansurat->save();
        }
        else if($request->idFormat == "11"){
          $pesanansurat = new PesananSurat;
          $pesanansurat->perihal = '-';
          $pesanansurat->mahasiswa_id = $realUser->id;
          $pesanansurat->formatsurat_id = $request->idFormat;
          $pesanansurat->penerimaSurat = $request->dosen->nama_dosen;
          // $pesanansurat->dataSurat = $request->data;
          $pesanansurat->dataSurat = $request->dataSurat;
          $pesanansurat->persetujuanDosenWali = true ;
          $pesanansurat->persetujuanKaprodi = true;
          $pesanansurat->persetujuanWDII = true;
          $pesanansurat->persetujuanWDI = true;
          $pesanansurat->persetujuanDekan = true;
          $pesanansurat->save();
        }
        else if($request->idFormat == "12"){
          $pesanansurat = new PesananSurat;
          $pesanansurat->perihal = '-';
          $pesanansurat->mahasiswa_id = $realUser->id;
          $pesanansurat->formatsurat_id = $request->idFormat;
          $pesanansurat->penerimaSurat = $request->dosen->nama_dosen;
          $pesanansurat->dataSurat = $request->dataSurat;
          $pesanansurat->persetujuanDosenWali = true ;
          $pesanansurat->persetujuanKaprodi = true;
          $pesanansurat->persetujuanWDII = true;
          $pesanansurat->persetujuanWDI = true;
          $pesanansurat->persetujuanDekan = true;
          $pesanansurat->save();
        }
        else if($request->idFormat == "13"){
          $pesanansurat = new PesananSurat;
          $pesanansurat->perihal = '-';
          $pesanansurat->mahasiswa_id = $realUser->id;
          $pesanansurat->formatsurat_id = $request->idFormat;
          $pesanansurat->penerimaSurat = $request->dosen->nama_dosen;
          $pesanansurat->dataSurat = $request->dataSurat;
          $pesanansurat->persetujuanDosenWali = true ;
          $pesanansurat->persetujuanKaprodi = true;
          $pesanansurat->persetujuanWDII = true;
          $pesanansurat->persetujuanWDI = true;
          $pesanansurat->persetujuanDekan = true;
          $pesanansurat->save();
        }
        else if($request->idFormat == "14"){
          $pesanansurat = new PesananSurat;
          $pesanansurat->perihal = '-';
          $pesanansurat->mahasiswa_id = $realUser->id;
          $pesanansurat->formatsurat_id = $request->idFormat;
          $pesanansurat->penerimaSurat = $request->dosen->nama_dosen;
          $pesanansurat->dataSurat = $request->dataSurat;
          $pesanansurat->persetujuanDosenWali = true ;
          $pesanansurat->persetujuanKaprodi = true;
          $pesanansurat->persetujuanWDII = true;
          $pesanansurat->persetujuanWDI = true;
          $pesanansurat->persetujuanDekan = true;
          $pesanansurat->save();
        }
        else if($request->idFormat == "15"){
          $pesanansurat = new PesananSurat;
          $pesanansurat->perihal = '-';
          $pesanansurat->mahasiswa_id = $realUser->id;
          $pesanansurat->formatsurat_id = $request->idFormat;
          $pesanansurat->penerimaSurat = $request->dosen->nama_dosen;
          $pesanansurat->dataSurat = $request->dataSurat;
          $pesanansurat->persetujuanDosenWali = true ;
          $pesanansurat->persetujuanKaprodi = true;
          $pesanansurat->persetujuanWDII = true;
          $pesanansurat->persetujuanWDI = true;
          $pesanansurat->persetujuanDekan = true;
          $pesanansurat->save();
        }
        else if($request->idFormat == "16"){
          $pesanansurat = new PesananSurat;
          $pesanansurat->perihal = '-';
          $pesanansurat->mahasiswa_id = $realUser->id;
          $pesanansurat->formatsurat_id = $request->idFormat;
          $pesanansurat->penerimaSurat = $request->dosen->nama_dosen;
          $pesanansurat->dataSurat = $request->dataSurat;
          $pesanansurat->persetujuanDosenWali = true ;
          $pesanansurat->persetujuanKaprodi = true;
          $pesanansurat->persetujuanWDII = true;
          $pesanansurat->persetujuanWDI = true;
          $pesanansurat->persetujuanDekan = true;
          $pesanansurat->save();
        }
        else if($request->idFormat == "17"){
          $pesanansurat = new PesananSurat;
          $pesanansurat->perihal = '-';
          $pesanansurat->mahasiswa_id = $realUser->id;
          $pesanansurat->formatsurat_id = $request->idFormat;
          $pesanansurat->penerimaSurat = $request->dosen->nama_dosen;
          $pesanansurat->dataSurat = $request->dataSurat;
          $pesanansurat->persetujuanDosenWali = true ;
          $pesanansurat->persetujuanKaprodi = true;
          $pesanansurat->persetujuanWDII = true;
          $pesanansurat->persetujuanWDI = true;
          $pesanansurat->persetujuanDekan = true;
          $pesanansurat->save();
        }
        else if($request->idFormat == "18"){
          $pesanansurat = new PesananSurat;
          $pesanansurat->perihal = '-';
          $pesanansurat->mahasiswa_id = $realUser->id;
          $pesanansurat->formatsurat_id = $request->idFormat;
          $pesanansurat->penerimaSurat = $request->dosen->nama_dosen;
          $pesanansurat->dataSurat = $request->dataSurat;
          $pesanansurat->persetujuanDosenWali = true ;
          $pesanansurat->persetujuanKaprodi = true;
          $pesanansurat->persetujuanWDII = true;
          $pesanansurat->persetujuanWDI = true;
          $pesanansurat->persetujuanDekan = true;
          $pesanansurat->save();
        }
        else if($request->idFormat == "19"){
          $pesanansurat = new PesananSurat;
          $pesanansurat->perihal = '-';
          $pesanansurat->mahasiswa_id = $realUser->id;
          $pesanansurat->formatsurat_id = $request->idFormat;
          $pesanansurat->penerimaSurat = $request->dosen->nama_dosen;
          $pesanansurat->dataSurat = $request->dataSurat;
          $pesanansurat->persetujuanDosenWali = true ;
          $pesanansurat->persetujuanKaprodi = true;
          $pesanansurat->persetujuanWDII = true;
          $pesanansurat->persetujuanWDI = true;
          $pesanansurat->persetujuanDekan = true;
          $pesanansurat->save();
        }
        else if($request->idFormat == "20"){
          $pesanansurat = new PesananSurat;
          $pesanansurat->perihal = '-';
          $pesanansurat->mahasiswa_id = $realUser->id;
          $pesanansurat->formatsurat_id = $request->idFormat;
          $pesanansurat->penerimaSurat = $request->dosen->nama_dosen;
          $pesanansurat->dataSurat = $request->dataSurat;
          $pesanansurat->persetujuanDosenWali = true ;
          $pesanansurat->persetujuanKaprodi = true;
          $pesanansurat->persetujuanWDII = true;
          $pesanansurat->persetujuanWDI = true;
          $pesanansurat->persetujuanDekan = true;
          $pesanansurat->save();
        }
        return redirect('/home_mahasiswa')->with('success_message', 'Surat berhasil dibuat');
    }

    /**
    * Untuk meng-generate JSON dari data input
    */
    private function buatJSON($request){
      $obj = "";
      if($request->jenis_surat == "1"){
          $obj = [
            'nama' => $request->nama,
            'prodi' => $request->prodi,
            'npm' => $request->npm,
            'semester' => $request->semester,
            'thnAkademik' => $request->thnAkademik,
            'penyediabeasiswa' => $request->penyediabeasiswa,
          ];
      }
      else if($request->jenis_surat == "2"){
          $obj = [
            'nama' => $request->nama,
            'prodi' => $request->prodi,
            'npm' => $request->npm,
            'kota_lahir' => $request->kota_lahir,
            'tglLahir' => $request->tglLahir,
            'alamat' => $request->alamat,
            'semester' => $request->semester,
          ];
      }
      else if($request->jenis_surat == "3"){
          $obj = [
            'nama' => $request->nama,
            'tglLahir' => $request->tglLahir,
            'kewarganegaraan' => $request->kewarganegaraan,
            'organisasiTujuan' => $request->organisasiTujuan,
            'thnAkademik' => $request->thnAkademik,
            'negaraTujuan' => $request->negaraTujuan,
            'tanggalKunjungan' => $request->tanggalKunjungan
          ];
      }
      else if($request->jenis_surat == "4"){
          $obj = [
            'nama' => $request->nama,
            'npm' => $request->npm,
            'prodi' => $request->prodi,
            'matkul' => $request->matkul,
            'topik' => $request->topik,
            'organisasi' => $request->organisasi,
            'alamatOrganisasi' => $request->alamatOrganisasi,
            'keperluanKunjungan' => $request->keperluanKunjungan,
            'kota' => $request->kota,
            'kepada' => $request->kepada
          ];
      }
      else if($request->jenis_surat == "5"){
        $obj = [
          'nama' => $request->nama,
          'npm' => $request->npm,
          'prodi' => $request->prodi,
          'matkul' => $request->matkul,
          'topik' => $request->topik,
          'organisasi' => $request->organisasi,
          'alamatOrganisasi' => $request->alamatOrganisasi,
          'keperluanKunjungan' => $request->keperluanKunjungan,
          'kota' => $request->kota,
          'kepada' => $request->kepada,
          'namaAnggota' => $request->namaAnggota,
          'npmAnggota' => $request->npmAnggota,
        ];
      }
      else if($request->jenis_surat == "6"){
        $obj = [
          'nama' => $request->nama,
          'npm' => $request->npm,
          'prodi' => $request->prodi,
          'matkul' => $request->matkul,
          'topik' => $request->topik,
          'organisasi' => $request->organisasi,
          'alamatOrganisasi' => $request->alamatOrganisasi,
          'keperluanKunjungan' => $request->keperluanKunjungan,
          'kota' => $request->kota,
          'kepada' => $request->kepada,
          'namaAnggota1' => $request->namaAnggota1,
          'npmAnggota1' => $request->npmAnggota1,
          'namaAnggota2' => $request->namaAnggota2,
          'npmAnggota2' => $request->npmAnggota2
        ];
      }
      else if($request->jenis_surat == "7"){
        $obj = [
          'nama' => $request->nama,
          'npm' => $request->npm,
          'prodi' => $request->prodi,
          'matkul' => $request->matkul,
          'topik' => $request->topik,
          'organisasi' => $request->organisasi,
          'alamatOrganisasi' => $request->alamatOrganisasi,
          'keperluanKunjungan' => $request->keperluanKunjungan,
          'kota' => $request->kota,
          'kepada' => $request->kepada,
          'namaAnggota1' => $request->namaAnggota1,
          'npmAnggota1' => $request->npmAnggota1,
          'namaAnggota2' => $request->namaAnggota2,
          'npmAnggota2' => $request->npmAnggota2,
          'namaAnggota3' => $request->namaAnggota3,
          'npmAnggota3' => $request->npmAnggota3
        ];
      }
      else if($request->jenis_surat == "8"){
        $obj = [
          'nama' => $request->nama,
          'npm' => $request->npm,
          'prodi' => $request->prodi,
          'matkul' => $request->matkul,
          'topik' => $request->topik,
          'organisasi' => $request->organisasi,
          'alamatOrganisasi' => $request->alamatOrganisasi,
          'keperluanKunjungan' => $request->keperluanKunjungan,
          'kota' => $request->kota,
          'kepada' => $request->kepada,
          'namaAnggota1' => $request->namaAnggota1,
          'npmAnggota1' => $request->npmAnggota1,
          'namaAnggota2' => $request->namaAnggota2,
          'npmAnggota2' => $request->npmAnggota2,
          'namaAnggota3' => $request->namaAnggota3,
          'npmAnggota3' => $request->npmAnggota3,
          'namaAnggota4' => $request->namaAnggota4,
          'npmAnggota4' => $request->npmAnggota4
        ];
      }
      else if($request->jenis_surat == "9"){
        $obj = [
          'nama' => $request->nama,
          'npm' => $request->npm,
          'prodi' => $request->prodi,
          'fakultas' => $request->fakultas,
          'alamat' => $request->alamat,
          'cutiStudiKe' => $request->cutiStudiKe,
          'alasanCutiStudi' => $request->alasanCutiStudi,
          'dosenWali' => $request->dosenWali,
          'semester' => $request->semester,
          'thnAkademik' => $request->thnAkademik,
          'persetujuanDosenWali' => $request->persetujuanDosenWali,
          'catatanDosenWali' => $request->catatanDosenWali,
          'persetujuanKaprodi' => $request->persetujuanKaprodi,
          'catatanKaprodi' => $request->catatanKaprodi,
          'persetujuanWDII' => $request->persetujuanWDII,
          'catatanWDII' => $request->catatanWDII,
          'persetujuanWDI' => $request->persetujuanWDI,
          'catatanWDI' => $request->catatanWDI,
          'persetujuanDekan' => $request->persetujuanDekan
        ];
      }
      else if($request->jenis_surat == "10"){
        $obj = [
          'nama' => $request->nama,
          'npm' => $request->npm,
          'alamat' => $request->alamat,
          'noTelepon' => $request->noTelepon,
          'namaOrtu' => $request->namaOrtu,
          'dosenWali' => $request->dosenWali,
          'semester' => $request->semester,
          'persetujuanDosenWali' => $request->persetujuanDosenWali,
          'catatanDosenWali' => $request->catatanDosenWali,
          'persetujuanKaprodi' => $request->persetujuanKaprodi,
          'catatanKaprodi' => $request->catatanKaprodi,
          'persetujuanWDII' => $request->persetujuanWDII,
          'catatanWDII' => $request->catatanWDII,
          'persetujuanWDI' => $request->persetujuanWDI,
          'catatanWDI' => $request->catatanWDI,
          'persetujuanDekan' => $request->persetujuanDekan
        ];
      }
      else if($request->jenis_surat == "11"){
        $obj = [
          'semester' => $request->semester,
          'thnAkademik' => $request->thnAkademik,
          'nama' => $request->nama,
          'prodi' => $request->prodi,
          'npm' => $request->npm,
          'namaWakil' => $request->namaWakil,
          'prodiWakil' => $request->prodiWakil,
          'npmWakil' => $request->npmWakil,
          'dosenWali' => $request->dosenWali,
          'alasan' => $request->alasan,
          'kodeMK' => $request->kodeMK,
          'matkul' => $request->matkul,
          'sks' => $request->sks,
          'formatsurat_id' => $request->formatsurat_id,
          'dataSurat' => $request->dataSurat
        ];
      }
      else if($request->jenis_surat == "12"){
        $obj = [
          'semester' => $request->semester,
          'thnAkademik' => $request->thnAkademik,
          'nama' => $request->nama,
          'prodi' => $request->prodi,
          'npm' => $request->npm,
          'namaWakil' => $request->namaWakil,
          'prodiWakil' => $request->prodiWakil,
          'npmWakil' => $request->npmWakil,
          'dosenWali' => $request->dosenWali,
          'alasan' => $request->alasan,
          'kodeMK1' => $request->kodeMK1,
          'matkul1' => $request->matkul1,
          'sks1' => $request->sks1,
          'kodeMK2' => $request->kodeMK2,
          'matkul2' => $request->matkul2,
          'sks2' => $request->sks2,
          'formatsurat_id' => $request->formatsurat_id,
          'dataSurat' => $request->dataSurat
        ];
      }
      else if($request->jenis_surat == "13"){
        $obj = [
          'semester' => $request->semester,
          'thnAkademik' => $request->thnAkademik,
          'nama' => $request->nama,
          'prodi' => $request->prodi,
          'npm' => $request->npm,
          'namaWakil' => $request->namaWakil,
          'prodiWakil' => $request->prodiWakil,
          'npmWakil' => $request->npmWakil,
          'dosenWali' => $request->dosenWali,
          'alasan' => $request->alasan,
          'kodeMK1' => $request->kodeMK1,
          'matkul1' => $request->matkul1,
          'sks1' => $request->sks1,
          'kodeMK2' => $request->kodeMK2,
          'matkul2' => $request->matkul2,
          'sks2' => $request->sks2,
          'kodeMK3' => $request->kodeMK3,
          'matkul3' => $request->matkul3,
          'sks3' => $request->sks3,
          'formatsurat_id' => $request->formatsurat_id,
          'dataSurat' => $request->dataSurat
        ];
      }
      else if($request->jenis_surat == "14"){
        $obj = [
          'semester' => $request->semester,
          'thnAkademik' => $request->thnAkademik,
          'nama' => $request->nama,
          'prodi' => $request->prodi,
          'npm' => $request->npm,
          'namaWakil' => $request->namaWakil,
          'prodiWakil' => $request->prodiWakil,
          'npmWakil' => $request->npmWakil,
          'dosenWali' => $request->dosenWali,
          'alasan' => $request->alasan,
          'kodeMK1' => $request->kodeMK1,
          'matkul1' => $request->matkul1,
          'sks1' => $request->sks1,
          'kodeMK2' => $request->kodeMK2,
          'matkul2' => $request->matkul2,
          'sks2' => $request->sks2,
          'kodeMK3' => $request->kodeMK3,
          'matkul3' => $request->matkul3,
          'sks3' => $request->sks3,
          'kodeMK4' => $request->kodeMK4,
          'matkul4' => $request->matkul4,
          'sks4' => $request->sks4,
          'formatsurat_id' => $request->formatsurat_id,
          'dataSurat' => $request->dataSurat
        ];
      }
      else if($request->jenis_surat == "15"){
        $obj = [
          'semester' => $request->semester,
          'thnAkademik' => $request->thnAkademik,
          'nama' => $request->nama,
          'prodi' => $request->prodi,
          'npm' => $request->npm,
          'namaWakil' => $request->namaWakil,
          'prodiWakil' => $request->prodiWakil,
          'npmWakil' => $request->npmWakil,
          'dosenWali' => $request->dosenWali,
          'alasan' => $request->alasan,
          'kodeMK1' => $request->kodeMK1,
          'matkul1' => $request->matkul1,
          'sks1' => $request->sks1,
          'kodeMK2' => $request->kodeMK2,
          'matkul2' => $request->matkul2,
          'sks2' => $request->sks2,
          'kodeMK3' => $request->kodeMK3,
          'matkul3' => $request->matkul3,
          'sks3' => $request->sks3,
          'kodeMK4' => $request->kodeMK4,
          'matkul4' => $request->matkul4,
          'sks4' => $request->sks4,
          'kodeMK5' => $request->kodeMK5,
          'matkul5' => $request->matkul5,
          'sks5' => $request->sks5,
          'formatsurat_id' => $request->formatsurat_id,
          'dataSurat' => $request->dataSurat
        ];
      }
      else if($request->jenis_surat == "16"){
        $obj = [
          'semester' => $request->semester,
          'thnAkademik' => $request->thnAkademik,
          'nama' => $request->nama,
          'prodi' => $request->prodi,
          'npm' => $request->npm,
          'namaWakil' => $request->namaWakil,
          'prodiWakil' => $request->prodiWakil,
          'npmWakil' => $request->npmWakil,
          'dosenWali' => $request->dosenWali,
          'alasan' => $request->alasan,
          'kodeMK1' => $request->kodeMK1,
          'matkul1' => $request->matkul1,
          'sks1' => $request->sks1,
          'kodeMK2' => $request->kodeMK2,
          'matkul2' => $request->matkul2,
          'sks2' => $request->sks2,
          'kodeMK3' => $request->kodeMK3,
          'matkul3' => $request->matkul3,
          'sks3' => $request->sks3,
          'kodeMK4' => $request->kodeMK4,
          'matkul4' => $request->matkul4,
          'sks4' => $request->sks4,
          'kodeMK5' => $request->kodeMK5,
          'matkul5' => $request->matkul5,
          'sks5' => $request->sks5,
          'kodeMK6' => $request->kodeMK6,
          'matkul6' => $request->matkul6,
          'sks6' => $request->sks6,
          'formatsurat_id' => $request->formatsurat_id,
          'dataSurat' => $request->dataSurat
        ];
      }
      else if($request->jenis_surat == "17"){
        $obj = [
          'semester' => $request->semester,
          'thnAkademik' => $request->thnAkademik,
          'nama' => $request->nama,
          'prodi' => $request->prodi,
          'npm' => $request->npm,
          'namaWakil' => $request->namaWakil,
          'prodiWakil' => $request->prodiWakil,
          'npmWakil' => $request->npmWakil,
          'dosenWali' => $request->dosenWali,
          'alasan' => $request->alasan,
          'kodeMK1' => $request->kodeMK1,
          'matkul1' => $request->matkul1,
          'sks1' => $request->sks1,
          'kodeMK2' => $request->kodeMK2,
          'matkul2' => $request->matkul2,
          'sks2' => $request->sks2,
          'kodeMK3' => $request->kodeMK3,
          'matkul3' => $request->matkul3,
          'sks3' => $request->sks3,
          'kodeMK4' => $request->kodeMK4,
          'matkul4' => $request->matkul4,
          'sks4' => $request->sks4,
          'kodeMK5' => $request->kodeMK5,
          'matkul5' => $request->matkul5,
          'sks5' => $request->sks5,
          'kodeMK6' => $request->kodeMK6,
          'matkul6' => $request->matkul6,
          'sks6' => $request->sks6,
          'kodeMK7' => $request->kodeMK7,
          'matkul7' => $request->matkul7,
          'sks7' => $request->sks7,
          'formatsurat_id' => $request->formatsurat_id,
          'dataSurat' => $request->dataSurat
        ];
      }
      else if($request->jenis_surat == "18"){
        $obj = [
          'semester' => $request->semester,
          'thnAkademik' => $request->thnAkademik,
          'nama' => $request->nama,
          'prodi' => $request->prodi,
          'npm' => $request->npm,
          'namaWakil' => $request->namaWakil,
          'prodiWakil' => $request->prodiWakil,
          'npmWakil' => $request->npmWakil,
          'dosenWali' => $request->dosenWali,
          'alasan' => $request->alasan,
          'kodeMK1' => $request->kodeMK1,
          'matkul1' => $request->matkul1,
          'sks1' => $request->sks1,
          'kodeMK2' => $request->kodeMK2,
          'matkul2' => $request->matkul2,
          'sks2' => $request->sks2,
          'kodeMK3' => $request->kodeMK3,
          'matkul3' => $request->matkul3,
          'sks3' => $request->sks3,
          'kodeMK4' => $request->kodeMK4,
          'matkul4' => $request->matkul4,
          'sks4' => $request->sks4,
          'kodeMK5' => $request->kodeMK5,
          'matkul5' => $request->matkul5,
          'sks5' => $request->sks5,
          'kodeMK6' => $request->kodeMK6,
          'matkul6' => $request->matkul6,
          'sks6' => $request->sks6,
          'kodeMK7' => $request->kodeMK7,
          'matkul7' => $request->matkul7,
          'sks7' => $request->sks7,
          'kodeMK8' => $request->kodeMK8,
          'matkul8' => $request->matkul8,
          'sks8' => $request->sks8,
          'formatsurat_id' => $request->formatsurat_id,
          'dataSurat' => $request->dataSurat
        ];
      }
      else if($request->jenis_surat == "19"){
        $obj = [
          'semester' => $request->semester,
          'thnAkademik' => $request->thnAkademik,
          'nama' => $request->nama,
          'prodi' => $request->prodi,
          'npm' => $request->npm,
          'namaWakil' => $request->namaWakil,
          'prodiWakil' => $request->prodiWakil,
          'npmWakil' => $request->npmWakil,
          'dosenWali' => $request->dosenWali,
          'alasan' => $request->alasan,
          'kodeMK1' => $request->kodeMK1,
          'matkul1' => $request->matkul1,
          'sks1' => $request->sks1,
          'kodeMK2' => $request->kodeMK2,
          'matkul2' => $request->matkul2,
          'sks2' => $request->sks2,
          'kodeMK3' => $request->kodeMK3,
          'matkul3' => $request->matkul3,
          'sks3' => $request->sks3,
          'kodeMK4' => $request->kodeMK4,
          'matkul4' => $request->matkul4,
          'sks4' => $request->sks4,
          'kodeMK5' => $request->kodeMK5,
          'matkul5' => $request->matkul5,
          'sks5' => $request->sks5,
          'kodeMK6' => $request->kodeMK6,
          'matkul6' => $request->matkul6,
          'sks6' => $request->sks6,
          'kodeMK7' => $request->kodeMK7,
          'matkul7' => $request->matkul7,
          'sks7' => $request->sks7,
          'kodeMK8' => $request->kodeMK8,
          'matkul8' => $request->matkul8,
          'sks8' => $request->sks8,
          'kodeMK9' => $request->kodeMK9,
          'matkul9' => $request->matkul9,
          'sks9' => $request->sks9,
          'formatsurat_id' => $request->formatsurat_id,
          'dataSurat' => $request->dataSurat
        ];
      }
      else if($request->jenis_surat == "20"){
        $obj = [
          'semester' => $request->semester,
          'thnAkademik' => $request->thnAkademik,
          'nama' => $request->nama,
          'prodi' => $request->prodi,
          'npm' => $request->npm,
          'namaWakil' => $request->namaWakil,
          'prodiWakil' => $request->prodiWakil,
          'npmWakil' => $request->npmWakil,
          'dosenWali' => $request->dosenWali,
          'alasan' => $request->alasan,
          'kodeMK1' => $request->kodeMK1,
          'matkul1' => $request->matkul1,
          'sks1' => $request->sks1,
          'kodeMK2' => $request->kodeMK2,
          'matkul2' => $request->matkul2,
          'sks2' => $request->sks2,
          'kodeMK3' => $request->kodeMK3,
          'matkul3' => $request->matkul3,
          'sks3' => $request->sks3,
          'kodeMK4' => $request->kodeMK4,
          'matkul4' => $request->matkul4,
          'sks4' => $request->sks4,
          'kodeMK5' => $request->kodeMK5,
          'matkul5' => $request->matkul5,
          'sks5' => $request->sks5,
          'kodeMK6' => $request->kodeMK6,
          'matkul6' => $request->matkul6,
          'sks6' => $request->sks6,
          'kodeMK7' => $request->kodeMK7,
          'matkul7' => $request->matkul7,
          'sks7' => $request->sks7,
          'kodeMK8' => $request->kodeMK8,
          'matkul8' => $request->matkul8,
          'sks8' => $request->sks8,
          'kodeMK9' => $request->kodeMK9,
          'matkul9' => $request->matkul9,
          'sks9' => $request->sks9,
          'kodeMK10' => $request->kodeMK10,
          'matkul10' => $request->matkul10,
          'sks10' => $request->sks10,
          'formatsurat_id' => $request->formatsurat_id,
          'dataSurat' => $request->dataSurat
        ];
      }
      if($obj == ""){
        dd("Uncaught exception");
      }
      return json_encode($obj);
    }

    /**
    * Untuk menampilkan data yang telah diisikan pada formulir
    */
    public function tampilkanPreview(Request $request){
      $loggedInUser = Auth::user();
      // dd($loggedInUser);
      $realUser = $this->getRealUser($loggedInUser);
      $foto = $realUser->foto_mahasiswa;
      if($request->jenis_surat == "1"){
        // dd($request);
        $nama = $request->nama;
        $prodi = $request->prodi;
        $npm = $request->npm;
        $semester = $request->semester;
        $thnAkademik = $request->thnAkademik;
        $penyediabeasiswa = $request->penyediabeasiswa;
        $formatsurat_id = $request->jenis_surat;
        // dd($penyediabeasiswa);
        $dataSurat = $this->buatJSON($request);
        // dd($dataSurat);
        return view('mahasiswa.preview_keterangan_beasiswa', [
            'nama' => $nama,
            'prodi' => $prodi,
            'npm' => $npm,
            'semester' => $semester,
            'thnAkademik' => $thnAkademik,
            'penyediabeasiswa' => $penyediabeasiswa,
            'formatsurat_id' => $formatsurat_id,
            'dataSurat' => $dataSurat,
            'user' => $realUser
        ]);
      }
      else if($request->jenis_surat == "2"){
        $nama = $request->nama;
        $prodi = $request->prodi;
        $npm = $request->npm;
        $kota_lahir = $request->kota_lahir;
        $tglLahir = $request->tglLahir;
        $semester = $request->semester;
        $alamat = $request->alamat;
        $formatsurat_id = $request->jenis_surat;
        $dataSurat = $this->buatJSON($request);
        // dd($dataSurat);
        return view('mahasiswa.preview_keterangan_mahasiswa_aktif', [
            'nama' => $nama,
            'prodi' => $prodi,
            'npm' => $npm,
            'kota_lahir' => $kota_lahir,
            'tglLahir' => $tglLahir,
            'alamat' => $alamat,
            'semester' => $semester,
            'formatsurat_id' => $formatsurat_id,
            'dataSurat' => $dataSurat,
            'user' => $realUser
        ]);
      }
      else if($request->jenis_surat == "3"){
        $nama = $request->nama;
        $tglLahir = $request->tglLahir;
        $kewarganegaraan = $request->kewarganegaraan;
        $organisasiTujuan = $request->organisasiTujuan;
        $thnAkademik = $request->thnAkademik;
        $negaraTujuan = $request->negaraTujuan;
        $tanggalKunjungan = $request->tanggalKunjungan;
        $formatsurat_id = $request->jenis_surat;
        $dataSurat = $this->buatJSON($request);
        // dd($dataSurat);
        return view('mahasiswa.preview_pembuatan_visa', [
            'nama' => $nama,
            'tglLahir' => $tglLahir,
            'kewarganegaraan' => $kewarganegaraan,
            'organisasiTujuan' => $organisasiTujuan,
            'thnAkademik' => $thnAkademik,
            'negaraTujuan' => $negaraTujuan,
            'tanggalKunjungan' => $tanggalKunjungan,
            'formatsurat_id' => $formatsurat_id,
            'dataSurat' => $dataSurat,
            'user' => $realUser
        ]);
      }
      else if($request->jenis_surat == "4"){
        $nama = $request->nama;
        $npm = $request->npm;
        $prodi = $request->prodi;
        $matkul = $request->matkul;
        $topik = $request->topik;
        $organisasi = $request->organisasi;
        $alamatOrganisasi = $request->alamatOrganisasi;
        $keperluanKunjungan = $request->keperluanKunjungan;
        $kota = $request->kota;
        $kepada = $request->kepada;
        $formatsurat_id = $request->jenis_surat;
        $dataSurat = $this->buatJSON($request);
        // dd($request);
        return view('mahasiswa.preview_izin_studi_lapangan_1org', [
            'nama' => $nama,
            'npm' => $npm,
            'prodi' => $prodi,
            'matkul' => $matkul,
            'topik' => $topik,
            'organisasi' => $organisasi,
            'alamatOrganisasi' => $alamatOrganisasi,
            'keperluanKunjungan' => $keperluanKunjungan,
            'kota' => $kota,
            'kepada' => $kepada,
            'formatsurat_id' => $formatsurat_id,
            'dataSurat' => $dataSurat,
            'user' => $realUser
        ]);
      }
      else if($request->jenis_surat == "5"){
        $nama = $request->nama;
        $npm = $request->npm;
        $prodi = $request->prodi;
        $matkul = $request->matkul;
        $topik = $request->topik;
        $organisasi = $request->organisasi;
        $alamatOrganisasi = $request->alamatOrganisasi;
        $keperluanKunjungan = $request->keperluanKunjungan;
        $kota = $request->kota;
        $kepada = $request->kepada;
        $namaAnggota = $request->namaAnggota;
        $npmAnggota = $request->npmAnggota;
        $formatsurat_id = $request->jenis_surat;
        $dataSurat = $this->buatJSON($request);
        return view('mahasiswa.preview_izin_studi_lapangan_2org', [
            'nama' => $nama,
            'npm' => $npm,
            'prodi' => $prodi,
            'matkul' => $matkul,
            'topik' => $topik,
            'organisasi' => $organisasi,
            'alamatOrganisasi' => $alamatOrganisasi,
            'keperluanKunjungan' => $keperluanKunjungan,
            'kota' => $kota,
            'kepada' => $kepada,
            'namaAnggota' => $namaAnggota,
            'npmAnggota' => $npmAnggota,
            'formatsurat_id' => $formatsurat_id,
            'dataSurat' => $dataSurat,
            'user' => $realUser
        ]);
      }
      else if($request->jenis_surat == "6"){
        $nama = $request->nama;
        $npm = $request->npm;
        $prodi = $request->prodi;
        $matkul = $request->matkul;
        $topik = $request->topik;
        $organisasi = $request->organisasi;
        $alamatOrganisasi = $request->alamatOrganisasi;
        $keperluanKunjungan = $request->keperluanKunjungan;
        $kota = $request->kota;
        $kepada = $request->kepada;
        $namaAnggota1 = $request->namaAnggota1;
        $npmAnggota1 = $request->npmAnggota1;
        $namaAnggota2 = $request->namaAnggota2;
        $npmAnggota2 = $request->npmAnggota2;
        $formatsurat_id = $request->jenis_surat;
        $dataSurat = $this->buatJSON($request);
        return view('mahasiswa.preview_izin_studi_lapangan_3org', [
            'nama' => $nama,
            'npm' => $npm,
            'prodi' => $prodi,
            'matkul' => $matkul,
            'topik' => $topik,
            'organisasi' => $organisasi,
            'alamatOrganisasi' => $alamatOrganisasi,
            'keperluanKunjungan' => $keperluanKunjungan,
            'kota' => $kota,
            'kepada' => $kepada,
            'namaAnggota1' => $namaAnggota1,
            'npmAnggota1' => $npmAnggota1,
            'namaAnggota2' => $namaAnggota2,
            'npmAnggota2' => $npmAnggota2,
            'formatsurat_id' => $formatsurat_id,
            'dataSurat' => $dataSurat,
            'user' => $realUser
        ]);
      }
      else if($request->jenis_surat == "7"){
        $nama = $request->nama;
        $npm = $request->npm;
        $prodi = $request->prodi;
        $matkul = $request->matkul;
        $topik = $request->topik;
        $organisasi = $request->organisasi;
        $alamatOrganisasi = $request->alamatOrganisasi;
        $keperluanKunjungan = $request->keperluanKunjungan;
        $kota = $request->kota;
        $kepada = $request->kepada;
        $namaAnggota1 = $request->namaAnggota1;
        $npmAnggota1 = $request->npmAnggota1;
        $namaAnggota2 = $request->namaAnggota2;
        $npmAnggota2 = $request->npmAnggota2;
        $namaAnggota3 = $request->namaAnggota3;
        $npmAnggota3 = $request->npmAnggota3;
        $formatsurat_id = $request->jenis_surat;
        $dataSurat = $this->buatJSON($request);
        return view('mahasiswa.preview_izin_studi_lapangan_4org', [
            'nama' => $nama,
            'npm' => $npm,
            'prodi' => $prodi,
            'matkul' => $matkul,
            'topik' => $topik,
            'organisasi' => $organisasi,
            'alamatOrganisasi' => $alamatOrganisasi,
            'keperluanKunjungan' => $keperluanKunjungan,
            'kota' => $kota,
            'kepada' => $kepada,
            'namaAnggota1' => $namaAnggota1,
            'npmAnggota1' => $npmAnggota1,
            'namaAnggota2' => $namaAnggota2,
            'npmAnggota2' => $npmAnggota2,
            'namaAnggota3' => $namaAnggota3,
            'npmAnggota3' => $npmAnggota3,
            'formatsurat_id' => $formatsurat_id,
            'dataSurat' => $dataSurat,
            'user' => $realUser
        ]);
      }
      else if($request->jenis_surat == "8"){
        $nama = $request->nama;
        $npm = $request->npm;
        $prodi = $request->prodi;
        $matkul = $request->matkul;
        $topik = $request->topik;
        $organisasi = $request->organisasi;
        $alamatOrganisasi = $request->alamatOrganisasi;
        $keperluanKunjungan = $request->keperluanKunjungan;
        $kota = $request->kota;
        $kepada = $request->kepada;
        $namaAnggota1 = $request->namaAnggota1;
        $npmAnggota1 = $request->npmAnggota1;
        $namaAnggota2 = $request->namaAnggota2;
        $npmAnggota2 = $request->npmAnggota2;
        $namaAnggota3 = $request->namaAnggota3;
        $npmAnggota3 = $request->npmAnggota3;
        $namaAnggota4 = $request->namaAnggota4;
        $npmAnggota4 = $request->npmAnggota4;
        $formatsurat_id = $request->jenis_surat;
        $dataSurat = $this->buatJSON($request);
        return view('mahasiswa.preview_izin_studi_lapangan_5org', [
            'nama' => $nama,
            'npm' => $npm,
            'prodi' => $prodi,
            'matkul' => $matkul,
            'topik' => $topik,
            'organisasi' => $organisasi,
            'alamatOrganisasi' => $alamatOrganisasi,
            'keperluanKunjungan' => $keperluanKunjungan,
            'kota' => $kota,
            'kepada' => $kepada,
            'namaAnggota1' => $namaAnggota1,
            'npmAnggota1' => $npmAnggota1,
            'namaAnggota2' => $namaAnggota2,
            'npmAnggota2' => $npmAnggota2,
            'namaAnggota3' => $namaAnggota3,
            'npmAnggota3' => $npmAnggota3,
            'namaAnggota4' => $namaAnggota4,
            'npmAnggota4' => $npmAnggota4,
            'formatsurat_id' => $formatsurat_id,
            'dataSurat' => $dataSurat,
            'user' => $realUser
        ]);
      }
      else if($request->jenis_surat == "9"){
        $nama = $request->nama;
        $npm = $request->npm;
        $prodi = $request->prodi;
        $fakultas = $request->fakultas;
        $alamat = $request->alamat;
        $cutiStudiKe = $request->cutiStudiKe;
        $alasanCutiStudi = $request->alasanCutiStudi;
        $dosenWali = $request->dosenWali;
        $semester = $request->semester;
        $thnAkademik = $request->thnAkademik;
        //upload
        $lampiran = $request->file('lampiran_CutiStudi');
        $destination_path = ('lampiran/cuti_studi/');
        $filename = $lampiran->getClientOriginalName();
        $namaDepan = explode(" ", $nama);
        $savedLampiran = ($namaDepan[0] . '_' . $namaDepan[1] . '_' .$filename);
        $lampiran->move($destination_path, $savedLampiran);

        $link = '127.0.0.1:8000/lampiran/cuti_studi/' . $filename;
        $persetujuanDosenWali = '-';
        $catatanDosenWali = '-';
        $persetujuanKaprodi = '-';
        $catatanKaprodi = '-';
        $persetujuanWDII = '-';
        $catatanWDII = '-';
        $persetujuanWDI = '-';
        $catatanWDI = '-';
        $persetujuanDekan = '-';
        $formatsurat_id = $request->jenis_surat;
        $dataSurat = $this->buatJSON($request);
        return view('mahasiswa.preview_izin_cuti_studi', [
            'nama' => $nama,
            'npm' => $npm,
            'prodi' => $prodi,
            'fakultas' => $fakultas,
            'alamat' => $alamat,
            'cutiStudiKe' => $cutiStudiKe,
            'alasanCutiStudi' => $alasanCutiStudi,
            'dosenWali' => $dosenWali,
            'semester' => $semester,
            'thnAkademik' => $thnAkademik,
            'persetujuanDosenWali' => $persetujuanDosenWali,
            'catatanDosenWali' => $catatanDosenWali,
            'persetujuanKaprodi' => $persetujuanKaprodi,
            'catatanKaprodi' => $catatanKaprodi,
            'persetujuanWDII' => $persetujuanWDII,
            'catatanWDII' => $catatanWDII,
            'persetujuanWDI' => $persetujuanWDI,
            'catatanWDI' => $catatanWDI,
            'persetujuanDekan' => $persetujuanDekan,
            'formatsurat_id' => $formatsurat_id,
            'dataSurat' => $dataSurat,
            'user' => $realUser
        ]);
      }
      else if($request->jenis_surat == "10"){
        $nama = $request->nama;
        $npm = $request->npm;
        $alamat = $request->alamat;
        $noTelepon = $request->noTelepon;
        $namaOrtu = $request->namaOrtu;
        $dosenWali = $request->dosenWali;
        $semester = $request->semester;
        //upload
        // $lampiran = $request->file('lampiran_CutiStudi');
        // $destination_path = ('lampiran/cuti_studi/');
        // $filename = $lampiran->getClientOriginalName();
        // $namaDepan = explode(" ", $nama);
        // $savedLampiran = ($namaDepan[0] . '_' . $namaDepan[1] . '_' .$filename);
        // $lampiran->move($destination_path, $savedLampiran);

        // $link = '127.0.0.1:8000/format_surat_latex/' . $filename;
        $persetujuanDosenWali = '-';
        $catatanDosenWali = '-';
        $persetujuanKaprodi = '-';
        $catatanKaprodi = '-';
        $persetujuanWDII = '-';
        $catatanWDII = '-';
        $persetujuanWDI = '-';
        $catatanWDI = '-';
        $persetujuanDekan = '-';
        $formatsurat_id = $request->jenis_surat;
        $dataSurat = $this->buatJSON($request);
        return view('mahasiswa.preview_izin_pengunduran_diri', [
            'nama' => $nama,
            'npm' => $npm,
            'alamat' => $alamat,
            'noTelepon' => $noTelepon,
            'namaOrtu' => $namaOrtu,
            'dosenWali' => $dosenWali,
            'semester' => $semester,
            'persetujuanDosenWali' => $persetujuanDosenWali,
            'catatanDosenWali' => $catatanDosenWali,
            'persetujuanKaprodi' => $persetujuanKaprodi,
            'catatanKaprodi' => $catatanKaprodi,
            'persetujuanWDII' => $persetujuanWDII,
            'catatanWDII' => $catatanWDII,
            'persetujuanWDI' => $persetujuanWDI,
            'catatanWDI' => $catatanWDI,
            'persetujuanDekan' => $persetujuanDekan,
            'formatsurat_id' => $formatsurat_id,
            'dataSurat' => $dataSurat,
            'user' => $realUser
        ]);
      }
      else if($request->jenis_surat == "11"){
        $semester = $request->semester;
        $thnAkademik = $request->thnAkademik;
        $nama = $request->nama;
        $prodi = $request->prodi;
        $npm = $request->npm;
        $namaWakil = $request->namaWakil;
        $prodiWakil = $request->prodiWakil;
        $npmWakil = $request->npmWakil;
        $dosenWali = $request->dosenWali;
        $alasan = $request->alasan;
        $kodeMK = $request->kodeMK;
        $matkul = $request->matkul;
        $sks = $request->sks;
        $formatsurat_id = $request->jenis_surat;
        $dataSurat = $this->buatJSON($request);
        return view('mahasiswa.preview_perwakilan_perwalian_1matkul', [
            'semester' => $semester,
            'thnAkademik' => $thnAkademik,
            'nama' => $nama,
            'prodi' => $prodi,
            'npm' => $npm,
            'namaWakil' => $namaWakil,
            'prodiWakil' => $prodiWakil,
            'npmWakil' => $npmWakil,
            'dosenWali' => $dosenWali,
            'alasan' => $alasan,
            'kodeMK' =>$kodeMK,
            'matkul' => $matkul,
            'sks' => $sks,
            'formatsurat_id' => $formatsurat_id,
            'dataSurat' => $dataSurat,
            'user' => $realUser
        ]);
      }
      else if($request->jenis_surat == "12"){
        $semester = $request->semester;
        $thnAkademik = $request->thnAkademik;
        $nama = $request->nama;
        $prodi = $request->prodi;
        $npm = $request->npm;
        $namaWakil = $request->namaWakil;
        $prodiWakil = $request->prodiWakil;
        $npmWakil = $request->npmWakil;
        $dosenWali = $request->dosenWali;
        $alasan = $request->alasan;
        $kodeMK1 = $request->kodeMK1;
        $matkul1 = $request->matkul1;
        $sks1 = $request->sks1;
        $kodeMK2 = $request->kodeMK2;
        $matkul2 = $request->matkul2;
        $sks2 = $request->sks2;
        $formatsurat_id = $request->jenis_surat;
        $dataSurat = $this->buatJSON($request);
        return view('mahasiswa.preview_perwakilan_perwalian_2matkul', [
            'semester' => $semester,
            'thnAkademik' => $thnAkademik,
            'nama' => $nama,
            'prodi' => $prodi,
            'npm' => $npm,
            'namaWakil' => $namaWakil,
            'prodiWakil' => $prodiWakil,
            'npmWakil' => $npmWakil,
            'dosenWali' => $dosenWali,
            'alasan' => $alasan,
            'kodeMK1' => $kodeMK1,
            'matkul1' => $matkul1,
            'sks1' => $sks1,
            'kodeMK2' => $kodeMK2,
            'matkul2' => $matkul2,
            'sks2' => $sks2,
            'formatsurat_id' => $formatsurat_id,
            'dataSurat' => $dataSurat,
            'user' => $realUser
        ]);
      }
      else if($request->jenis_surat == "13"){
        $semester = $request->semester;
        $thnAkademik = $request->thnAkademik;
        $nama = $request->nama;
        $prodi = $request->prodi;
        $npm = $request->npm;
        $namaWakil = $request->namaWakil;
        $prodiWakil = $request->prodiWakil;
        $npmWakil = $request->npmWakil;
        $dosenWali = $request->dosenWali;
        $alasan = $request->alasan;
        $kodeMK1 = $request->kodeMK1;
        $matkul1 = $request->matkul1;
        $sks1 = $request->sks1;
        $kodeMK2 = $request->kodeMK2;
        $matkul2 = $request->matkul2;
        $sks2 = $request->sks2;
        $kodeMK3 = $request->kodeMK3;
        $matkul3 = $request->matkul3;
        $sks3 = $request->sks3;
        $formatsurat_id = $request->jenis_surat;
        $dataSurat = $this->buatJSON($request);
        return view('mahasiswa.preview_perwakilan_perwalian_3matkul', [
            'semester' => $semester,
            'thnAkademik' => $thnAkademik,
            'nama' => $nama,
            'prodi' => $prodi,
            'npm' => $npm,
            'namaWakil' => $namaWakil,
            'prodiWakil' => $prodiWakil,
            'npmWakil' => $npmWakil,
            'dosenWali' => $dosenWali,
            'alasan' => $alasan,
            'kodeMK1' => $kodeMK1,
            'matkul1' => $matkul1,
            'sks1' => $sks1,
            'kodeMK2' => $kodeMK2,
            'matkul2' => $matkul2,
            'sks2' => $sks2,
            'kodeMK3' => $kodeMK3,
            'matkul3' => $matkul3,
            'sks3' => $sks3,
            'formatsurat_id' => $formatsurat_id,
            'dataSurat' => $dataSurat,
            'user' => $realUser
        ]);
      }
      else if($request->jenis_surat == "14"){
        $semester = $request->semester;
        $thnAkademik = $request->thnAkademik;
        $nama = $request->nama;
        $prodi = $request->prodi;
        $npm = $request->npm;
        $namaWakil = $request->namaWakil;
        $prodiWakil = $request->prodiWakil;
        $npmWakil = $request->npmWakil;
        $dosenWali = $request->dosenWali;
        $alasan = $request->alasan;
        $kodeMK1 = $request->kodeMK1;
        $matkul1 = $request->matkul1;
        $sks1 = $request->sks1;
        $kodeMK2 = $request->kodeMK2;
        $matkul2 = $request->matkul2;
        $sks2 = $request->sks2;
        $kodeMK3 = $request->kodeMK3;
        $matkul3 = $request->matkul3;
        $sks3 = $request->sks3;
        $kodeMK4 = $request->kodeMK4;
        $matkul4 = $request->matkul4;
        $sks4 = $request->sks4;
        $formatsurat_id = $request->jenis_surat;
        $dataSurat = $this->buatJSON($request);
        return view('mahasiswa.preview_perwakilan_perwalian_4matkul', [
            'semester' => $semester,
            'thnAkademik' => $thnAkademik,
            'nama' => $nama,
            'prodi' => $prodi,
            'npm' => $npm,
            'namaWakil' => $namaWakil,
            'prodiWakil' => $prodiWakil,
            'npmWakil' => $npmWakil,
            'dosenWali' => $dosenWali,
            'alasan' => $alasan,
            'kodeMK1' => $kodeMK1,
            'matkul1' => $matkul1,
            'sks1' => $sks1,
            'kodeMK2' => $kodeMK2,
            'matkul2' => $matkul2,
            'sks2' => $sks2,
            'kodeMK3' => $kodeMK3,
            'matkul3' => $matkul3,
            'sks3' => $sks3,
            'kodeMK4' => $kodeMK4,
            'matkul4' => $matkul4,
            'sks4' => $sks4,
            'formatsurat_id' => $formatsurat_id,
            'dataSurat' => $dataSurat,
            'user' => $realUser
        ]);
      }
      else if($request->jenis_surat == "15"){
        $semester = $request->semester;
        $thnAkademik = $request->thnAkademik;
        $nama = $request->nama;
        $prodi = $request->prodi;
        $npm = $request->npm;
        $namaWakil = $request->namaWakil;
        $prodiWakil = $request->prodiWakil;
        $npmWakil = $request->npmWakil;
        $dosenWali = $request->dosenWali;
        $alasan = $request->alasan;
        $kodeMK1 = $request->kodeMK1;
        $matkul1 = $request->matkul1;
        $sks1 = $request->sks1;
        $kodeMK2 = $request->kodeMK2;
        $matkul2 = $request->matkul2;
        $sks2 = $request->sks2;
        $kodeMK3 = $request->kodeMK3;
        $matkul3 = $request->matkul3;
        $sks3 = $request->sks3;
        $kodeMK4 = $request->kodeMK4;
        $matkul4 = $request->matkul4;
        $sks4 = $request->sks4;
        $kodeMK5 = $request->kodeMK5;
        $matkul5 = $request->matkul5;
        $sks5 = $request->sks5;
        $formatsurat_id = $request->jenis_surat;
        $dataSurat = $this->buatJSON($request);
        return view('mahasiswa.preview_perwakilan_perwalian_5matkul', [
            'semester' => $semester,
            'thnAkademik' => $thnAkademik,
            'nama' => $nama,
            'prodi' => $prodi,
            'npm' => $npm,
            'namaWakil' => $namaWakil,
            'prodiWakil' => $prodiWakil,
            'npmWakil' => $npmWakil,
            'dosenWali' => $dosenWali,
            'alasan' => $alasan,
            'kodeMK1' => $kodeMK1,
            'matkul1' => $matkul1,
            'sks1' => $sks1,
            'kodeMK2' => $kodeMK2,
            'matkul2' => $matkul2,
            'sks2' => $sks2,
            'kodeMK3' => $kodeMK3,
            'matkul3' => $matkul3,
            'sks3' => $sks3,
            'kodeMK4' => $kodeMK4,
            'matkul4' => $matkul4,
            'sks4' => $sks4,
            'kodeMK5' => $kodeMK5,
            'matkul5' => $matkul5,
            'sks5' => $sks5,
            'formatsurat_id' => $formatsurat_id,
            'dataSurat' => $dataSurat,
            'user' => $realUser
        ]);
      }
      else if($request->jenis_surat == "16"){
        $semester = $request->semester;
        $thnAkademik = $request->thnAkademik;
        $nama = $request->nama;
        $prodi = $request->prodi;
        $npm = $request->npm;
        $namaWakil = $request->namaWakil;
        $prodiWakil = $request->prodiWakil;
        $npmWakil = $request->npmWakil;
        $dosenWali = $request->dosenWali;
        $alasan = $request->alasan;
        $kodeMK1 = $request->kodeMK1;
        $matkul1 = $request->matkul1;
        $sks1 = $request->sks1;
        $kodeMK2 = $request->kodeMK2;
        $matkul2 = $request->matkul2;
        $sks2 = $request->sks2;
        $kodeMK3 = $request->kodeMK3;
        $matkul3 = $request->matkul3;
        $sks3 = $request->sks3;
        $kodeMK4 = $request->kodeMK4;
        $matkul4 = $request->matkul4;
        $sks4 = $request->sks4;
        $kodeMK5 = $request->kodeMK5;
        $matkul5 = $request->matkul5;
        $sks5 = $request->sks5;
        $kodeMK6 = $request->kodeMK6;
        $matkul6 = $request->matkul6;
        $sks6 = $request->sks6;
        $formatsurat_id = $request->jenis_surat;
        $dataSurat = $this->buatJSON($request);
        return view('mahasiswa.preview_perwakilan_perwalian_6matkul', [
            'semester' => $semester,
            'thnAkademik' => $thnAkademik,
            'nama' => $nama,
            'prodi' => $prodi,
            'npm' => $npm,
            'namaWakil' => $namaWakil,
            'prodiWakil' => $prodiWakil,
            'npmWakil' => $npmWakil,
            'dosenWali' => $dosenWali,
            'alasan' => $alasan,
            'kodeMK1' => $kodeMK1,
            'matkul1' => $matkul1,
            'sks1' => $sks1,
            'kodeMK2' => $kodeMK2,
            'matkul2' => $matkul2,
            'sks2' => $sks2,
            'kodeMK3' => $kodeMK3,
            'matkul3' => $matkul3,
            'sks3' => $sks3,
            'kodeMK4' => $kodeMK4,
            'matkul4' => $matkul4,
            'sks4' => $sks4,
            'kodeMK5' => $kodeMK5,
            'matkul5' => $matkul5,
            'sks5' => $sks5,
            'kodeMK6' => $kodeMK6,
            'matkul6' => $matkul6,
            'sks6' => $sks6,
            'formatsurat_id' => $formatsurat_id,
            'dataSurat' => $dataSurat,
            'user' => $realUser
        ]);
      }
      else if($request->jenis_surat == "17"){
        $semester = $request->semester;
        $thnAkademik = $request->thnAkademik;
        $nama = $request->nama;
        $prodi = $request->prodi;
        $npm = $request->npm;
        $namaWakil = $request->namaWakil;
        $prodiWakil = $request->prodiWakil;
        $npmWakil = $request->npmWakil;
        $dosenWali = $request->dosenWali;
        $alasan = $request->alasan;
        $kodeMK1 = $request->kodeMK1;
        $matkul1 = $request->matkul1;
        $sks1 = $request->sks1;
        $kodeMK2 = $request->kodeMK2;
        $matkul2 = $request->matkul2;
        $sks2 = $request->sks2;
        $kodeMK3 = $request->kodeMK3;
        $matkul3 = $request->matkul3;
        $sks3 = $request->sks3;
        $kodeMK4 = $request->kodeMK4;
        $matkul4 = $request->matkul4;
        $sks4 = $request->sks4;
        $kodeMK5 = $request->kodeMK5;
        $matkul5 = $request->matkul5;
        $sks5 = $request->sks5;
        $kodeMK6 = $request->kodeMK6;
        $matkul6 = $request->matkul6;
        $sks6 = $request->sks6;
        $kodeMK7 = $request->kodeMK7;
        $matkul7 = $request->matkul7;
        $sks7 = $request->sks7;
        $formatsurat_id = $request->jenis_surat;
        $dataSurat = $this->buatJSON($request);
        return view('mahasiswa.preview_perwakilan_perwalian_7matkul', [
            'semester' => $semester,
            'thnAkademik' => $thnAkademik,
            'nama' => $nama,
            'prodi' => $prodi,
            'npm' => $npm,
            'namaWakil' => $namaWakil,
            'prodiWakil' => $prodiWakil,
            'npmWakil' => $npmWakil,
            'dosenWali' => $dosenWali,
            'alasan' => $alasan,
            'kodeMK1' => $kodeMK1,
            'matkul1' => $matkul1,
            'sks1' => $sks1,
            'kodeMK2' => $kodeMK2,
            'matkul2' => $matkul2,
            'sks2' => $sks2,
            'kodeMK3' => $kodeMK3,
            'matkul3' => $matkul3,
            'sks3' => $sks3,
            'kodeMK4' => $kodeMK4,
            'matkul4' => $matkul4,
            'sks4' => $sks4,
            'kodeMK5' => $kodeMK5,
            'matkul5' => $matkul5,
            'sks5' => $sks5,
            'kodeMK6' => $kodeMK6,
            'matkul6' => $matkul6,
            'sks6' => $sks6,
            'kodeMK7' => $kodeMK7,
            'matkul7' => $matkul7,
            'sks7' => $sks7,
            'formatsurat_id' => $formatsurat_id,
            'dataSurat' => $dataSurat,
            'user' => $realUser
        ]);
      }
      else if($request->jenis_surat == "18"){
        $semester = $request->semester;
        $thnAkademik = $request->thnAkademik;
        $nama = $request->nama;
        $prodi = $request->prodi;
        $npm = $request->npm;
        $namaWakil = $request->namaWakil;
        $prodiWakil = $request->prodiWakil;
        $npmWakil = $request->npmWakil;
        $dosenWali = $request->dosenWali;
        $alasan = $request->alasan;
        $kodeMK1 = $request->kodeMK1;
        $matkul1 = $request->matkul1;
        $sks1 = $request->sks1;
        $kodeMK2 = $request->kodeMK2;
        $matkul2 = $request->matkul2;
        $sks2 = $request->sks2;
        $kodeMK3 = $request->kodeMK3;
        $matkul3 = $request->matkul3;
        $sks3 = $request->sks3;
        $kodeMK4 = $request->kodeMK4;
        $matkul4 = $request->matkul4;
        $sks4 = $request->sks4;
        $kodeMK5 = $request->kodeMK5;
        $matkul5 = $request->matkul5;
        $sks5 = $request->sks5;
        $kodeMK6 = $request->kodeMK6;
        $matkul6 = $request->matkul6;
        $sks6 = $request->sks6;
        $kodeMK7 = $request->kodeMK7;
        $matkul7 = $request->matkul7;
        $sks7 = $request->sks7;
        $kodeMK8 = $request->kodeMK8;
        $matkul8 = $request->matkul8;
        $sks8 = $request->sks8;
        $formatsurat_id = $request->jenis_surat;
        $dataSurat = $this->buatJSON($request);
        return view('mahasiswa.preview_perwakilan_perwalian_8matkul', [
            'semester' => $semester,
            'thnAkademik' => $thnAkademik,
            'nama' => $nama,
            'prodi' => $prodi,
            'npm' => $npm,
            'namaWakil' => $namaWakil,
            'prodiWakil' => $prodiWakil,
            'npmWakil' => $npmWakil,
            'dosenWali' => $dosenWali,
            'alasan' => $alasan,
            'kodeMK1' => $kodeMK1,
            'matkul1' => $matkul1,
            'sks1' => $sks1,
            'kodeMK2' => $kodeMK2,
            'matkul2' => $matkul2,
            'sks2' => $sks2,
            'kodeMK3' => $kodeMK3,
            'matkul3' => $matkul3,
            'sks3' => $sks3,
            'kodeMK4' => $kodeMK4,
            'matkul4' => $matkul4,
            'sks4' => $sks4,
            'kodeMK5' => $kodeMK5,
            'matkul5' => $matkul5,
            'sks5' => $sks5,
            'kodeMK6' => $kodeMK6,
            'matkul6' => $matkul6,
            'sks6' => $sks6,
            'kodeMK7' => $kodeMK7,
            'matkul7' => $matkul7,
            'sks7' => $sks7,
            'kodeMK8' => $kodeMK8,
            'matkul8' => $matkul8,
            'sks8' => $sks8,
            'formatsurat_id' => $formatsurat_id,
            'dataSurat' => $dataSurat,
            'user' => $realUser
        ]);
      }
      else if($request->jenis_surat == "19"){
        $semester = $request->semester;
        $thnAkademik = $request->thnAkademik;
        $nama = $request->nama;
        $prodi = $request->prodi;
        $npm = $request->npm;
        $namaWakil = $request->namaWakil;
        $prodiWakil = $request->prodiWakil;
        $npmWakil = $request->npmWakil;
        $dosenWali = $request->dosenWali;
        $alasan = $request->alasan;
        $matkul1 = $request->matkul1;
        $kodeMK1 = $request->kodeMK1;
        $matkul1 = $request->matkul1;
        $sks1 = $request->sks1;
        $kodeMK2 = $request->kodeMK2;
        $matkul2 = $request->matkul2;
        $sks2 = $request->sks2;
        $kodeMK3 = $request->kodeMK3;
        $matkul3 = $request->matkul3;
        $sks3 = $request->sks3;
        $kodeMK4 = $request->kodeMK4;
        $matkul4 = $request->matkul4;
        $sks4 = $request->sks4;
        $kodeMK5 = $request->kodeMK5;
        $matkul5 = $request->matkul5;
        $sks5 = $request->sks5;
        $kodeMK6 = $request->kodeMK6;
        $matkul6 = $request->matkul6;
        $sks6 = $request->sks6;
        $kodeMK7 = $request->kodeMK7;
        $matkul7 = $request->matkul7;
        $sks7 = $request->sks7;
        $kodeMK8 = $request->kodeMK8;
        $matkul8 = $request->matkul8;
        $sks8 = $request->sks8;
        $kodeMK9 = $request->kodeMK9;
        $matkul9 = $request->matkul9;
        $sks9 = $request->sks9;
        $formatsurat_id = $request->jenis_surat;
        $dataSurat = $this->buatJSON($request);
        return view('mahasiswa.preview_perwakilan_perwalian_9matkul', [
            'semester' => $semester,
            'thnAkademik' => $thnAkademik,
            'nama' => $nama,
            'prodi' => $prodi,
            'npm' => $npm,
            'namaWakil' => $namaWakil,
            'prodiWakil' => $prodiWakil,
            'npmWakil' => $npmWakil,
            'dosenWali' => $dosenWali,
            'alasan' => $alasan,
            'kodeMK1' => $kodeMK1,
            'matkul1' => $matkul1,
            'sks1' => $sks1,
            'kodeMK2' => $kodeMK2,
            'matkul2' => $matkul2,
            'sks2' => $sks2,
            'kodeMK3' => $kodeMK3,
            'matkul3' => $matkul3,
            'sks3' => $sks3,
            'kodeMK4' => $kodeMK4,
            'matkul4' => $matkul4,
            'sks4' => $sks4,
            'kodeMK5' => $kodeMK5,
            'matkul5' => $matkul5,
            'sks5' => $sks5,
            'kodeMK6' => $kodeMK6,
            'matkul6' => $matkul6,
            'sks6' => $sks6,
            'kodeMK7' => $kodeMK7,
            'matkul7' => $matkul7,
            'sks7' => $sks7,
            'kodeMK8' => $kodeMK8,
            'matkul8' => $matkul8,
            'sks8' => $sks8,
            'kodeMK9' => $kodeMK9,
            'matkul9' => $matkul9,
            'sks9' => $sks9,
            'formatsurat_id' => $formatsurat_id,
            'dataSurat' => $dataSurat,
            'user' => $realUser
        ]);
      }
      else if($request->jenis_surat == "20"){
        $semester = $request->semester;
        $thnAkademik = $request->thnAkademik;
        $nama = $request->nama;
        $prodi = $request->prodi;
        $npm = $request->npm;
        $namaWakil = $request->namaWakil;
        $prodiWakil = $request->prodiWakil;
        $npmWakil = $request->npmWakil;
        $dosenWali = $request->dosenWali;
        $alasan = $request->alasan;
        $kodeMK1 = $request->kodeMK1;
        $matkul1 = $request->matkul1;
        $sks1 = $request->sks1;
        $kodeMK2 = $request->kodeMK2;
        $matkul2 = $request->matkul2;
        $sks2 = $request->sks2;
        $kodeMK3 = $request->kodeMK3;
        $matkul3 = $request->matkul3;
        $sks3 = $request->sks3;
        $kodeMK4 = $request->kodeMK4;
        $matkul4 = $request->matkul4;
        $sks4 = $request->sks4;
        $kodeMK5 = $request->kodeMK5;
        $matkul5 = $request->matkul5;
        $sks5 = $request->sks5;
        $kodeMK6 = $request->kodeMK6;
        $matkul6 = $request->matkul6;
        $sks6 = $request->sks6;
        $kodeMK7 = $request->kodeMK7;
        $matkul7 = $request->matkul7;
        $sks7 = $request->sks7;
        $kodeMK8 = $request->kodeMK8;
        $matkul8 = $request->matkul8;
        $sks8 = $request->sks8;
        $kodeMK9 = $request->kodeMK9;
        $matkul9 = $request->matkul9;
        $sks9 = $request->sks9;
        $kodeMK10 = $request->kodeMK10;
        $matkul10 = $request->matkul10;
        $sks10 = $request->sks10;
        $formatsurat_id = $request->jenis_surat;
        $dataSurat = $this->buatJSON($request);
        return view('mahasiswa.preview_perwakilan_perwalian_10matkul', [
            'semester' => $semester,
            'thnAkademik' => $thnAkademik,
            'nama' => $nama,
            'prodi' => $prodi,
            'npm' => $npm,
            'namaWakil' => $namaWakil,
            'prodiWakil' => $prodiWakil,
            'npmWakil' => $npmWakil,
            'dosenWali' => $dosenWali,
            'alasan' => $alasan,
            'kodeMK1' => $kodeMK1,
            'matkul1' => $matkul1,
            'sks1' => $sks1,
            'kodeMK2' => $kodeMK2,
            'matkul2' => $matkul2,
            'sks2' => $sks2,
            'kodeMK3' => $kodeMK3,
            'matkul3' => $matkul3,
            'sks3' => $sks3,
            'kodeMK4' => $kodeMK4,
            'matkul4' => $matkul4,
            'sks4' => $sks4,
            'kodeMK5' => $kodeMK5,
            'matkul5' => $matkul5,
            'sks5' => $sks5,
            'kodeMK6' => $kodeMK6,
            'matkul6' => $matkul6,
            'sks6' => $sks6,
            'kodeMK7' => $kodeMK7,
            'matkul7' => $matkul7,
            'sks7' => $sks7,
            'kodeMK8' => $kodeMK8,
            'matkul8' => $matkul8,
            'sks8' => $sks8,
            'kodeMK9' => $kodeMK9,
            'matkul9' => $matkul9,
            'sks9' => $sks9,
            'kodeMK10' => $kodeMK10,
            'matkul10' => $matkul10,
            'sks10' => $sks10,
            'formatsurat_id' => $formatsurat_id,
            'dataSurat' => $dataSurat,
            'user' => $realUser
        ]);
      }
    }
}

\end{lstlisting}

\begin{lstlisting}[language=tex,basicstyle=\tiny,caption=AuthenticateController.php]
	<?php
//gatau nih kepake ga ya...
namespace App\Http\Controllers\api;

use JWTAuth;
use Tymon\JWTAuth\Exceptions\JWTException;
use App\Http\Controllers\Controller;
use Illuminate\Http\Request;

class AuthenticateController extends Controller
{

    public function __construct(){
      // Apply the jwt.auth middleware to all methods in this controller
      // except for the authenticate method. We don't want to prevent
      // the user from retrieving their token if they don't already have it
      $this->middleware('jwt.auth', ['except' => ['authenticate']]);
    }

    public function authenticate(Request $request)
    {
        // grab credentials from the request
        $credentials = $request->only('username', 'password');

        try {
            // attempt to verify the credentials and create a token for the user
            if (! $token = JWTAuth::attempt($credentials)) {
                return response()->json(['error' => 'invalid_credentials'], 401);
            }
        } catch (JWTException $e) {
            // something went wrong whilst attempting to encode the token
            return response()->json(['error' => 'could_not_create_token'], 500);
        }

        // all good so return the token
        return response()->json(compact('token'));
    }
}

\end{lstlisting}

\begin{lstlisting}[language=tex,basicstyle=\tiny,caption=AuthController.php]
<?php

namespace App\Http\Controllers\Auth;

use App\User;
use App\Repositories\UserRepository;
use App\Http\Requests;
use App\Http\Controllers\Controller;
use Illuminate\Support\Facades\Input;
use Illuminate\Http\Request;
use Illuminate\Support\Facades\Auth;
use Illuminate\Support\Facades\Validator;
use Illuminate\Foundation\Auth\RegistersUsers;

class RegisterController extends Controller
{
    /*
    |--------------------------------------------------------------------------
    | Register Controller
    |--------------------------------------------------------------------------
    |
    | This controller handles the registration of new users as well as their
    | validation and creation. By default this controller uses a trait to
    | provide this functionality without requiring any additional code.
    |
    */

    use RegistersUsers;

    /**
     * Where to redirect users after registration.
     *
     * @var string
     */
    protected $redirectTo = '/home';

    /**
     * Create a new controller instance.
     *
     * @return void
     */
    public function __construct()
    {
        $this->middleware('guest');
    }

    /**
     * Get a validator for an incoming registration request.
     *
     * @param  array  $data
     * @return \Illuminate\Contracts\Validation\Validator
     */
    protected function validator(array $data){
        return Validator::make($data, [
            'name' => 'required|max:255',
            'username' => 'required',
            // 'email' => 'required|email|max:255|unique:users',
            'password' => 'required|min:8|confirmed',
            'jabatan' => 'required'
        ]);
    }

    public function register(Request $request){

      $validator = validator($request->all());
      // dd($validator);
      if($validator->fails()){
        return redirect('/login')
        ->withErrors($validator)
        ->withInput([
          'username' => $request->username,
          'name' => $request->name,
          'jabatan' => $request->jabatan,
        ]);
      }
      // dd("not fail");
      $savedUser = User::create(array(
        'name' => $request->name,
        'username' => $request->username,
        'jabatan' => $request->jabatan,
        // 'email' => Input::get('email'),
        'password' => bcrypt($request->password),
        'jabatan' => $request->jabatan
      ));
      // dd($savedUser);
      return redirect('login')->with('success_message','Registrasi Berhasil');
    }

}
\end{lstlisting}

\begin{lstlisting}[language=tex,basicstyle=\tiny,caption=FormatsuratController.php]
<?php

namespace App\Http\Controllers;
use App\Repositories\TURepository;
use App\Repositories\DosenRepository;
use App\Repositories\MahasiswaRepository;
use App\Mahasiswa;
use App\Dosen;
use Validator;
use App\Http\Controllers\Controller;
use App\Http\Requests\LoginRequest;
use Illuminate\Http\Request;

class AuthController extends Controller
{
    /*
    |--------------------------------------------------------------------------
    | Registration & Login Controller
    |--------------------------------------------------------------------------
    |
    | This controller handles the registration of new users, as well as the
    | authentication of existing users. By default, this controller uses
    | a simple trait to add these behaviors. Why don't you explore it?
    |
    */


    /**
     * Where to redirect users after login / registration.
     *
     * @var string
     */
    protected $mahasiswaRepo;
    protected $dosenRepo;
    protected $TURepo;

    /**
     * Create a new authentication controller instance.
     *
     * @return void
     */
    public function __construct(MahasiswaRepository $mahasiswaRepo, DosenRepository $dosenRepo, TURepository $TURepo){
        $this->mahasiswaRepo = $mahasiswaRepo;
        $this->dosenRepo = $dosenRepo;
        $this->TURepo = $TURepo;
    }contains('d')contains('d')

    public function authenticate(Request $request){
      // dd($request);
      $username = $request->username;
      $password = $request->password;



    }

    public function logout(){
        Auth::logout();
        return redirect()->route('login');
    }
    /**
     * Get a validator for an incoming registration request.
     *
     * @param  array  $data
     * @return \Illuminate\Contracts\Validation\Validator
     */
    protected function validator(array $data)
    {
        return Validator::make($data, [
            'name' => 'required',
            'username' => 'required',
            'email' => 'required',
            'password' => 'required',
            'jabatan' => 'required'
        ]);
    }

    /**
     * Create a new user instance after a valid registration.
     *
     * @param  array  $data
     * @return User
     */
    protected function create(array $data)
    {
        return User::create([
            'name' => $data['name'],
            'username' > $data['username'],
            'email' => $data['email'],
            'password' => bcrypt($data['password']),
            'jabatan' => $data['jabatan'],
        ]);
    }
}

\end{lstlisting}