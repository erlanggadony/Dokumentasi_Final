\chapter{Kesimpulan dan Saran}
\label{chap:kesimpulan_dan_saran}

Pada bab ini berisi kesimpulan yang didapat setelah melakukan pembangunan \textit{website} SIKSA serta saran-saran yang dapat digunakan untuk pengembangan penelitian selanjutnya.

\section{Kesimpulan}
\label{sec:kesimpulan}
Setelah melakukan pembangunan \textit{website} SIKSA dapat ditarik beberapa kesimpulan dalam penelitian ini, yaitu :
\begin{enumerate}
	\item Telah berhasil mengidentifikasikan semua surat akademik bersama dengan kebutuhannya. Pada saat dilakukan survei didapat ada 10 surat akademik yang disediakan oleh bagian TU FTIS. Namun karena tidak semua surat tersebut menghasilkan surat yang kemudian akan dikembalikan kepada mahasiswa yang bersangkutan untuk kemudian disampaikan kepada lembaga yang membutuhkan surat tersebut. Maka didapat 6 surat yang kemudian menjadi 7 surat karena adanya surat dengan fungsi ganda.
	\item Telah berhasil mengidentifikasi dan mempelajari setiap \textit{template} surat akademik yang disediakan oleh bagian TU FTIS dan mengimplementasikan setiap \textit{template} surat akademik tersebut ke dalam format \LaTeX. Untuk beberapa format surat ada yang memiliki lebih dari 1 \textit{template} surat akademik dikarenakan ada beberapa perbedaan pada \textit{input}.
	\item Telah berhasil mengimplementasikan pembuatan surat akademik secara otomatis berdasarkan \textit{template} yang telah ditentukan. \textit{Website} yang telah dibangun mengimplementasikan konsep \textit{mailmerge} pada \LaTeX.
	\item Membangun \textit{website} yang dapat memproduksi surat akademik menggunakan \textit{framework} Laravel dan \textit{css bootstrap}. \textit{Website} yang dibangun mengadaptasi formulir isian yang sebelumnya telah digunakan sehingga tidak akan mempersulit bagi pengguna apabila hendak melakukan pemesanan surat.

\end{enumerate}

\section{Saran}
\label{sec:saran}
\textit{Website} SIKSA yang dibangun ini masih dapat dikembangkan lagi. Berdasarkan analisis \textit{website} yang dirancang, ada beberapa saran yang dapat diberikan untuk mengembangkan sistem ini, yaitu:
\begin{enumerate}
	\item Menambahkan fitur \textit{upload} data mahasiswa untuk menambahkan mahasiswa ke \textit{database}.
	\item Menambahkan foto dosen dan petugas TU pada bagian profil pengguna.
	\item Apabila \textit{website} ini hendak digunakan oleh pihak selain Fakultas Teknologi Informasi dan Sains, sebaiknya relasi TU dan fakultas pada \textit{database} dibuat terhubung agar TU hanya dari fakultas bersangkutan yang dapat melakukan pengubahan isi pada \textit{database}.
	\item Menambahkan \textit{field} kategori pada tabel format surat untuk mempermudah dalam mengkategorikan surat pada halaman "Pilih Kategori Surat" di menu milik mahasiswa.
	\item Apabila \textit{website} ini hendak digunakan oleh pihak lain selain Fakultas Teknologi Informasi dan Sains, perlu dibuatkan koneksi antara perangkat lunak dengan \textit{database} pusat yang akan digunakan.
	\item Menambahkan tabel beasiswa pada \textit{database} sehingga saat mahasiswa hendak memilih perusahaan penyedia beasiswa pada pembuatan surat keterangan beasiswa tidak terjadi kesalahan penulisan perusahaan.
	\item Memperbaiki fungsi \textit{search} yang masih belum berfungsi pada halaman \textit{home} pejabat, \textit{history} pejabat dan \textit{home} mahasiswa.  
\end{enumerate}